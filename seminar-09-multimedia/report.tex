% реферат по 9 семинару
% компиляция через pdflatex + biber!!!
\documentclass{SibFU-docs}
\usepackage{newtxtext}
\usepackage{newtxmath}

% графика и математика
\usepackage{graphicx}
\usepackage{amsmath}

% листинги кода
\usepackage{listings}
\lstset{basicstyle=\ttfamily\small,breaklines=true,frame=single}

% таблицы
\usepackage{booktabs}
\usepackage{multirow}
\usepackage{array}

% подписи и float-окружения
\usepackage{caption}
\usepackage{subcaption}
\usepackage{float}

% микротипография
\usepackage[expansion=false]{microtype}

% гиперссылки
\usepackage{hyperref}
\hypersetup{
    colorlinks=true,
    linkcolor=black,
    citecolor=blue,
    urlcolor=blue,
    pdfauthor={Студенты ГФ25-02Б},
    pdftitle={Мультимедиа: обзор и технологии}
}

% ===== библиография =====
\addbibresource{report.bib}

% ===== условные определения шрифтовых команд для избежания ошибок с fontspec =====
\makeatletter
\@ifundefined{setmainfont}{\newcommand{\setmainfont}[2][]{}}{}
\@ifundefined{setsansfont}{\newcommand{\setsansfont}[2][]{}}{}
\@ifundefined{setmonofont}{\newcommand{\setmonofont}[2][]{}}{}
\makeatother

\begin{document}

% ----------------- титульный лист -----------------
\makecovertitle
{Гуманитарный институт}
{Кафедра прикладной информатики в искусстве и интерактивных медиа}
{МУЛЬТИМЕДИА: ТЕОРИЯ, ТЕХНОЛОГИИ И ПРАКТИКА}
{ГФ25-02Б}

% ----------------- аннотация -----------------
\begin{abstract}
Реферат представляет целостный обзор современного понятия мультимедиа как синергетической интеграции разнородных цифровых данных в интерактивной среде. Работа структурирована по шести аналитическим блокам: концептуальное переосмысление мультимедиа и роль дигитализации \parencite{manovich2002,jenkins2006,cyberleninka2023}; типы и модели представления данных с когнитивными и психолого-педагогическими основаниями \parencite{paivio-dual-coding,mayer-principles,web115,web117}; инструментарий создания и редактирования контента в графике, видео, аудио, анимации и 3D \parencite{skypro,lifehacker,dtf,mediascope}; кодеки и контейнеры как фундамент совместимости и эффективности передачи \parencite{richardson_video_codec_design,probell_brief_history_video_coding,richardson_overview_h264}; применение мультимедиа в образовательной и коммуникационной среде с анализом когнитивных принципов и трендов \parencite{unesco-ict,horizon-report,clark-media,sweller-cognitive-load}; правовые и этические параметры использования контента \parencite{gavrilov2019,vasilev2020,kuznetsova2022,belov2018}. В тексте сопоставляются технологические, педагогические, правовые и социокультурные аспекты, формируя междисциплинарное представление о мультимедиа как ядре цифровой культуры и инфраструктуры знаний.
\end{abstract}

	\tableofcontents
\listoftables


% ----------------- введение -----------------
\section{Введение}
Феномен мультимедиа эволюционировал от простого «соседства» разнородных медиакомпонентов к состоятельной парадигме интерактивных, пользовательски-ориентированных, семиотически насыщенных сред. Классические культурологические и медиа-теоретические работы \cite{manovich2002,jenkins2006} описывают переход от изолированных каналов коммуникации к конвергентным платформам, где принцип цифровой унификации (оцифровка текста, изображения, аудио, видео в универсальный двоичный код) становится условием технической совместимости и экономии транзакционных издержек.

Практическая актуальность мультимедиа обусловлена несколькими взаимосвязанными факторами: (1) масштабируемостью цифровых инфраструктур; (2) изменением моделей восприятия, где пользователь ожидает интерактивной адаптивности контента; (3) педагогическим запросом на множественные репрезентации знаний; (4) необходимостью правовой и этической устойчивости при массовом тиражировании информационных объектов. Научные и методические источники \cite{paivio-dual-coding,mayer-principles,clark-media,sweller-cognitive-load} формируют когнитивную рамку оптимального мультимедийного дизайна на основе принципа двойного кодирования, сегментации и снижения когнитивной нагрузки.

Применительно к индустриальному и прикладному контексту форматы, кодеки и протоколы \parencite{richardson_video_codec_design,probell_brief_history_video_coding,richardson_overview_h264} обеспечивают компромисс между качеством, битрейтом, задержкой и универсальностью воспроизведения. В сфере образования и глобальных коммуникаций отчёты и международные инициативы \parencite{unesco-ict,horizon-report,coursera-impact} фиксируют тренды персонализации, интеграции искусственного интеллекта, иммерсивных сред (VR/AR/MR) и метрик вовлеченности.

Правовой и этический измерения \parencite{gavrilov2019,vasilev2020,kuznetsova2022,fedorov2018} задают нормативные ограничения и стандарты добросовестности: баланс между творческой свободой, защитой персональных данных, предотвращением манипулятивных искажений и поддержанием академической честности. В работе используются источники различной природы: теоретические книги, научные статьи, профессиональные онлайн-ресурсы и нормативные материалы, что позволяет сопоставить модельные конструкции с прикладной практикой. Методология носит синтетический характер: дескриптивный анализ + сравнительные таблицы + систематизация принципов и процессов.

Цель: предложить междисциплинарную карту мультимедиа как среды интеграции технологий, коммуникаций, обучения и регулирования, выделив принципы синергии, интерактивности, когнитивной оптимизации и правовой устойчивости. Задачи: (1) уточнить современное определение; (2) типологизировать данные и модели; (3) систематизировать инструменты; (4) раскрыть роль кодеков и контейнеров; (5) проанализировать образовательные и коммуникационные применения; (6) очертить правовые и этические рамки; (7) синтезировать перспективы развития.

% ----------------- 1 вопрос -----------------
\section{Современное понимание мультимедиа как формы интеграции данных}
\subsection{Эволюция и концептуальный сдвиг}
Исторически мультимедиа обозначало комбинированное использование нескольких каналов (например, диафильм + озвучка). В современной трактовке акцент смещается к интерактивному конструированию опыта, где пользователь становится соавтором среды \cite{cyberleninka2023,web83}. Конвергенция аппаратных платформ (смартфон как узел связи, съемки, воспроизведения и вычислений) иллюстрирует переход от многообразия медиасредств к единой цифровой прослойке, поддерживающей разные типы данных.

\subsection{Роль дигитализации и конвергенции}
Дигитализация обеспечивает универсальный код представления и трансфера: текст, аудио, графика, видео редуцируются к бинарным последовательностям, что ликвидирует барьеры формата и ускоряет трансформацию сценариев использования \cite{manovich2002}. Конвергенция технологий — слияние вычислительных, сетевых и сенсорных компонент — формирует условия для синергетической интеграции, расширяя когнитивный и эмоциональный спектр взаимодействия.

\subsection{Типы данных и их функциональная комплементарность}
Текст задает семантический каркас (контекст, метаданные, навигация); графика — визуальную структуру и внимание; аудио — эмоционально-подсознательный канал; видео и анимация — динамику процессов; интерактивные элементы — агентность пользователя. Синергия возникает при принципиальной \emph{комплементарности}: каждый тип раскрывает слой смысла, усиливая другие, что соответствует принципу множественных представлений \cite{mayer-principles,paivio-dual-coding}.

\subsection{Принципы интеграции}
\begin{itemize}
	\item \textbf{Синхронизация}: временное согласование модальностей (лицевая анимация + речевой трек).
	\item \textbf{Комплементарность}: распределение ролей медиа в раскрытии идеи.
	\item \textbf{Гипермедийность}: нелинейная сеть узлов и связей \cite{web107}.
	\item \textbf{Имерсивность}: эффект присутствия (VR/AR/MR) через пространственный звук, панорамное видео и интерактивность.
	\item \textbf{Интерактивность}: спектр от базовой навигации до контентогенерации (игровые среды).
\end{itemize}

\subsection{Уровни интерактивности}
Низкий (пауза/перемотка), средний (выбор маршрутов или сюжетных веток), высокий (создание или модификация мира). Парадигма участника повышает мотивацию, укрепляет ощущение контроля и усиливает долгосрочное запоминание \cite{clark-media}.

\subsection{Сферы применения}
Образование (симуляторы, виртуальные лаборатории), искусство (генеративные инсталляции), коммуникации (социальные платформы), развлечения (игровые движки как вершина интеграции модальностей), маркетинг (3D-конфигураторы). Каждая сфера переопределяет UX через адаптацию интерактивности и персонализации \cite{horizon-report}.

\subsection{Итог понимания}
Современное мультимедиа — интерактивная целостная среда, порождаемая синергетической интеграцией гетерогенных цифровых данных. Его функциональная цель — формирование богатого пользовательского опыта (UX), в котором пользователь переходит от модели потребления к модели совместного конструирования значения.

\subsection{Теоретические модели и формализация}
Для более строгой постановки задачи интеграции мультимедийных модальностей полезно ввести формальные модели, сопоставляющие информационные потоки и когнитивные ресурсы. Пускай каждая модальность $m\in\{\text{text},\text{image},\text{audio},\text{video}\}$ генерирует информационный поток с энтропией $H(m)$ (в битах/с). Тогда общая информационная нагрузка на потребителя в момент времени может быть приближена суммой взвешенных энтропий:
$$L(t) = \sum_m w_m(t)\,H(m,t),$$
где веса $w_m(t)\in[0,1]$ отражают относительную важность модальности в текущем фрагменте (например, синтез речи может иметь больший вес при лекции, графика — при визуализации данных). Практическая задача проектирования мультимедиа состоит в том, чтобы обеспечить $L(t)$ ниже порога рабочей памяти пользователя $C_{wm}$ на протяжении критических фрагментов, и при этом максимизировать показатель полезной передачи информации $U$.

Эта идея даёт методологию для принятия проектных решений: при подготовке лекционного видео следует уменьшать $H(\text{video})$ (снижать визуальную сложность или частоту смены кадров) в тех местах, где одновременно подаётся насыщенная текстовая или аудиальная информация.

% ----------------- 2 вопрос -----------------
\section{Типы мультимедийных данных, модели представления и когнитивные принципы}
\subsection{Текст как мультимедийный компонент}
Текст перестает быть изолированным каналом и интегрируется как динамический слой: гиперссылки, субтитры, адаптивные подсказки, семантические аннотации. Мультимедийный текст формирует поликодовую структуру, где вербальный и невербальный компоненты совместно оптимизируют передачу смысла \cite{web9}. Принцип согласованности требует исключения нерелевантных элементов для снижения когнитивной нагрузки \cite{web115,web117}.

\subsection{Аудио}
Аудио усиливает эмоциональную модальность, задает ритм и глубину контекста. Оно не автономно, а действует как слой, синхронизируемый с визуальными компонентами по времени и событийному триггеру. В обучающих системах голосовое озвучивание уменьшает визуальную перегрузку, перераспределяя когнитивную нагрузку между каналами \cite{paivio-dual-coding}.

\subsection{Видео и анимация}
Видео — последовательная визуальная репрезентация движения и процесса. Характеристики (частота кадров, битрейт) определяют формальные пределы плавности и детализации. Анимация создает иллюзию движения через серию кадров, расширяя выразительность и снижая абстрактность сложных концептов \cite{web11,web12}. Отличие анимации от мультипликации — степень трансформации уже присутствующих визуальных элементов.

\subsection{Графика: типология}
Статические изображения обеспечивают фиксацию состояния и структуру. Векторная графика масштабируется без потерь (геометрическое описание примитивов) \cite{web35,web36}; растровая — пиксельная сетка с высокой фотореалистичностью и зависимостью от разрешения \cite{web26,web27}. Фрактальная графика демонстрирует самоподобие и бесконечный уровень детализации \cite{web54,web56}. Линейная графика организует форму через разнообразие типов линий \cite{web63,web66}. 3D-графика вводит глубину, освещение, материалы, камеры, анимацию \cite{web72,web73}.

\subsection{Модели представления мультимедиа}
\begin{itemize}
	\item \textbf{Линейная}: последовательно организованный поток (видео-лекция) \cite{web91}.
	\item \textbf{Нелинейная}: интерактивное ветвление, пользовательская навигация \cite{web91}.
	\item \textbf{Гипермедийная}: сеть узлов различных типов данных, ассоциативная структура \cite{web107}.
	\item \textbf{Объектная}: модульные медиакомпоненты с атрибутами и поведением, удобные для повторного использования.
\end{itemize}

\subsection{Принципы взаимодействия и когнитивные основания}
Двойное кодирование (Paivio) \cite{paivio-dual-coding} обосновывает преимущества совмещения визуального и вербального каналов; модальность распределяет нагрузку между слуховыми и зрительными ресурсами; пространственная и временная смежность оптимизируют интеграцию стимулов; сегментация позволяет адаптировать темп обучения; активное вовлечение повышает устойчивость памяти \cite{mayer-principles,web115,web117}. Снижение когнитивной нагрузки соответствует модели рабочей памяти \cite{sweller-cognitive-load}.

\subsection{Модальность и мультимодальность}
Мультимодальность повышает эффективность за счет параллельной обработки в независимых системах восприятия. Баланс критичен: избыток несинхронизированных стимулов приводит к перегрузке, снижая глубину переработки (эффект рассеивания внимания).

\subsection{Техническая детальность: дискретизация и квантование (аудио)}
Аудиосигнал при оцифровке проходит два ключевых преобразования: дискретизацию по времени и квантование по амплитуде. Важные параметры — частота дискретизации $F_s$ и битовая глубина $N$. Для практических задач выбор $F_s$ и $N$ диктуется требуемым динамическим диапазоном, шумовыми характеристиками и последующей обработкой. Таблица~\ref{tab:sampling} иллюстрирует типичные сочетания и область применения.

\begin{table}[H]
\centering
\caption{Типичные конфигурации дискретизации и применение}
\label{tab:sampling}
\begin{tabular}{@{}lcc@{}}
	\toprule
Применение & Частота $F_s$ & Битовая глубина $N$ \\
\midrule
Телефония, VoIP & 8--16 kHz & 8--16 бит \\
Потребительское аудио (MP3) & 44.1 kHz & 16 бит \\
Профессиональная запись & 48--96 kHz & 24 бит \\
Архив и мастеринг & 96--192 kHz & 24--32 бит (float) \\
\bottomrule
\end{tabular}
\end{table}

Практический совет: при многократной обработке сигналов рекомендуется работать во внутреннем представлении с плавающей точкой (32-bit float), чтобы избежать накопления округлительных ошибок и клиппинга на промежуточных этапах.

% таблица примеров типов данных и ролей
\begin{table}[H]
	\centering
	\caption{Типы мультимедийных данных и их функциональные роли}
	\begin{tabular}{p{3cm}p{10cm}}
			\toprule
			\textbf{Тип} & \textbf{Функциональная роль} \\
		\midrule
		Текст & Семантическое ядро, навигация, метаданные, аннотации \\
		Графика & Визуализация структуры, акценты внимания, эстетика \\
		Аудио & Эмоциональная поддержка, ритм, когнитивное разгрузление \\
		Видео & Демонстрация процессов, повествовательная динамика \\
		Анимация & Уточнение движения, поэтапная визуализация сложных трансформаций \\
		3D & Пространственная ориентация, иммерсивность, симуляция \\
		Интерактив & Агентность пользователя, персонализация, адаптация \\
		\bottomrule
	\end{tabular}
\end{table}

% доп. таблица: сравнение инструментов
\begin{table}[H]
	\centering
	\caption{Сравнение популярных инструментов для мультимедиа}
	\begin{tabular}{p{3cm}p{3cm}p{4cm}p{3cm}}
			\toprule
			\textbf{Инструмент} & \textbf{Тип} & \textbf{Ключевые возможности} & \textbf{Уровень} \\
		\midrule
	Photoshop & Растровая графика & Слои, маски, корректирующие слои, фильтры & Профессиональный \\
	Illustrator & Векторная графика & Векторные примитивы, кривые Безье, экспорт SVG & Профессиональный \\
	GIMP & Растровая графика & Бесплатный аналог Photoshop, плагины & Любитель/полупрофи \\
	Inkscape & Векторная графика & Бесплатный аналог Illustrator, SVG & Любитель/полупрофи \\
	Blender & 3D/анимация & Моделирование, рендер, анимация, VFX & Профессиональный/любитель \\
	DaVinci Resolve & Видео & Многодорожечный монтаж, цветокоррекция & Профессиональный \\
	Audacity & Аудио & Запись, базовая обработка, шумоподавление & Любитель/полупрофи \\
	Audition/Pro Tools & Аудио & Профессиональное сведение и мастеринг & Профессиональный \\
	Canva & Универсальный & Шаблоны, быстрый дизайн, коллаборация & Новички/маркетинг \\
	Figma & UI/UX & Прототипирование, совместная работа & Дизайн/команда \\
	\bottomrule
	\end{tabular}
\end{table}

% таблица: кодеки и контейнеры
\begin{table}[H]
	\centering
	\caption{Популярные кодеки и контейнеры: преимущества и ограничения}
	\begin{tabular}{p{3cm}p{4cm}p{5cm}p{3cm}}
			\toprule
			\textbf{Название} & \textbf{Тип} & \textbf{Преимущества} & \textbf{Ограничения} \\
		\midrule
	H.264 & Видео кодек & Широкая поддержка, хорошая компрессия & Патентные лицензии, устарение в 4K/8K \\
	H.265 (HEVC) & Видео кодек & Улучшенная компрессия (~50\% vs H.264) & Сложные лицензии, частичная поддержка \\
	VP9 & Видео кодек & Открытый, эффективен для веб & Требует больше CPU для кодирования \\
	AV1 & Видео кодек & Бесплатный и эффективный, для стриминга будущего & Высокая вычислительная сложность кодирования \\
	MP4 & Контейнер & Универсальность, поддержка DRM & Ограниченные расширенные функции по сравнению с MKV \\
	MKV & Контейнер & Очень гибкий, поддерживает множество дорожек & Не всегда совместим с мобильными плеерами \\
	WebM & Контейнер & Оптимизирован для веб (VP9/AV1) & Меньшая совместимость с десктопными проигрывателями \\
	Opus & Аудио кодек & Отлично для речи и музыки, адаптивность & Не всегда поддерживается в старом оборудовании \\
	FLAC & Аудио (lossless) & Без потерь, хорош для архивов & Больший размер по сравнению с lossy \\
	\bottomrule
	\end{tabular}
\end{table}

% таблица: сравнение образовательных форм
\begin{table}[H]
	\centering
	\caption{Формы мультимедийного обучения и их дидактические функции}
	\begin{tabular}{p{4cm}p{6cm}p{4cm}}
			\toprule
			\textbf{Форма} & \textbf{Функция} & \textbf{Преимущества} \\
		\midrule
	Интерактивные симуляторы & Практика и отработка навыков & Безопасность, репликация сценариев, мгновенная обратная связь \\
	Видеолекции & Передача знаний, повествование & Масштабируемость, асинхронность, визуализация \\
	VR/AR модули & Иммерсивная тренировка & Глубокая вовлечённость, пространственное обучение \\
	Микрообучение (короткие модули) & Быстрая передача навыков & Соответствие вниманию современных пользователей \\
	Интерактивные тесты и адаптивные курсы & Оценка и персонализация & Персонализация траекторий обучения, адаптация сложности \\
	\bottomrule
	\end{tabular}
\end{table}

% ----------------- 3 вопрос -----------------
\section{Инструменты создания, редактирования и обработки мультимедийного контента}
\subsection{Графика}
Профессиональные редакторы (Photoshop, Illustrator) сохранили статус отраслевых стандартов, однако экосистема дополнилась доступными альтернативами (Affinity, GIMP) и облачными платформами быстрой композиции (Canva, Photopea). Векторный дизайн важен для масштабируемых идентификационных объектов (логотипы, интерфейсные иконки), растровый — для фотокоррекции и художественной живописи. Инструменты UI/UX (Figma) обеспечивают коллаборативное прототипирование \cite{lifehacker,skypro}.

\subsection{Видео}
Профессиональные монтажные среды (Premiere Pro, DaVinci Resolve, Final Cut Pro) интегрируют цветокоррекцию, многодорожечный монтаж, VFX и управление медиатеками. Любительский сегмент (CapCut, Filmora, Kdenlive) предоставляет шаблоны и упрощенный интерфейс, снижая порог входа для контент-креаторов.

\subsection{Аудио}
DAW (Digital Audio Workstations) — Audition, Ableton Live, FL Studio — объединяют запись, редактирование, сведение и мастеринговые цепочки с использованием плагин-архитектуры (VST/AU). Бесплатные решения (Audacity) сохраняют значимость для базовых задач (шумоподавление, обрезка) \cite{dtf}.

\subsection{Анимация и 3D}
After Effects (композитинг, моушн-дизайн), Blender (полный стек: моделирование, риггинг, симуляции, рендеринг), Animate (2D-векторная анимация для веб). Blender как открытая платформа ускоряет инновационные итерации через сообщество.

\subsection{Онлайн и универсальные платформы}
Canva, Pixlr, WeVideo, Soundtrap — примеры сервисной модели (SaaS), где вычислительная нагрузка и обновления централизованы. Ускоряется коллаборация, снижается зависимость от локальной конфигурации.

\subsection{Критерии выбора}
\begin{itemize}
	\item Уровень подготовки (порог освоения интерфейса).
	\item Тип задачи (растровая коррекция vs логотип vs цветокоррекция видео).
	\item Лицензирование и доступность (подписка, бессрочная лицензия, свободное ПО).
	\item Интеграция (поддержка форматов, экспорт в производственные пайплайны).
	\item Коллаборация (совместное редактирование, управление версиями).
\end{itemize}

\subsection{Практические рабочие цепочки}
Ниже приведены две типичные цепочки «от записи до публикации», которые иллюстрируют последовательность инструментов и ключевые контрольные точки.

\paragraph{Цепочка для учебного видео}
\begin{enumerate}
	\item Подготовка: сценарий, слайды (SVG/PDF), сториборд.
	\item Запись: видео рекордер (ProRes), микрофоны (48 kHz, 24 bit), запасные дорожки.
	\item Монтаж: Premiere/Resolve — организация таймлайнов, rough cut, fine cut.
	\item Аудио: экспорт отдельных дорожек в DAW, обработка (шумоподавление, эквалайзер, компрессия), мастеринг.
	\item Цветокоррекция: DaVinci Resolve — базовая коррекция, LUTs, финальный рендер мастер-файла.
	\item Конвертация: FFmpeg — экспорт в H.264/H.265/AV1 для веб; генерация субтитров и метаданных.
	\item Публикация: загрузка на LMS/CDN, тесты на устройствах.
\end{enumerate}

\paragraph{Цепочка для подкаста и аудиоматериалов}
\begin{enumerate}
	\item Подготовка: план выпуска, вопросы, тайминги, шоу-нотсы.
	\item Запись: многодорожечный рекордер, запасные микрофоны.
	\item Редакция/сведение: удаление шумов, вырезание пауз, эквализация, компрессия, лимитирование.
	\item Мастеринг: нормализация уровня, конвертация в MP3/Opus и сохранение lossless-версии (FLAC).
	\item Публикация: загрузка на хостинг, генерация RSS/метаданных, публикация на платформах.
\end{enumerate}

\subsection{Цветовые пайплайны и управление качеством}
Цветовой пайплайн в видеопроизводстве критичен для обеспечения согласованности между устройствами. Базовые этапы: запись в лог-режиме (Log), преобразование в рабочее цветовое пространство (Rec.709 / Rec.2020), применение LUT, финальная градация и экспорт с указанием цветового профиля.

Практический чеклист по качеству медиаконтента:
\begin{itemize}
	\item Проверка синхронизации аудио/видео и отсутствие дрейфа во времени.
	\item Контроль пиковых уровней (пиковый метр) и интегрального уровня (LUFS) для стриминга.
	\item Наличие субтитров и альтернативных форматов (транскрипция).
	\item Проверка на устройствах с разной производительностью (мобильные, десктопные, SmartTV).
\end{itemize}

% ----------------- 4 вопрос -----------------
\section{Роль кодеков и контейнеров в совместимости мультимедиа}
\subsection{Разграничение понятий}
Кодек — алгоритм сжатия и восстановления (кодирование/декодирование); контейнер — структурная оболочка, агрегирующая потоки (видео, аудио, субтитры), метаданные и обеспечивающая синхронизацию. Различие критично: несовместимость может возникать либо на уровне структуры файла (контейнер), либо на уровне декодирования потока (кодек) \cite{richardson_video_codec_design}.

\subsection{Эволюция кодирования}
Историческая траектория — от ранних стандартов (H.120, H.261) к H.264 как массовому компромиссу качества/битрейта, далее к HEVC, VP9 и AV1, стремящимся увеличить эффективную компрессию при уменьшении сетевых затрат \cite{probell_brief_history_video_coding,richardson_overview_h264}. Новые поколения учитывают HDR, высокие частоты кадров, адаптивный стриминг.

\subsection{Функции для совместимости}
\begin{itemize}
	\item Сжатие (уменьшение объема для хранения/передачи).
	\item Универсальность воспроизведения (широкая поддержка в плеерах и устройствах).
	\item Гибкость (множественные аудио-дорожки, субтитры, главы в MKV/MP4).
	\item Поддержка современных фич (HDR, многоканальный звук, адаптивные сегменты).
\end{itemize}

\subsection{Аспекты эффективности}
Компромисс между степенью сжатия, вычислительной сложностью кодирования и декодирования, лицензионными ограничениями и энергопотреблением (актуально для мобильных устройств).

\subsection{Контейнеры и экосистемы}
MP4 — универсальный стандарт; MKV — гибкость и расширяемость; WebM — ориентация на открытые кодеки (VP9, AV1, Opus). AVI — пример устаревших ограничений, сниженной функциональности в современных сценариях.

\subsection{Перспективы развития}
Переход к более эффективным открытым решениям (AV1), рост аппаратного ускорения декодирования, интеграция семантических метаданных (для интеллектуального поиска и адаптации), расширение поддержки объемных форматов и объектного медиакодирования.

\subsection{Количественный подход к оценке битрейта}
Практический расчёт минимально требуемого битрейта для видео зависит от разрешения, частоты кадров и желаемого качества визуального восприятия. В упрощённой модели можно принять, что требуемый битрейт $B$ (в кбит/с) пропорционален числу пикселей на кадр $P$ и частоте кадров $F$ и обратно пропорционален коэффициенту эффективности кодека $\eta$:
$$B \approx \frac{P \cdot F \cdot q}{\eta},$$
где $q$ — эмпирический коэффициент качества (бит на пиксель), а $\eta$ отражает эффективность кодека (например, для H.264 $\eta\approx1$, для HEVC $\eta\approx1.6$–2.0 при прочих равных).

Пример: для FullHD (1920×1080) при 30 fps и желаемом среднем качестве $q=0.1$ бит/пиксель, при эффективности H.264 ($\eta=1$) получаем:
$$B \approx \frac{1920\cdot1080\cdot30\cdot0.1}{1} \approx 6\,220\,800\ \text{бит/с} \approx 6221\ \text{kbit/s}.$$ 
При использовании HEVC ($\eta=1.8$) тот же уровень качества может быть достигнут приблизительно при $\sim3456$ кбит/с.

\begin{table}[H]
\centering
\caption{Оценка битрейта для различных разрешений (приблизительно)}
\begin{tabular}{@{}lrrr@{}}
	\toprule
Разрешение & fps & H.264 (кбит/с) & HEVC/AV1 (кбит/с) \\
\midrule
720p (1280×720) & 30 & 1500--2500 & 900--1400 \\
1080p (1920×1080) & 30 & 3000--6000 & 1600--3500 \\
4K (3840×2160) & 30 & 15000--25000 & 8000--15000 \\
\bottomrule
\end{tabular}
\end{table}

Эти оценки помогают при выборе параметров кодирования для целевых платформ и ограничения пропускной способности.

% ----------------- 5 вопрос -----------------
\section{Применение мультимедиа в образовательных и коммуникационных процессах}
\subsection{Образовательные формы}
Интерактивные презентации (динамическая навигация, встроенные видео), видео- и анимационные лекции, VR/AR-модули (виртуальные экскурсии, хирургические тренажеры), симуляторы и игровые среды, онлайн-платформы курсов (Moodle, Coursera) \cite{coursera-impact,unesco-ict}. Электронные учебники расширяются гиперссылками, интегрированными тестами и мультимодальными аннотациями.

\subsection{Дидактические преимущества}
Повышение наглядности; активизация познавательной деятельности через интерактив; индивидуализация темпа; межпредметная интеграция; формирование практических навыков в безопасной среде; мотивация за счет игровой эстетики \cite{mayer-principles,paivio-dual-coding}.

\subsection{Коммуникационные применения}
Веб-конференции (Zoom, Teams) — синхронная многомодальная коммуникация; инфографика и дашборды — ускоренная аналитика; маркетинг — продуктовые видео, AR-примерочные; научные коммуникации — визуализация сложных моделей; социальные сети — краткие видео, сторис, интерактивные карты.

\subsection{Преимущества мультимедийной коммуникации}
Убедительность, скорость восприятия, преодоление языковых барьеров визуальными символами, интерактивная обратная связь, глобальный охват инфраструктурой сетей доставки.

\subsection{Проблемы и ограничения}
Техническая инфраструктура (неравенство доступа), когнитивная перегрузка, качество и достоверность контента, ресурсные затраты производства, риск снижения роли педагога, здоровьесберегающие аспекты.

\subsection{Тенденции}
Интеграция искусственного интеллекта (персонализация, генерация контента), расширение иммерсивности (XR), микрообучение, метрики вовлеченности и адаптивные сценарии на базе аналитики поведения пользователей \cite{horizon-report}.

\subsection{Педагогические сценарии: подробные примеры}
Ниже приведены развёрнутые сценарии использования мультимедиа в учебных модулях.

\paragraph{Сценарий A: Демонстрация лабораторного эксперимента}
Студент получает видеоряд с интерактивными метками (hotspots), которые раскрывают измерения, графики в реальном времени и подсказки. Техническая реализация: синхронизация видеоплеера с JSON-тайм-кодами и SVG-аннотациями. Педагогическая цель — развить практические навыки и умение интерпретировать данные.

\paragraph{Сценарий B: Диагностический модуль с адаптацией}
После диагностического теста система выстраивает индивидуальную траекторию — предлагает подборку коротких мультимедийных фрагментов для закрытия пробелов и проверяет прогресс через несколько мини-тестов. Эффективность такого подхода подтверждается исследованиями по адаптивному обучению \cite{coursera-impact}.

\section{Правовые и этические аспекты использования мультимедийных материалов}

% ----------------- 6 вопрос -----------------
\section{Правовые и этические аспекты использования мультимедийных материалов}
\subsection{Интеллектуальная собственность и авторское право}
Авторское право возникает при создании произведения без обязательной регистрации и охраняет копирование, публичный показ, распространение. Смежные права защищают исполнителей, производителей фонограмм и вещательные организации \cite{gavrilov2019,popov2021}.

\subsection{Право на изображение гражданина}
Использование фотографии/видео с человеком требует согласия, особенно коммерческого. Исключения: публичные мероприятия, общественная значимость, публичные фигуры \cite{belov2018}.

\subsection{Законные способы использования и цитирование}
Допускается добросовестное цитирование с указанием источника в научных и образовательных целях. Произведения в общественном достоянии (по истечении срока охраны) свободны для использования \cite{fedorov2018}.

\subsection{Этические измерения}
Манипуляция (искажение реальности через монтаж) подрывает доверие; приватность требует оценки контекстуальной уместности публикации; чувствительные материалы (насилие, травма) — строгая необходимость общественного интереса и предупредительные маркировки \cite{kuznetsova2022}. Плагиат разрушает академическую добросовестность и репутацию \cite{novikova2019}.

\subsection{Баланс свободы и ответственности}
Создатель контента балансирует между инновационностью, правами субъекта данных и достоверностью. Этика становится инструментом долгосрочной устойчивости цифровых экосистем.

\subsection{Подробные юридические сценарии и рекомендации}
\paragraph{Сценарий: использование чужих видеоматериалов в образовательных целях}
В ряде юрисдикций допускается использование фрагментов произведений для целей образования (fair use/fair dealing), но ограничения и критерии варьируются. Рекомендация: использовать минимально необходимый фрагмент, указывать источник, по возможности запросить лицензию или использовать материалы с лицензией Creative Commons.

\paragraph{Сценарий: обработка персональных данных в видеороликах}
Если видеоролик содержит распознаваемые изображения людей, требуется оценить правовую основу обработки: согласие, законный интерес и т.п. В учебных проектах предпочтительнее получать письменные согласия и предусматривать механизмы удаления материалов по требованию.

\paragraph{Сценарий: генеративный контент и deepfake}
Генеративные модели создают правовые и этические риски — манипуляция, дезинформация, нарушение прав на изображение. Рекомендация: маркировать синтетический контент, хранить provenance-метаданные и использовать watermarks или blockchain-метки для подтверждения происхождения.

\subsection{Этические чеклисты для практики}
\begin{itemize}
	\item Оценить риски вреда и приватности перед публикацией.
	\item Предоставить альтернативные форматы для людей с ограниченными возможностями.
	\item Обеспечить прозрачность происхождения контента (атрибуция, лицензии).
	\item Разработать политику реагирования на жалобы и удаления контента.
\end{itemize}

% ----------------- заключение -----------------
\section{Заключение}
Представленный междисциплинарный анализ мультимедиа выявил его системную роль как инфраструктуры когнитивной, коммуникативной и культурной трансформации цифрового общества. Интеграция типов данных опирается на принципы синхронизации, комплементарности, гипермедийности и интерактивности; когнитивная оптимизация достигается использованием двойного кодирования, сегментации и управления модальностями; технологическая совместимость обеспечивается эволюцией кодеков и контейнеров; образование и коммуникации получают инструменты визуализации, иммерсивности и персонализации; правовые и этические рамки устанавливают границы ответственного использования.

Перспективные направления: (1) семантически обогащенные форматы с интеграцией онтологий для адаптивного представления контента; (2) объектно-ориентированное кодирование пространственных медиа; (3) AI-оптимизация мультимодального дизайна с учетом когнитивных метрик; (4) расширение нормативных моделей для генеративного контента и синтетических медиа; (5) устойчивые практики микролёрнинга и XR-экосистемы.

Мультимедиа выступает не добавочным слоем цифровой среды, а её организующим принципом, переводящим структурную сложность информации в доступный, эмоционально и семантически насыщенный пользовательский опыт. Понимание системной взаимосвязи технологических, когнитивных и правовых аспектов создаёт основу для ответственных инноваций и устойчивого развития информационных экосистем.

% ----------------- список литературы -----------------
\printbibliography

\end{document}
