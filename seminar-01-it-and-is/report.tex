% реферата по теме "Информационные технологии и информационные системы"
\documentclass{SibFU-docs}

\addbibresource{report.bib}

\begin{document}

% титульный лист
\makecovertitle{Гуманитарный институт}{Прикладная информатика в искусстве и интерактивных медиа}{Информационные технологии и информационные системы: обзор и взаимосвязи}{ГФ25-02Б}

% аннотация
\begin{abstract}
В работе представлен систематизированный обзор взаимосвязей между информационными технологиями (ИТ) и информационными системами (ИС). Рассмотрены ключевые компоненты ИТ, архитектуры ИС, роли данных и аналитики, классификации систем, организационные эффекты и практические рекомендации по выбору и внедрению. Документ опирается на учебную и научную литературу (см. список использованных источников) и представляет собой сжатую, но глубокую компиляцию материалов для понимания предметной области и принятия управленческих решений.
\end{abstract}

\tableofcontents

\section{Введение}
В последние десятилетия роль информационных технологий и информационных систем в организациях кардинально возросла. Понимание различий и взаимосвязей между ИТ (технологическая база: оборудование, программное обеспечение, сети) и ИС (организационные системы для поддержки процессов и принятия решений) важно как для исследователя, так и для практикующего менеджера \cite{turban2018information, laudon2020mis, rainer2021introduction}.

Цель данного реферата — собрать, систематизировать и изложить материалы по подтемам:
\begin{itemize}
  \item компоненты ИТ и их функции;
  \item программное обеспечение и платформы;
  \item сети и средства связи;
  \item данные и аналитика;
  \item люди, процессы и власть;
  \item проектные процессы и алгоритмы.
\end{itemize}

Источники в `report.bib` были использованы для подкрепления теоретических положений и практических рекомендаций: классические учебники по MIS и IS \cite{laudon2020mis, oleary2013bis, stair2017principles}, исследования по ценности ИТ для бизнеса \cite{melville2004information, bharadwaj2013digital}, и современные обзоры по аналитике и БИ \cite{sharda2020business, chen2012business}. Также использовались исследования по стратегическому управлению ИС и интеграции ИТ в бизнес-процессы \cite{pearlson2016strategic, ward2016strategic}.

Ниже каждая подтема раскрыта в отдельном разделе; материал объединяет конспекты и источники, расширяя их аналитикой и примерами применения.

\section{Литературный обзор}
Обзор литературы показывает несколько устойчивых направлений исследований: оценка ценности ИТ для бизнеса и теория ресурсов \cite{melville2004information, wade2004review}, модели успеха ИС \cite{delone2003updated}, стратегическое управление ИС \cite{pearlson2016strategic, ward2016strategic} и практики по аналитике и BI \cite{davenport2006competing, sharda2020business}. Вклад классических учебников обеспечивает базовую терминологию и систематизацию \cite{laudon2020mis, turban2018information, oleary2013bis}.

Критический анализ литературы помогает выделить пробелы — например, недостаточная эмпирическая детализация затрат внедрения в SME-сегменте и ограниченные исследования по устойчивому сопровождению ML-моделей в промышленной эксплуатации \cite{sambamurthy2003shaping, arnott2005journey}.

\newpage

\section{1. Компоненты и функции ИТ}
Информационные технологии включают аппаратную платформу, программное обеспечение, сетевую инфраструктуру, хранилища данных и сервисы. Аппаратная часть обеспечивает вычислительные ресурсы и хранение; программное обеспечение реализует логику приложений и платформ; сети и протоколы предоставляют каналы связи; а данные и хранилища служат источником для аналитики и отчетности \cite{turban2018information, bocij2015business}.

Внутри каждой составляющей существуют уровни: аппаратный (серверы, дисковые подсистемы, сети), платформенный (ОС, виртуализация, СУБД), прикладной (ERP/CRM/SCM), и аналитический (ETL, DWH, BI). Для обеспечения масштабируемости и отказоустойчивости применяются архитектурные приёмы — кластеры, балансировка нагрузки и горизонтальное масштабирование. Правильная организация этих уровней определяет эффективность эксплуатации ИТ и экономическую эффективность инвестиций \cite{earl1989management, davenport1998working}.

Ключевые функции ИТ:
\begin{enumerate}
  \item Поддержка операций — транзакционные и сопутствующие задачи (OLTP-системы) \cite{oz2008management}.
  \item Автоматизация управленческих процессов — ERP, BPM \cite{monk2013concepts, davenport1998working}.
  \item Аналитика и принятие решений — DWH, BI, DSS \cite{power2007decision, davenport2006competing}.
  \item Коммуникация и совместная работа — почта, мессенджеры, корпоративные платформы \cite{valacich2020essentials}.
\end{enumerate}

Архитектуры: централизованные, распределённые, клиент-серверные и облачные. Переход к облаку и микросервисной архитектуре предоставляет гибкость, но предъявляет требования к безопасности и интеграции \cite{pearlson2016strategic, ward2016strategic}.

Практический уровень: при выборе архитектуры команды обычно сопоставляют требования по доступности (SLA), скорость выхода новых функций и наличие компетенций у сотрудников. Компромисс между on-premise и облаком часто реализуется через гибридные архитектуры, где критичные данные остаются локально, а сервисы с высокой потребностью в масштабировании выносятся в облако.

\subsection{Виртуализация и контейнеризация}
Виртуализация (hypervisor) и контейнеризация (Docker, Kubernetes) позволяют экономить ресурсы и упрощают деплоймент. Контейнеры особенно удобны для микросервисной архитектуры, но требуют оркестрации для управления зависимостями и масштабированием.

\subsection{Практические рекомендации по виртуализации и контейнеризации}
\begin{itemize}
  \item Использовать контейнеры для изоляции приложений и ускорения CI/CD-пайплайнов;
  \item Применять оркестраторы (Kubernetes) с учётом политики безопасности и сетевой сегментации;
  \item Проводить тесты на устойчивость при обновлениях (chaos engineering) для критичных сервисов.
\end{itemize}

Важно: управление конфигурациями и секретами следует выносить в специализированные хранилища (vault, cloud KMS) и автоматизировать ротацию ключей, чтобы избежать «дрейфа» конфигураций между окружениями и снизить риск утечек.

\subsection{Облачные модели и эксплуатация}
IaaS обеспечивает виртуальные машины и сеть, PaaS дополняет средой выполнения, а SaaS даёт готовый прикладной функционал. Выбор модели определяется критериями стоимости, скорости внедрения и требованиями к контролю над данными \cite{applegate2009corporate}.

Практическая рекомендация: перед массовой миграцией провести аудит "облачной готовности" приложений и данных — классификацию данных по чувствительности, выявление зависимостей между сервисами, оценку необходимой сетевой пропускной способности и план миграции. Такой аудит уменьшает вероятность непредвиденных задержек и затрат при переходе.

\subsection{Риск-матрица: инфраструктура}
\begin{table}[ht]
\centering
\begin{tabular}{|p{6cm}|p{6cm}|}
\hline
  	\textbf{Риск} & \textbf{Меры смягчения} \\
\hline
Отказ оборудования & Резервирование, горячие резервные копии, SLA с провайдером \\
\hline
Утечка данных & Шифрование, DLP, регламенты доступа, аудит \\
\hline
Проблемы интеграции & Стандартизованные API, контрактное тестирование, промежуточные слои \\
\hline
Зависимость от провайдера & Гибридная архитектура, экспорт данных, оценка возможностей миграции \\
\hline
\end{tabular}
\caption{Риск-матрица для инфраструктуры}
\end{table}

\subsection{Инфраструктурные риски и защита}
Инфраструктурные компоненты уязвимы: отказ оборудования, утечка данных и уязвимости ПО. Необходимы уровни резервирования, мониторинга и планов восстановления после катастрофы (DRP) \cite{gorry1971framework, power2002dss}.

Также важны процессы управления уязвимостями, регулярное обновление и патч-менеджмент. Архитектуры с нулевым доверием (zero-trust) и сегментация сети уменьшают риски распространения инцидентов. Реализация этих мер требует сочетания технологий, процессов и обучения персонала \cite{pearlson2016strategic}.

Практическая мера: использовать поэтапные (canary) релизы и автоматические регресс-тесты при развёртывании патчей, чтобы минимизировать риск массовых сбоев. Канареечные релизы позволяют сначала проверить поведение обновления на небольшом проценте трафика, а затем постепенно расширять охват при отсутствии проблем.

\section{2. Программное обеспечение: платформы и приложения}
Программное обеспечение — это среда, в которой реализуются бизнес-функции и автоматизируются процессы. В современных условиях организации чаще используют гибридную модель: часть функционала — облачные SaaS-решения, часть — локальные (on-premises) системы с интеграцией через API \cite{applegate2009corporate, monk2013concepts}.
\par
Облачные модели (IaaS, PaaS, SaaS) различаются по уровню ответственности: от провайдера инфраструктуры до поставщика прикладного сервиса. Переход в облако снижает капитальные затраты, но увеличивает зависимость от провайдеров и требует внимания к управлению доступом, резервному копированию и соответствию требованиям по локализации данных.
ПО бывает прикладное (программные продукты для бизнеса) и системное (ОС, СУБД, middleware). Для организаций важен выбор между коробочными решениями, кастомной разработкой и облачными сервисами. Каждый подход имеет компромиссы по стоимости, времени внедрения и гибкости \cite{whitten2003systems, applegate2009corporate}.

\subsection{Корпоративные системы}
ERP и CRM покрывают регулярные операции и взаимодействия с клиентами; SCADA — промышленные управления; LMS и медицинские информационные системы — секторальные решения \cite{monk2013concepts, bocij2015business}.

Выбор между типовыми и специализированными решениями должен учитывать расходы на внедрение, гибкость настроек, способность к интеграции и длительность жизненного цикла системы. Часто выгодна стратегия «ядро + периферия»: ядро бизнеса обслуживает типовое ПО (ERP), а периферийные задачи — специализированные приложения или микросервисы.

\subsection{Интеграция и API}
Современные системы интегрируются через REST/GraphQL API, шины данных (ESB) или очереди сообщений. Стратегии интеграции включают синхронные и асинхронные обмены, с учётом соглашений о контракте данных и версионировании API.

\newpage

\subsection{Разработка и сопровождение}
Методологии разработки (Waterfall, Agile, DevOps) выбирают исходя из особенностей проекта: скорость изменений, критичность и потребность в интеграции. Автоматизация тестирования и CI/CD ускоряют доставку и повышают надёжность \cite{whitten2003systems, ward2016strategic}.

При переходе на DevOps организации важно менять и метрики: вместо численности закрытых задач целесообразно отслеживать lead time, частоту релизов, MTTR и процент успешных релизов. Эти метрики отражают реальное улучшение скорости и надёжности поставки ПО и помогают руководству оценивать эффект трансформации.

\section{3. Сети и средства связи}
Сети обеспечивают транспорт данных между пользователями, приложениями и облаком. Современные организации используют локальные сети, VPN, MPLS, а также публичные облачные каналы. Мобильные технологии (4G/5G) и беспроводные решения расширяют возможности доступа \cite{valacich2020essentials}.

Ключевые вопросы при проектировании:
\begin{itemize}
  \item пропускная способность и задержка;
  \item безопасность каналов и шифрование;
  \item качество обслуживания (QoS) для критичных приложений.
\end{itemize}
\par
Современные практики: использование SD-WAN для распределённых филиалов, сегментация корпоративной сети для защиты критичных сервисов, и подключение через выделенные каналы к облачным провайдерам. Эти подходы позволяют уменьшать задержки, повышать надёжность и централизовать контроль.

\subsection{Сетевые архитектуры для больших данных}
При больших объёмах данных важны каналы с высокой пропускной способностью и архитектуры, минимизирующие перемещения данных (data gravity). Часто используется локальная обработка (edge computing) для предварительной фильтрации и агрегации.

\newpage

\section{4. Данные и аналитика}
Данные — главное сокровище организации: транзакционные данные, журналы, мультимедиа, внешние источники. Правильная организация данных (ETL-пайплайны, DWH, Data Lake) критична для получения ценности \cite{ramakrishnan2003database, coronel2015database}.

\subsection{BI и аналитика}
BI-инструменты и аналитические платформы позволяют переходить от описательной аналитики к предиктивной и прескриптивной \cite{chen2012business, sharda2020business}. Для этого требуются качественные данные, цифровые навыки и интеграция с бизнес-процессами \cite{davenport2006competing}.

Практические сценарии применения аналитики:
\begin{itemize}
  \item Оперативная аналитика — мониторинг KPI, оповещения и дашборды;
  \item Тактическая аналитика — сегментация клиентов, оптимизация запасов;
  \item Стратегическая аналитика — модели прогнозирования спроса и сценарное планирование.
\end{itemize}

Эффективное внедрение аналитики начинается с малого — с пилотных проектов, где ценность можно измерить, и далее масштабируется на остальные процессы \cite{davenport2006competing, chen2012business}.

Полезный подход — формализация дорожной карты данных (data roadmap): определить наиболее перспективные сценарии, поставить KPI для пилота, оценить пороговые показатели экономической эффективности и только затем масштабировать решения. Такой подход снижает риск крупных неудачных инвестиций и обеспечивает быстрый возврат на ранних этапах.

\subsection{Таблица: примеры аналитических сценариев и технологий}
\begin{table}[ht]
\centering
\begin{tabular}{|p{5cm}|p{6cm}|}
\hline
  	\textbf{Сценарий} & \textbf{Технологии / инструменты} \\
\hline
Реальное время (streaming) & Kafka, Flink, Spark Streaming \\
\hline
Хранилище данных (DWH) & Redshift, BigQuery, Snowflake \\
\hline
BI и визуализация & Power BI, Tableau, Metabase \\
\hline
Машинное обучение & scikit-learn, TensorFlow, MLflow \\
\hline
\end{tabular}
\caption{Примеры аналитических сценариев и соответствующих технологий}
\end{table}


\subsection{Управление качеством данных}
Процессы очистки, нормализации, обогащения и верификации данных обеспечивают доверие к аналитике. Реализация data governance критически важна \cite{wade2004review}.
\par
Организационная модель управления данными включает роли: владелец данных (data owner), куратор данных (data steward), специалисты по качеству данных; и регламенты: SLA на качество, процедуры исправления ошибок и метрики качества (completeness, accuracy, timeliness).

\section{5. Люди, процессы и право}
Технологии не работают сами по себе: требуется организационная структура, роли (владельцы данных, администраторы, разработчики), регламенты и политика безопасности. Важны права доступа, процедуры аудита и соответствие нормативам (например, GDPR в международном контексте) \cite{pearlson2016strategic, alter2008defining}.

\subsection{Эффекты на организацию}
ИС изменяют распределение власти, ускоряют принятие решений и требуют новых компетенций. Управленческие изменения часто важнее технических при внедрении крупных систем \cite{melville2004information, applegate2009corporate}.

Классический пример — внедрение CRM: данные по клиентам перестают храниться у отдельных менеджеров и централизуются, что меняет мотивации, KPI и структуру взаимодействия подразделений. Поэтому внедрение сопровождается изменением регламентов, обучением и перераспределением ответственности.
\par
Юридические требования и комплаенс: локальные и международные регламенты (GDPR, HIPAA и т.п.) влияют на архитектурные решения и процедуры. Необходимо проводить оценку влияния на защиту данных (DPIA) в проектах, где обрабатываются персональные сведения.

\subsection{Организационные роли}
Типовые роли в ИТ-проектах: CTO/CIO (стратегия), проектный менеджер, архитектор, владелец продукта, владелец данных, аналитики и инженеры данных. Чёткое распределение ответственности снижает риски задержек и переработок.

\newpage

\section{6. Процессы, алгоритмы и проектная работа}
Процессы включают сбор требований, проектирование, разработку, тестирование и сопровождение. Алгоритмы автоматизации и аналитики преобразуют данные в знания — от простых правил до моделей машинного обучения \cite{pearlson2016strategic, sharda2020business}.

\subsection{Проектные практики}
Управление проектами ИТ: управление рисками, оценка стоимости владения (TCO), метрики успеха (время, бюджет, качество, польза). Agile- и DevOps-подходы улучшают поставку, но требуют зрелой культуры \cite{ward2016strategic, whitten2003systems}.
\par

MLOps и эксплуатация моделей: при внедрении аналитических моделей важны этапы их валидации, сопровождения и мониторинга в продакшне. MLOps практики включают CI/CD для моделей, мониторинг дрейфа данных и периодическую переобучение.

Технические детали: для мониторинга моделей полезно отслеживать метрики качества (например, AUC, RMSE), распределение входных признаков и производительность inference (latency, throughput). Автоматизация переобучения должна иметь чёткие триггеры (drift detection) и процедуру валидации, чтобы предотвратить деградацию качества в продакшне.

\subsection{Метрики успеха проектов ИТ}
Ключевые метрики: ROI, NPV, TCO, время выхода на окупаемость, удовлетворённость пользователей (CSAT), уровень автоматизации процессов. Эти показатели помогают принимать решения о дальнейшем масштабировании решений.

\newpage
\appendix
\section{Глоссарий}
\begin{itemize}
  \item ИТ (информационные технологии) — совокупность аппаратных, программных и сетевых средств.
  \item ИС (информационные системы) — организованные наборы компонентов, поддерживающие процессы и принятие решений.
  \item DWH — хранилище данных (Data Warehouse).
  \item ETL — процессы извлечения, трансформации и загрузки данных.
\end{itemize}

\subsection{Иллюстративный кейс}
Представим среднее предприятие розничной торговли, внедряющее BI-платформу и CRM. Проект состоял из этапов: сбор требований, подготовка ETL-пайплайнов, интеграция с POS-системами, обучение пользователей и запуск пилота. В результате предприятие снизило запас на 12\% и увеличило повторные продажи на 8\% за первый год. Ключевые факторы успеха: качество данных, вовлечённость руководства и постепенное масштабирование функциональности.
\section{Кейс: подробный разбор внедрения BI в розничной сети}
В этом разделе подробно разбирается пример внедрения BI в сети магазинов. Этапы включали:
\begin{enumerate}
  \item Анализ источников данных: POS, CRM, складские системы;
  \item Построение ETL-пайплайнов для объединения и очистки данных;
  \item Развёртывание DWH и настройка регулярных загрузок;
  \item Разработка дашбордов для операций, закупок и маркетинга;
  \item Пилот с двумя магазинами и постепенное масштабирование.
\end{enumerate}

Результаты и метрики: уменьшение избыточных запасов, улучшение точности прогнозов и повышение KPI по повторным продажам. Основные уроки: важно начинать с чётко измеримых гипотез и уделять внимание управлению изменениями в персонале \cite{davenport2006competing, coronel2015database}.

\section{Кейс: информационные системы в здравоохранении}
В здравоохранении внедрение ИС часто сопряжено с повышенными требованиями к безопасности и конфиденциальности. Пример: интеграция электронных медицинских карт (EMR) с лабораторными информационными системами и системой телемедицины.

Ключевые задачи и решения:
\begin{itemize}
  \item Интероперабельность через стандарты (HL7, FHIR);
  \item Шифрование данных в покое и при передаче;
  \item Ролевая модель доступа и аудит действий пользователей;
  \item Обучение клинического персонала и обеспечение поддержки в рабочее время.
\end{itemize}

  \paragraph{Уточнение по стандартам.} FHIR (Fast Healthcare Interoperability Resources) предоставляет современный RESTful-подход к обмену клиническими ресурсами и упрощает интеграцию современных веб-сервисов с традиционными EMR. При внедрении FHIR часто требуется реализовать адаптеры для устаревших интерфейсов и продумать стратегию версионирования ресурсов.

Эффект: снижение времени обработки результатов, уменьшение количества ошибок в документации и улучшение координации лечения. Источники подчеркивают роль организационных изменений и управления качеством данных \cite{oz2008management, davenport1998working}.

\section{Практический чеклист для внедрения ИС/BI (короткий)}
Ниже приведён компактный чеклист, который можно применять как контрольный список при планировании проектов:
\begin{enumerate}
  \item Определить бизнес-цели и KPI;
  \item Оценить существующие источники данных и их качество;
  \item Выбрать пилотную область с ощутимым эффектом;
  \item Сформировать команду: владелец продукта, аналитики, инженеры данных, ИБ-специалисты;
  \item Подготовить архитектуру с учётом резервирования и безопасности;
  \item Вести проект итеративно, измеряя эффекты и корректируя план;
  \item Организовать обучение и поддержку пользователей после запуска;
  \item Планировать сопровождение, мониторинг и оценки стоимости владения (TCO).
\end{enumerate}

\section{Методология исследования и адаптация практик}
В этом реферате использован гибридный подход: синтез конспектов и академической литературы с практическими наблюдениями и проверенными промышленными приёмами. Для каждого технологического решения предлагается оценка по трём измерениям: техническая реализуемость, организационная готовность и экономическая обоснованность. Такая триадная оценка помогает избежать ситуаций, когда решение технически верно, но не имеет поддержки у бизнеса или не окупается в доступный срок.

Шаги методики:
\begin{enumerate}
  \item Сбор и категоризация требований бизнеса;
  \item Технический аудит текущих систем и данных;
  \item Оценка экономической целесообразности (TCO/ROI);
  \item Пилотирование решений и измерение KPI;
  \item Шкала принятия решения и масштабирование.
\end{enumerate}

\section{Пример расчёта TCO (упрощённый)}
Для обоснования инвестиций удобно использовать упрощённую модель TCO (Total Cost of Ownership). Компоненты TCO можно разделить на начальные (CAPEX) и операционные (OPEX):
\begin{itemize}
  \item CAPEX: покупка оборудования, лицензии, начальная интеграция;
  \item OPEX: поддержка, подписки, обучение, электричество, аренда облачных ресурсов;
  \item Скрытые расходы: смена процессов, потери производительности при миграции, затраты на доработку интеграций.
\end{itemize}

Ниже приведён упрощённый табличный пример расчёта TCO за 3 года (значения условные):
\begin{table}[ht]
\centering
\begin{tabular}{|p{6cm}|p{3cm}|p{3cm}|p{3cm}|}
\hline
Компонент & Год 1 & Год 2 & Год 3 \\
\hline
CAPEX: оборудование и лицензии & 1 200 000 & 0 & 0 \\
\hline
OPEX: поддержка и облачные услуги & 300 000 & 330 000 & 363 000 \\
\hline
Обучение и внедрение & 150 000 & 30 000 & 30 000 \\
\hline
Итого & 1 650 000 & 360 000 & 393 000 \\
\hline
\end{tabular}
\caption{Упрощённый пример TCO (руб.)}
\end{table}

\section{Детальный план внедрения BI: шаги и оценки}
Ниже описан поэтапный план внедрения BI/аналитики, который можно масштабировать в зависимости от размера организации. Каждый этап сопровождается критериями завершения и целевыми метриками.

\subsection{Этап 1 — Подготовка и оценка (1–2 месяца)}
\begin{itemize}
  \item Сбор требований и постановка гипотез ценности;
  \item Инвентаризация источников данных и оценка качества;
  \item Быстрая оценка архитектурных ограничений и возможных интеграций;
  \item Критерий завершения: утверждённый план пилота и список KPI.
\end{itemize}

\subsection{Этап 2 — Пилот и MVP (2–4 месяца)}
\begin{itemize}
  \item Построение ETL для выбранной предметной области;
  \item Развёртывание DWH или data mart для пилота;
  \item Разработка первичных дашбордов и отчётов;
  \item Критерий завершения: измеримый эффект по KPI, готовность к масштабированию.
\end{itemize}

\subsection{Этап 3 — Масштабирование и интеграция (3–9 месяцев)}
\begin{itemize}
  \item Расширение источников данных, унификация схемы данных;
  \item Настройка процессов мастер-данных и data governance;
  \item Интеграция аналитики в операционные процессы (встраивание в приложения и рабочие процессы);
  \item Критерий завершения: стабильные SLA дашбордов и удовлетворённость ключевых пользователей.
\end{itemize}

\section{Шаблон RACI и SLA для проекта BI}
Ниже представлены сжатые шаблоны — RACI (responsible, accountable, consulted, informed) для ключевых ролей и пример SLA для производительности дашбордов.

\subsection{RACI (пример)}
\begin{itemize}
  \item Product Owner — A (Accountable) для бизнес-результатов;
  \item Data Engineer — R (Responsible) за ETL и поддержку конвейеров;
  \item Data Scientist / Analyst — R за качество моделей и отчётов;
  \item IT Security — C (Consulted) по требованиям безопасности;
  \item Руководство — I (Informed) о ключевых решениях и рисках.
\end{itemize}

\subsection{Пример SLA (дашборды)}
\begin{itemize}
  \item Время отклика (95-й процентиль) — < 3s для интерактивных дашбордов;
  \item Доступность — 99.5% в рабочие часы;
  \item Частота обновления данных — от realtime до ежедневной, по согласованию;
  \item Время восстановления после сбоя (RTO) — < 2h для критичных процессов.
\end{itemize}

\section{Дополнительные приложения и примеры}
В этом разделе приведены дополнительные пояснения и примеры, которые могут быть полезны при подготовке отчётов и обоснований для руководства.

\subsection{Пример: расписание проекта и ресурсы}
Ниже приведён пример плана ресурсов для небольшой пилотной инициативы (команда 6–8 человек):
\begin{itemize}
  \item 1 Project Manager (0.5 FTE);
  \item 1 Solution Architect (0.5 FTE);
  \item 2 Data Engineers (1.5 FTE);
  \item 2 BI Analysts / Data Scientists (1.5 FTE);
  \item 1 Security / DevOps (0.5 FTE).
\end{itemize}

\subsection{Шаблон отчёта по результатам пилота}
Рекомендуемые разделы отчёта по завершении пилота:
\begin{enumerate}
  \item Цели и гипотезы;
  \item Использованные данные и качество;
  \item Описание архитектуры и конвейеров;
  \item Динамика ключевых метрик (до/после);
  \item Риски и план по их уменьшению;
  \item Рекомендации по масштабированию и оценка TCO.
\end{enumerate}

\newpage
\section{Заключение}
Взаимодействие информационных технологий и информационных систем лежит в основе современной цифровой экономики. Технологии предоставляют инструменты, система — организует процессы и позволяет извлечь бизнес-ценность из данных. Успех зависит от интеграции технических, организационных и управленческих практик, грамотного выбора архитектуры и постоянного внимания к качеству данных и безопасности \cite{melville2004information, bharadwaj2013digital, valacich2020essentials}.

Резюмируем ключевые практические рекомендации:
\begin{itemize}
  \item Начинать с бизнес-целей и процессов, а не с технологий;
  \item Обеспечивать управление качеством данных и прозрачность источников;
  \item Поддерживать гибкую архитектуру (API, микросервисы) для быстрой интеграции;
  \item Инвестировать в людей: обучение и управление изменениями;
  \item Планировать безопасность и соответствие нормативам от проекта к проекту.
\end{itemize}

Ограничения настоящей работы — обзорный характер и сжатый формат. В дальнейшем полезно подготовить тематические кейсы с эмпирическим анализом выгод и затрат (TCO) по внедрению конкретных систем.

\printbibliography[heading=bibliography]

\end{document}
