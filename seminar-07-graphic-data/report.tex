% Реферат по теме "Технологии создания и обработки графических данных"
\documentclass{SibFU-docs}

% Дополнительные пакеты
\usepackage{graphicx}
\usepackage{amsmath}
\usepackage{listings}
\usepackage{xcolor}
\usepackage{hyperref}

% Настройка библиографии
\addbibresource{report.bib}

% Настройка листингов для кода
\lstset{
    basicstyle=\ttfamily\small,
    breaklines=true,
    frame=single,
    numbers=left,
    numberstyle=\tiny,
    keywordstyle=\color{blue},
    commentstyle=\color{green!60!black},
    stringstyle=\color{red}
}

\begin{document}

% Титульный лист
\makecovertitle{Гуманитарный институт}
{Прикладная информатика в искусстве и интерактивных медиа}
{ТЕХНОЛОГИИ СОЗДАНИЯ И ОБРАБОТКИ ГРАФИЧЕСКИХ ДАННЫХ}
{ГФ25-02Б}


\begin{abstract}
В данном реферате представлен комплексный анализ технологий создания и обработки графических данных. Работа охватывает ключевые аспекты компьютерной графики: от базовых различий между векторной и растровой графикой до современных методов обработки изображений и организации графических коллекций. Особое внимание уделяется цветовым моделям, форматам файлов и методам сжатия. Материал основан на актуальных источниках и структурирован для последовательного изучения всех аспектов работы с графическими данными.
\end{abstract}

\tableofcontents
\newpage

\section{Введение}
В современном цифровом мире компьютерная графика является неотъемлемой частью множества областей — от веб-дизайна и полиграфии до кинематографа и научных исследований. Технологии создания и обработки графических данных стали фундаментальным инструментом в работе специалистов различных направлений. В данной работе мы рассмотрим основные аспекты этих технологий, опираясь на авторитетные источники в области компьютерной графики \cite{Foley2013}, цветовой теории \cite{hunt2004reproduction} и технических стандартов \cite{W3CSVG2}.

Представление данных на экране компьютера в графической форме впервые было реализовано в середине XX века для крупных вычислительных систем, применявшихся в научных и военных целях. С развитием технологий цифровой обработки информация в виде изображений стала неотъемлемой частью большинства компьютерных систем, от персональных до серверных. Сегодня графические интерфейсы и обработка изображений — базовый компонент программного обеспечения, используемого в различных областях: от инженерии и науки до искусства и развлечений.

Важность данной темы обусловлена не только повсеместным использованием графических данных в различных сферах, но и постоянно растущими требованиями к качеству визуального контента. Современные технологии обработки изображений позволяют решать сложные задачи:
\begin{itemize}
    \item Улучшение качества изображений для различных целей
    \item Оптимизация графических данных для веб-публикаций
    \item Создание профессиональных материалов для полиграфии
    \item Разработка интерфейсов и визуальных элементов
    \item Обработка научной и технической визуализации
\end{itemize}

В рамках данного исследования мы подробно рассмотрим различные аспекты работы с графическими данными, начиная от базовых принципов представления изображений и заканчивая современными методами их обработки и организации. Особое внимание будет уделено практическим аспектам применения различных технологий и их оптимальному выбору для конкретных задач.

\section{Отличие векторной и растровой графики}
\subsection{Основные принципы представления графических данных}
В современном цифровом мире существуют два фундаментально различных подхода к представлению изображений: растровая и векторная графика \cite{Foley2013}. Понимание их различий, сильных и слабых сторон критически важно для эффективного выбора инструмента в конкретной задаче.

\subsection{Растровая графика: изображение из пикселей}
Растровое изображение (bitmap) состоит из множества мелких элементов — пикселей (pixels), организованных в прямоугольную сетку. Каждый пиксель имеет свой строго определенный цвет и положение. Совокупность этих разноцветных точек формирует целостную картинку \cite{AdobePhotoshopDocs}.

Ключевой характеристикой растрового изображения является разрешение (resolution), которое определяет количество пикселей на единицу длины (чаще всего на дюйм — dots per inch, DPI). Чем выше разрешение, тем больше деталей может содержать изображение.

Основные характеристики растровой графики:
\begin{itemize}
    \item Зависимость качества от разрешения
    \item Точная передача фотореалистичных изображений
    \item Большой объем данных при высоком качестве
    \item Сложность масштабирования без потери качества
    \item Возможность передачи тонких цветовых переходов
    \item Пригодность для работы с фотографиями
    \item Аппаратная поддержка в большинстве устройств вывода
\end{itemize}

\subsection{Векторная графика: изображение из математики}
Векторное изображение описывается не сеткой пикселей, а математическими формулами (векторами), которые определяют геометрические примитивы: точки, линии, кривые, окружности и многоугольники. Эти объекты состоят из опорных точек и кривых Безье, соединяющих их \cite{W3CSVG2}.

Преимущества векторной графики:
\begin{itemize}
    \item Масштабируемость без потери качества (разрешающая независимость)
    \item Небольшой размер файлов при простых изображениях
    \item Легкость редактирования отдельных элементов
    \item Точность геометрических форм
    \item Простота трансформации объектов
    \item Независимость качества от разрешения устройства вывода
    \item Идеальное качество при печати в любом масштабе
\end{itemize}

\subsection{Сравнительная характеристика и области применения}
\subsubsection{Растровая графика}
Оптимальна для:
\begin{itemize}
    \item Фотографий и фотореалистичных изображений
    \item Сложных изображений с множеством деталей и оттенков
    \item Художественных иллюстраций с эффектами текстур и кистей
    \item Обработки изображений (ретушь, цветокоррекция)
\end{itemize}

\subsubsection{Векторная графика}
Лучше всего подходит для:
\begin{itemize}
    \item Логотипов и фирменной символики
    \item Технических иллюстраций и чертежей
    \item Шрифтов и типографики
    \item Инфографики и схем
    \item Пользовательских интерфейсов
    \item Анимации без сложных текстур
\end{itemize}

\subsection{Взаимодействие форматов}
Важно понимать возможности и ограничения конвертации между форматами:
\begin{itemize}
    \item Преобразование из векторного в растровый формат (растрирование) всегда возможно и дает предсказуемый результат
    \item Преобразование из растрового в векторный формат (трассировка) часто приводит к потере качества и требует ручной доработки
    \item Некоторые форматы (например, PDF) могут содержать как векторные, так и растровые элементы
\end{itemize}

\section{Структура и принципы работы цветовых моделей}
\subsection{Теоретические основы цвета}
Цвет выступает одним из показателей восприятия человеком излучения света. Белый цвет включает в себя все цвета радуги, содержа излучения всех длин волн видимого человеком цветового диапазона \cite{hunt2004reproduction}. 

Восприятие цвета определяется следующими оптическими свойствами:
\begin{itemize}
    \item \textbf{Цветовой тон} — определяется длиной волны, преобладающей в спектре излучения
    \item \textbf{Чистота тона (насыщенность)} — определяется количеством присутствующего белого цвета
    \item \textbf{Яркость} — зависит от интенсивности светового излучения
\end{itemize}

\subsection{Цветовые модели и их классификация}
Цветовые модели представляют собой математические системы описания цвета, позволяющие стандартизировать его представление в цифровом виде \cite{cie2018colorimetry}. В описании цветовых моделей используются три основные системы:
\begin{itemize}
    \item \textbf{Аддитивные (световые)} — основаны на сложении световых потоков
    \item \textbf{Субтрактивные (красочные)} — основаны на поглощении части спектра
    \item \textbf{Перцепционные} — основаны на восприятии человека
\end{itemize}

\subsection{Эталонная модель CIE XYZ}
В 1931 году организация CIE разработала эталонную цветовую модель — цветовое пространство XYZ \cite{cie2018colorimetry}. Эта модель:
\begin{itemize}
    \item Основана на человеческом цветовосприятии
    \item Использует три гипотетических компонента (X,Y,Z)
    \item Служит основой для калибровки устройств
    \item Является эталоном для сравнения других моделей
\end{itemize}

\subsection{Основные цветовые модели}
\subsubsection{RGB модель}
RGB является аддитивной моделью \cite{poynton2012digital}, где цвета создаются сложением световых потоков. Характеристики модели:
\begin{itemize}
    \item Основные цвета: Red (красный), Green (зеленый), Blue (синий)
    \item Диапазон значений: 0-255 для каждого канала
    \item Белый цвет: максимум всех компонентов (255,255,255)
    \item Черный цвет: минимум всех компонентов (0,0,0)
    \item Основное применение: экраны, мониторы, проекторы
\end{itemize}

\subsubsection{CMYK модель}
CMYK представляет собой субтрактивную модель \cite{fraser2005real}, используемую преимущественно в полиграфии. Особенности:
\begin{itemize}
    \item Основные компоненты:
    \begin{itemize}
        \item Cyan (голубой)
        \item Magenta (пурпурный)
        \item Yellow (желтый)
        \item Key/Black (черный)
    \end{itemize}
    \item Принцип работы: вычитание цветов из белого
    \item Белый цвет: отсутствие красок (0,0,0,0)
    \item Черный цвет: максимум компонентов или только черной краски
\end{itemize}

\subsection{Перцепционные модели}
\subsubsection{LAB модель}
LAB является аппаратно-независимой моделью \cite{joblove1978color}, основанной на восприятии человека:
\begin{itemize}
    \item L (Lightness) — яркость
    \item A — положение цвета между красным и зеленым
    \item B — положение цвета между синим и желтым
\end{itemize}

\subsubsection{HSL/HSV модели}
Модели, основанные на интуитивном восприятии цвета:
\begin{itemize}
    \item Hue (тон) — собственно цвет
    \item Saturation (насыщенность) — чистота цвета
    \item Lightness/Value (светлота/яркость) — близость к белому или черному
\end{itemize}

\subsection{Управление цветом}
Современные системы управления цветом \cite{fraser2005real} обеспечивают:
\begin{itemize}
    \item Калибровку устройств ввода и вывода
    \item Преобразование между цветовыми пространствами
    \item Профилирование устройств
    \item Согласованную цветопередачу в различных условиях
\end{itemize}

\section{Форматы файлов и методы сжатия изображений}
\subsection{Введение в сжатие изображений}
В цифровой фотографии и веб-дизайне правильный выбор формата изображения напрямую влияет на качество визуального контента, скорость загрузки веб-страниц и эффективность использования памяти \cite{calltouch2022}. Сжатие графических файлов решает несколько критически важных задач:

\begin{itemize}
    \item \textbf{Экономия места для хранения} -- позволяет сохранять тысячи изображений на ограниченном пространстве жестких дисков и карт памяти
    \item \textbf{Ускорение передачи данных} -- уменьшенные файлы быстрее передаются по интернету
    \item \textbf{Оптимизация работы приложений} -- программы и браузеры быстрее обрабатывают сжатые изображения
    \item \textbf{Эффективное использование пропускной способности} -- особенно важно для мобильных сетей
\end{itemize}

\subsection{Основные форматы файлов}
\subsubsection{Растровые форматы}
\paragraph{JPEG/JPG}
Наиболее распространенный формат для фотографий \cite{kokoc2025}:
\begin{itemize}
    \item Использует сжатие с потерями
    \item Позволяет значительно уменьшить размер файла
    \item Оптимален для фотографий и изображений с плавными переходами
    \item Не подходит для изображений с четкими линиями или текстом
    \item Поддерживает различные уровни качества
\end{itemize}

\paragraph{PNG}
Формат с универсальным применением \cite{dssign2025}:
\begin{itemize}
    \item Сжатие без потерь по алгоритму Deflate
    \item Поддержка прозрачности (альфа-канал)
    \item Идеален для веб-графики, логотипов, скриншотов
    \item Больший размер файла по сравнению с JPEG
    \item Отличное качество для изображений с текстом и четкими краями
\end{itemize}

\paragraph{TIFF}
Профессиональный формат \cite{kokoc2025}:
\begin{itemize}
    \item Предназначен для полиграфии и профессиональной обработки
    \item Поддерживает различные методы сжатия
    \item Возможность хранения слоев и дополнительных каналов
    \item Большой размер файлов
    \item Сохранение максимального качества
\end{itemize}

\paragraph{WebP}
Современный формат от Google \cite{calltouch2022}:
\begin{itemize}
    \item Сочетает преимущества JPEG и PNG
    \item Поддерживает как сжатие с потерями, так и без потерь
    \item Обеспечивает меньший размер при том же качестве
    \item Поддержка анимации и прозрачности
    \item Оптимален для веб-применения
\end{itemize}

\subsubsection{Векторные форматы}
\paragraph{SVG (Scalable Vector Graphics)}
Открытый стандарт для веб-графики \cite{W3CSVG2}:
\begin{itemize}
    \item Основан на XML
    \item Масштабируемость без потери качества
    \item Поддержка анимации и интерактивности
    \item Возможность стилизации через CSS
    \item Небольшой размер для простых изображений
\end{itemize}

\paragraph{AI и EPS}
Профессиональные форматы для дизайна \cite{AdobeIllustratorDocs}:
\begin{itemize}
    \item AI -- родной формат Adobe Illustrator
    \item EPS -- универсальный формат для полиграфии
    \item Поддержка слоев и сложных эффектов
    \item Совместимость с различным ПО
\end{itemize}

\subsection{Методы сжатия изображений}
\subsubsection{Сжатие без потерь}
Методы сжатия без потерь обеспечивают полное восстановление исходного изображения \cite{habr2011}:

\paragraph{RLE (Run-Length Encoding)}
\begin{itemize}
    \item Принцип: замена последовательностей одинаковых пикселей парами "значение-количество"
    \item Эффективен для изображений с большими областями одного цвета
    \item Простота реализации
    \item Используется в форматах BMP и TIFF
\end{itemize}

\paragraph{LZW (Lempel-Ziv-Welch)}
\begin{itemize}
    \item Создание словаря часто встречающихся последовательностей
    \item Замена повторяющихся паттернов короткими кодами
    \item Применяется в форматах GIF и некоторых вариантах TIFF
    \item Эффективен для изображений с повторяющимися элементами
\end{itemize}

\paragraph{Deflate}
\begin{itemize}
    \item Комбинация алгоритма LZ77 и кодирования Хаффмана
    \item Используется в формате PNG
    \item Хорошая степень сжатия для различных типов данных
    \item Универсальность применения
\end{itemize}

\subsubsection{Сжатие с потерями}
Методы сжатия с потерями уменьшают размер файла за счет удаления части информации \cite{aco2023}:

\paragraph{Дискретное косинусное преобразование (DCT)}
\begin{itemize}
    \item Преобразование изображения из пространственной области в частотную
    \item Разбиение на блоки 8×8 пикселей
    \item Анализ частотных компонентов
    \item Основа алгоритма JPEG
\end{itemize}

\paragraph{Квантование}
\begin{itemize}
    \item Отбрасывание высокочастотных компонентов
    \item Округление коэффициентов до ближайших значений
    \item Управление степенью сжатия
    \item Возможность настройки качества результата
\end{itemize}

\subsection{Выбор оптимального формата}
При выборе формата необходимо учитывать \cite{kingservers2022}:
\begin{itemize}
    \item Тип изображения (фото, графика, текст)
    \item Требования к качеству
    \item Ограничения по размеру файла
    \item Необходимость прозрачности или анимации
    \item Целевую платформу или носитель
\end{itemize}

\section{Основы редактирования графических данных}
\subsection{История и развитие графических редакторов}
Представление данных на экране компьютера в графической форме впервые было реализовано в середине XX века. С развитием технологий появились различные инструменты для создания и редактирования изображений \cite{kpfu2023}. Современные графические редакторы предоставляют широкий спектр возможностей для работы с различными типами изображений.

\subsection{Классификация графических редакторов}
\subsubsection{По типу обрабатываемой графики}
\begin{itemize}
    \item \textbf{Растровые редакторы} (Adobe Photoshop, GIMP)
    \begin{itemize}
        \item Работа с пиксельной графикой
        \item Широкие возможности обработки фотографий
        \item Поддержка фильтров и эффектов
        \item Инструменты для ретуши и коррекции
    \end{itemize}
    \item \textbf{Векторные редакторы} (Adobe Illustrator, Inkscape)
    \begin{itemize}
        \item Создание и редактирование векторных объектов
        \item Работа с кривыми Безье
        \item Инструменты для создания логотипов и иллюстраций
        \item Поддержка масштабирования без потери качества
    \end{itemize}
\end{itemize}

\subsection{Принципы редактирования}
\subsubsection{Разрушающее редактирование}
Метод прямого изменения пикселей \cite{inkscape2008}:
\begin{itemize}
    \item Необратимое изменение исходного изображения
    \item Прямое воздействие на значения пикселей
    \item Отсутствие возможности отмены изменений после сохранения
    \item Меньшее потребление ресурсов компьютера
\end{itemize}

\subsubsection{Неразрушающее редактирование}
Современный подход к обработке изображений:
\begin{itemize}
    \item Использование слоев и масок
    \item Сохранение исходных данных
    \item Возможность корректировки на любом этапе
    \item Гибкость в настройке эффектов
\end{itemize}

\subsection{Основные инструменты и операции}
\subsubsection{Базовые инструменты}
\begin{itemize}
    \item \textbf{Инструменты выделения}
    \begin{itemize}
        \item Прямоугольное и овальное выделение
        \item Лассо (свободное выделение)
        \item Магнитное лассо
        \item Быстрое выделение
    \end{itemize}
    \item \textbf{Инструменты рисования}
    \begin{itemize}
        \item Кисть и карандаш
        \item Заливка
        \item Градиент
        \item Ластик
    \end{itemize}
\end{itemize}

\subsection{Методы коррекции изображений}
\subsubsection{Тоновая коррекция}
\begin{itemize}
    \item Уровни (Levels)
    \item Кривые (Curves)
    \item Яркость и контраст
    \item Тени и света
\end{itemize}

\subsubsection{Цветовая коррекция}
\begin{itemize}
    \item Цветовой баланс
    \item Тон/Насыщенность
    \item Выборочная коррекция цвета
    \item Замена цвета
\end{itemize}

\subsection{Работа со слоями}
Слои являются основой современного редактирования изображений:
\begin{itemize}
    \item \textbf{Типы слоев}
    \begin{itemize}
        \item Обычные слои с изображением
        \item Корректирующие слои
        \item Слои-маски
        \item Текстовые слои
    \end{itemize}
    \item \textbf{Режимы наложения}
    \begin{itemize}
        \item Умножение (Multiply)
        \item Перекрытие (Overlay)
        \item Экран (Screen)
        \item Мягкий свет (Soft Light)
    \end{itemize}
\end{itemize}

\subsection{Автоматизация и пакетная обработка}
Современные редакторы поддерживают:
\begin{itemize}
    \item Создание и использование действий (Actions)
    \item Пакетную обработку файлов
    \item Создание сценариев и макросов
    \item Автоматизацию рутинных операций
\end{itemize}

\section{Цифровая обработка изображений}
\subsection{Введение в цифровую обработку}
Цифровая обработка изображений (ЦОИ) — это область науки и техники, занимающаяся анализом и обработкой изображений с помощью вычислительных алгоритмов. С развитием цифровых технологий она стала неотъемлемой частью многих сфер: от медицины и астрономии до повседневной фотографии и систем видеонаблюдения.

\subsection{Основные понятия}
\subsubsection{Цифровое изображение}
Цифровое изображение представляет собой:
\begin{itemize}
    \item Двумерный массив чисел (пикселей)
    \item Каждый пиксель имеет координаты и значение интенсивности
    \item Для цветных изображений используется модель RGB
    \item Разрешение определяет детализацию изображения
\end{itemize}

\subsection{Методы улучшения качества изображений}
\subsubsection{Улучшение контраста и яркости}
\begin{itemize}
    \item \textbf{Гистограммная обработка}
    \begin{itemize}
        \item Эквализация гистограммы
        \item Линейное контрастирование
        \item Гамма-коррекция
    \end{itemize}
    \item \textbf{Цветовая коррекция}
    \begin{itemize}
        \item Баланс белого
        \item Насыщенность
        \item Цветовая температура
    \end{itemize}
\end{itemize}

\subsubsection{Подавление шума}
\paragraph{Линейные фильтры}
\begin{itemize}
    \item Усредняющий фильтр
    \item Гауссов фильтр
    \item Винеровская фильтрация
\end{itemize}

\paragraph{Нелинейные фильтры}
\begin{itemize}
    \item Медианный фильтр
    \item Билатеральный фильтр
    \item Анизотропная диффузия
\end{itemize}

\subsection{Современные методы обработки}
\subsubsection{Глубокое обучение в обработке изображений}
Применение нейронных сетей \cite{goodfellow2014gan} для:
\begin{itemize}
    \item Шумоподавления
    \item Супер-разрешения
    \item Колоризации
    \item Сегментации
    \item Детекции объектов
\end{itemize}

\subsubsection{Генеративные модели}
GAN-сети используются для:
\begin{itemize}
    \item Генерации реалистичных изображений
    \item Перевода изображений из одного домена в другой
    \item Восстановления поврежденных участков
    \item Улучшения качества старых фотографий
\end{itemize}

\subsection{Области применения}
\subsubsection{Медицинская визуализация}
\begin{itemize}
    \item Улучшение качества медицинских снимков
    \item Автоматическое обнаружение патологий
    \item Трехмерная реконструкция
    \item Планирование хирургических операций
\end{itemize}

\subsubsection{Системы безопасности}
\begin{itemize}
    \item Распознавание лиц
    \item Видеонаблюдение
    \item Биометрическая идентификация
    \item Анализ поведения
\end{itemize}

\subsubsection{Промышленное применение}
\begin{itemize}
    \item Контроль качества продукции
    \item Автоматическая инспекция
    \item Робототехника и машинное зрение
    \item Анализ материалов
\end{itemize}

\section{Организация графических коллекций}
\subsection{Проблемы цифрового хранения}
С ростом объема цифровых данных пользователи сталкиваются с рядом проблем \cite{berry2012digital}:
\begin{itemize}
    \item Потерянные часы на поиск файлов
    \item Стресс и пропущенные дедлайны
    \item Риск безвозвратной потери важных работ
    \item Сложности с организацией и каталогизацией
    \item Проблемы с резервным копированием
\end{itemize}

\subsection{Ключевые принципы организации}
\subsubsection{Структура папок ("Скелет")}
Основные подходы к организации \cite{gemmell2006long}:
\begin{itemize}
    \item \textbf{Хронологический принцип}
    \begin{itemize}
        \item Год → Месяц → Событие
        \item Идеален для фотографов
        \item Исключает дублирование
        \item Простая навигация по времени
    \end{itemize}
    \item \textbf{Проектный принцип}
    \begin{itemize}
        \item Структура по проектам
        \item Все файлы проекта в одном месте
        \item Удобен для дизайнеров
        \item Четкое разделение задач
    \end{itemize}
    \item \textbf{Тематический принцип}
    \begin{itemize}
        \item Организация по темам и категориям
        \item Удобен для коллекций референсов
        \item Возможность перекрестных ссылок
        \item Гибкая категоризация
    \end{itemize}
\end{itemize}

\subsection{Работа с метаданными}
\subsubsection{Типы метаданных}
\paragraph{EXIF (технические метаданные)} \cite{lupacchini2011metadata}
\begin{itemize}
    \item Автоматически записываются камерой
    \item Содержат технические параметры съемки
    \item Включают геолокацию (если доступно)
    \item Дата и время создания
\end{itemize}

\paragraph{IPTC (описательные метаданные)}
\begin{itemize}
    \item Добавляются вручную
    \item Авторские права и контакты
    \item Описание и ключевые слова
    \item Категории и рейтинги
\end{itemize}

\paragraph{XMP (расширяемые метаданные)}
\begin{itemize}
    \item Универсальный формат метаданных
    \item Поддержка пользовательских полей
    \item Совместимость с различными форматами
    \item История редактирования
\end{itemize}

\subsection{Системы управления цифровыми активами}
\subsubsection{Функциональность DAM-систем}
Современные системы обеспечивают \cite{adobe2025bridge}:
\begin{itemize}
    \item Централизованное хранение
    \item Поиск по метаданным
    \item Контроль версий
    \item Управление правами доступа
    \item Автоматизация рабочих процессов
\end{itemize}

\subsection{Стратегии резервного копирования}
\subsubsection{Правило 3-2-1}
Основные принципы:
\begin{itemize}
    \item 3 копии данных
    \item 2 различных типа носителей
    \item 1 копия в другом месте
\end{itemize}

\subsubsection{Автоматизация резервного копирования}
Рекомендуемые практики:
\begin{itemize}
    \item Регулярное расписание
    \item Проверка целостности данных
    \item Тестирование восстановления
    \item Ротация носителей
\end{itemize}

\subsection{Практические рекомендации}
\subsubsection{Организация рабочего процесса}
\begin{itemize}
    \item Создание четкой структуры папок
    \item Использование осмысленных имен файлов
    \item Регулярное обновление метаданных
    \item Систематическое резервное копирование
\end{itemize}

\subsubsection{Оптимизация поиска}
\begin{itemize}
    \item Использование тегов и ключевых слов
    \item Создание превью и миниатюр
    \item Внедрение поисковых систем
    \item Регулярная актуализация каталогов
\end{itemize}

\section{Заключение}
В результате проведенного исследования были рассмотрены ключевые аспекты технологий создания и обработки графических данных. Анализ показал, что современные методы работы с графической информацией представляют собой сложный комплекс технологий, требующий глубокого понимания как теоретических основ, так и практических аспектов их применения.

Особое внимание было уделено различиям между векторной и растровой графикой, что позволило выявить оптимальные области применения каждого типа. Исследование цветовых моделей показало их важность для точной цветопередачи в различных устройствах вывода. Анализ форматов файлов и методов сжатия продемонстрировал необходимость правильного выбора формата в зависимости от конкретной задачи.

\printbibliography[title=СПИСОК ИСПОЛЬЗОВАННЫХ ИСТОЧНИКОВ]

\end{document}
