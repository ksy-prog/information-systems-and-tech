% реферат по теме 'Дополненная и виртуальная реальность'
\documentclass{SibFU-docs}

% доп. пакеты
\usepackage{graphicx}
\usepackage{float}
\usepackage{hyperref}
\usepackage{caption}
\usepackage{subcaption}
\usepackage{amsmath}
% предотвартим ошибку с \Bbbk перед загрузкой пакета amssymb
\let\Bbbk\relax
\usepackage{amssymb}
\usepackage{booktabs}
\usepackage{tabularx}
\usepackage{longtable}
\usepackage{enumitem}
\usepackage{multirow}

% гиперссылки
\hypersetup{
    colorlinks=true,
    linkcolor=black,
    citecolor=blue,
    urlcolor=blue
}

% библиография
\addbibresource{report.bib}

% титульный лист
\begin{document}
\setlist[itemize]{left=1.3cm, itemsep=1pt, topsep=2pt} % для выравнивания списков
\setlist[enumerate]{left=1.3cm, itemsep=1pt, topsep=2pt}
\renewcommand{\contentsname}{}
\mycovertitle
{Гуманитарный институт}
{Кафедра прикладной информатики в искусстве и интерактивном медиа}
{Основы архитектуры информационных систем и администрирования}
{ст. преподаватель И.~Р.~Нигматуллин}
{Захаров И.М., Мельников С.В., Фиряго О.А., Янцева А. Д., Тетерина П.А.,\\Чуняев М.Д., Рябинин Т.В.}
 
\begin{center}
\bfseries РЕФЕРАТ
\end{center}
\noindent

Реферат по теме: «Основы архитектуры информационных систем и администрирования» содержит 43 страницы, 6 таблиц и 48 использованных источников.

Ключевые слова: ИНФОРМАЦИОННАЯ СИСТЕМА, АРХИТЕКТУРА, МНОГОУРОВНЕВАЯ АРХИТЕКТУРА, КЛИЕНТ-СЕРВЕР, УПРАВЛЕНИЕ ДОСТУПОМ, АДМИНИСТРИРОВАНИЕ, БЕЗОПАСНОСТЬ, ВИРТУАЛИЗАЦИЯ, КОНТЕЙНЕРИЗАЦИЯ, РЕЗЕРВНОЕ КОПИРОВАНИЕ, ИНЦИДЕНТЫ, DEVOPS.

В работе исследуются фундаментальные принципы построения архитектуры информационных систем и основные задачи системного администрирования. Рассмотрены компоненты трёхуровневой архитектуры, роли серверов, клиентов и сетевых устройств, функции пользователей, операторов и администраторов, модели управления доступом (DAC, MAC, RBAC, ABAC), а также современные инструменты и практики администрирования.

Основные выводы:

\begin{enumerate}
    \item Правильное разделение архитектуры на уровни (представления, бизнес-логики, данных) обеспечивает масштабируемость, гибкость и безопасность системы.
    
    \item Чёткое разграничение ролей (пользователь, оператор, администратор) согласно принципам минимальных привилегий и разделения обязанностей критично для информационной безопасности.
    
    \item Современные практики (DevOps, SRE, Infrastructure as Code) трансформируют роль администратора в направлении инженера-разработчика.
\end{enumerate}

Рекомендации:

\begin{itemize}[label=-]
    \item Использовать трёхуровневую архитектуру с разделением ответственности между компонентами.
    \item Внедрять управление доступом на основе ролей (RBAC) или атрибутов (ABAC).
    \item Применять правило 3-2-1 для резервного копирования критических данных.
    \item Использовать инструменты автоматизации (Ansible, Terraform) и CI/CD пайплайны.
\end{itemize}

\newpage

\thispagestyle{empty}
\begin{center}
\bfseries СОДЕРЖАНИЕ
\end{center}
\vspace{-6pt}
\tableofcontents
\clearpage

% фантомные боли для секций, чтобы не было нумераций кроме тела реферата
\section*{}
\vspace*{-2.5em}
\begin{center}\textbf{ВВЕДЕНИЕ}\end{center}
\phantomsection
\addcontentsline{toc}{section}{Введение}

Информационные системы (ИС) стали критически важным активом современных организаций. От эффективности их работы зависят скорость принятия решений, качество услуг и конкурентоспособность. Надёжная работа ИС требует не только правильного архитектурного решения, но и качественного системного администрирования.

Современное состояние проблемы. Архитектура ИС находится на пересечении классических подходов (трёхуровневая архитектура) и современных инноваций (облачные технологии, контейнеризация). Роль администратора эволюционировала от технического специалиста к DevOps/SRE-инженеру. Модели управления доступом развиваются от простых (DAC, MAC) к гибким и контекстно-зависимым (RBAC, ABAC).

Актуальность обусловлена критичностью ИС для бизнеса, растущей сложностью технологий, возрастающими требованиями к безопасности и нехваткой квалифицированных кадров.

Новизна заключается в объединении классических концепций архитектуры с современными практиками администрирования (DevOps, SRE, Infrastructure as Code), демонстрации их взаимосвязи и практического применения.

Цель работы: изучение принципов архитектуры ИС и задач системного администрирования; выявление взаимосвязи между архитектурными решениями и практическими аспектами эксплуатации.

Задачи: определить компоненты архитектуры ИС; описать роли серверов, клиентов и сетевых устройств; проанализировать функции пользователей, администраторов и операторов; изучить принципы управления доступом; рассмотреть задачи администрирования; описать современные инструменты и практики.

Методы: системный анализ, сравнительный анализ архитектур и моделей доступа, анализ литературы и нормативных документов (ГОСТ, ФСТЭК).

\newpage
% вручную увеличить номер секции и записать в оглавление с номером,
% чтобы в теле не печатался автоматический номер, но в tableofcontents он остался
\refstepcounter{section}
\addcontentsline{toc}{section}{\protect\numberline{\thesection} Основные компоненты архитектуры информационной системы и их взаимодействие}
\vspace*{-2.5em}
\begin{center}\textbf{\thesection\ ОСНОВНЫЕ КОМПОНЕНТЫ АРХИТЕКТУРЫ ИНФОРМАЦИОННОЙ СИСТЕМЫ И ИХ ВЗАИМОДЕЙСТВЕ}\end{center}
\vspace*{-1.5em}
\ManualSubsection{Понятие и цели архитектуры информационной системы}

Архитектура информационной системы представляет собой фундаментальную организацию системы, воплощённую в её компонентах, их взаимоотношениях и принципах развития \cite{fowler2002patterns}. В современных условиях архитектура служит мостом между бизнес-требованиями и технической реализацией, определяя не только структуру системы, но и принципы её эволюции.

\ManualSubsubsection{Определение архитектуры ИС}

Архитектура информационной системы представляет собой структурный план, охватывающий несколько взаимосвязанных аспектов. Прежде всего, она определяет компоненты системы --- логические и физические части, из которых состоит ИС. Далее архитектура описывает функции каждого компонента, устанавливая зону его ответственности. Не менее важным является описание взаимодействий между компонентами --- способов обмена данными и координации работы. Наконец, архитектура фиксирует принципы и стандарты, которые регламентируют выбор технологий и подходов к проектированию.

Таким образом, архитектура ИС выступает связующим звеном между высокоуровневыми бизнес-требованиями («что нужно организации») и конкретной технической реализацией («как это воплотить в коде и инфраструктуре»).

\ManualSubsubsection{Ключевые цели и преимущества качественной архитектуры}

Качественная архитектура решает множество стратегических задач, которые подробно рассматриваются в работах \cite{richards2020fundamentals} и \cite{bass2012software}. Снижение сложности достигается за счёт разделения системы на модули, что делает её более понятной и управляемой для команды разработки и эксплуатации. Масштабируемость обеспечивает возможность наращивания вычислительной мощности и обработки растущих объёмов данных без полной перестройки системы.

Надёжность и отказоустойчивость позволяют системе противостоять сбоям в отдельных компонентах и продолжать функционирование, что критически важно для бизнес-критичных приложений. Безопасность реализуется через встроенные механизмы защиты данных от несанкционированного доступа на всех уровнях архитектуры. Сопровождаемость и расширяемость обеспечивают лёгкость внесения изменений и добавления функционала, снижая стоимость развития системы.

Производительность гарантирует требуемое время отклика и пропускную способность, что напрямую влияет на удовлетворённость пользователей. Интероперабельность даёт возможность системе взаимодействовать с другими системами и внешними сервисами, что особенно важно в условиях интеграции с партнёрами и облачными платформами. Оптимизация затрат позволяет эффективно управлять расходами на протяжении всего жизненного цикла системы.

\ManualSubsection{Основные компоненты архитектуры информационной системы}

Архитектуру ИС можно рассматривать с разных точек зрения: логической, физической и программной. Наиболее универсальным является логическое представление в виде многоуровневой (N-уровневой) архитектуры.

\ManualSubsubsection{Трёхуровневая архитектура}

Классической является трёхуровневая архитектура, которая легла в основу большинства современных веб-приложений \cite{kleppmann2017designing}. Данный подход разделяет систему на три логически независимых слоя, каждый из которых решает свой круг задач и может развиваться относительно автономно.

\vspace{1.5em}
\noindent Таблица 1 – Сравнение трёх уровней архитектуры

{\small
\noindent
\begin{tabular}{|p{0.20\textwidth}|p{0.37\textwidth}|p{0.37\textwidth}|}
\hline
\textbf{Уровень} & \textbf{Назначение} & \textbf{Технологии} \\
\noalign{\hrule height 1.2pt}
\hline
Представления & Визуализация данных, обработка пользовательского ввода, валидация на клиенте & HTML, CSS, JavaScript, React, Angular, Vue.js, мобильные приложения \\
\hline
Бизнес-логики & Обработка бизнес-правил, оркестрация workflows, управление транзакциями & Java, C\#, Python, Node.js, Spring Boot, .NET Core, Docker, Kubernetes \\
\hline
Данных & Постоянное хранение, извлечение данных, обеспечение целостности & PostgreSQL, MySQL, MongoDB, Redis, Apache Kafka, Hadoop HDFS \\
\hline
\end{tabular}
}

\ManualSubsection{Уровень представления (Presentation Layer)}

Уровень представления является «лицом» системы и точкой взаимодействия с конечным пользователем. \cite{martin2017clean} определяет его роль как критически важную для пользовательского опыта.

Данный уровень отвечает за визуализацию данных, включая отображение пользовательского интерфейса с кнопками, формами и графиками. Здесь происходит обработка пользовательского ввода --- приём действий пользователя, таких как клики, ввод текста и жесты. Валидация данных на клиенте позволяет проверять корректность введённых данных без обращения к серверу, что ускоряет отклик системы. Логика отображения определяет, какие элементы интерфейса показывать пользователю в зависимости от его роли и контекста.

С технологической точки зрения уровень представления реализуется через веб-браузеры с использованием HTML, CSS, JavaScript и современных фреймворков (React, Angular, Vue.js), через мобильные приложения (нативные на Swift и Kotlin или кроссплатформенные), а также через десктоп-приложения на базе Windows Forms, WPF, JavaFX или Electron.

\ManualSubsection{Бизнес-уровень (Application Layer)}

Бизнес-уровень представляет собой «мозг» системы, где реализована основная логика предметной области. \cite{hohpe2003enterprise} рассматривают этот уровень как центральный для обработки бизнес-правил.

На этом уровне происходит обработка бизнес-правил --- реализация алгоритмов и логики, уникальных для предметной области, например, расчёт налогов или проверка кредитоспособности. Оркестрация workflows координирует последовательность действий для выполнения сложных операций, таких как процесс оформления заказа с резервированием товара, списанием бонусов и вызовом службы доставки. Трансформация данных обеспечивает преобразование между форматами, удобными для уровня представления и уровня данных.

Управление транзакциями гарантирует целостность операций по принципам ACID (Atomicity, Consistency, Isolation, Durability). Кэширование обеспечивает временное хранение часто запрашиваемых данных для повышения производительности. Авторизация и контроль доступа проверяют права пользователя на выполнение конкретных операций.

Технологический стек бизнес-уровня включает серверные языки программирования (Java, C\#, Python, Node.js, PHP, Go), фреймворки (Spring Boot, .NET Core, Express.js), а также инструменты контейнеризации (Docker) и оркестрации (Kubernetes).

\ManualSubsection{Уровень данных (Data Access Layer)}

Уровень данных является «памятью» системы, отвечающей за надёжное хранение и управление данными. \cite{kleppmann2017designing} подчёркивает критическую важность правильной архитектуры этого уровня.

Постоянное хранение данных обеспечивает запись информации на долговременные носители. Извлечение данных реализует эффективный поиск и выборку по запросам. Целостность данных гарантирует точность и непротиворечивость информации с помощью ограничений, ключей и транзакций. Резервное копирование и восстановление защищают данные от потерь. Управление параллельным доступом разрешает конфликты при одновременных обращениях нескольких пользователей.

\vspace{1.5em}
\noindent Таблица 2 – Типы хранилищ данных

{\small
\noindent
\begin{tabular}{|p{0.22\textwidth}|p{0.38\textwidth}|p{0.34\textwidth}|}
\hline
\textbf{Тип хранилища} & \textbf{Характеристика} & \textbf{Примеры} \\
\noalign{\hrule height 1.2pt}
\hline
Реляционные СУБД & Структурированные данные с чёткими связями, язык SQL & PostgreSQL, MySQL, Oracle, SQL Server \\
\hline
Документные NoSQL & Хранение JSON-подобных документов, гибкая схема & MongoDB, Couchbase \\
\hline
Ключ-значение & Кэширование и сессии, высокая скорость доступа & Redis, Amazon DynamoDB \\
\hline
Графовые БД & Данные со сложными связями, социальные графы & Neo4j \\
\hline
Брокеры сообщений & Буферизация и перемещение потоков данных & Apache Kafka, RabbitMQ \\
\hline
\end{tabular}
}

Технологии уровня данных включают реляционные СУБД (PostgreSQL, MySQL, Microsoft SQL Server, Oracle Database), нереляционные NoSQL-решения (MongoDB, Cassandra, Redis, Neo4j), брокеры сообщений (Apache Kafka, RabbitMQ), хранилища больших данных (Hadoop HDFS, Amazon S3) и системы ETL/ELT (Apache Airflow, Talend).

\ManualSubsection{Сквозные компоненты архитектуры}

Помимо трёх основных уровней, существуют компоненты, пронизывающие всю архитектуру. \cite{aws2023wellarchitected} определяет их как критически важные для успеха системы.

Компоненты безопасности включают аутентификацию (проверку подлинности пользователя), авторизацию (проверку прав доступа к ресурсам), шифрование (защиту данных при передаче и хранении) и аудит с логированием (фиксацию событий безопасности).

Компоненты наблюдаемости охватывают логирование (запись событий и ошибок), мониторинг (сбор метрик о работе системы) и трассировку (отслеживание пути запроса через микросервисы).

Интеграция с внешними системами реализуется через API (REST, GraphQL, gRPC) и ESB (Enterprise Service Bus) --- промежуточное программное обеспечение для маршрутизации сообщений между разнородными системами.

\newpage
\refstepcounter{section}
\addcontentsline{toc}{section}{\protect\numberline{\thesection}Роли серверов, клиентов и сетевых устройств в архитектуре системы}
\begin{center}\textbf{\thesection\ РОЛИ СЕРВЕРОВ, КЛИЕНТОВ И СЕТЕВЫХ УСТРОЙСТВ В АРХИТЕКТУРЕ СИСТЕМЫ}\end{center}
\vspace*{-1.5em}
\ManualSubsection{Клиент-серверная архитектура: основы и значение}

Клиент-серверная архитектура представляет собой фундаментальную вычислительную модель, разделяющую функциональность между поставщиками услуг (серверами) и потребителями (клиентами). \cite{supermicro2024clientserver} определяет эту архитектуру как основу современных информационных систем.

\ManualSubsubsection{Ключевые характеристики}

Централизация управления и ресурсов означает, что сервер выступает единой точкой контроля, что упрощает администрирование и обеспечивает согласованность данных. Масштабируемость достигается за счёт возможности горизонтального и вертикального расширения серверной части без необходимости модификации клиентов. Разделение ответственности обеспечивает чёткое распределение функций между клиентом и сервером, что способствует независимой разработке и обновлению компонентов. Модульность и гибкость позволяют обновлять клиентские и серверные компоненты независимо друг от друга.

\ManualSubsection{Роли серверов в системе}

Серверы выполняют различные функции в зависимости от их специализации. 

\vspace{1.5em}
\noindent Таблица 3 – Типы серверов

{\small
\noindent
\begin{tabular}{|p{0.18\textwidth}|p{0.40\textwidth}|p{0.36\textwidth}|}
\hline
\textbf{Тип сервера} & \textbf{Функции} & \textbf{Примеры} \\
\noalign{\hrule height 1.2pt}
\hline
Веб-сервер & Приём HTTP/HTTPS-запросов, возврат HTML-страниц и статического контента & Apache HTTP Server, Nginx, Microsoft IIS \\
\hline
Сервер приложений & Выполнение бизнес-логики, управление сессиями, кэширование & Tomcat, JBoss, Node.js \\
\hline
Сервер БД & Хранение и управление данными, обеспечение ACID-свойств & MySQL, PostgreSQL, Oracle, SQL Server \\
\hline
Файловый сервер & Централизованное хранение файлов, управление доступом & Windows File Server, NFS, Samba \\
\hline
Сервер кэширования & Ускорение доступа к часто запрашиваемым данным & Redis, Memcached \\
\hline
\end{tabular}
}

Веб-сервер получает HTTP/HTTPS-запросы и возвращает веб-ресурсы, такие как HTML-страницы, изображения и файлы. К типичным представителям относятся Apache HTTP Server, Nginx и Microsoft IIS \cite{kurose2021computer}. Веб-серверы оптимизированы для быстрой обработки статического контента, однако для выполнения сложной бизнес-логики они передают запросы на сервер приложений.

Сервер приложений выступает промежуточным звеном, выполняющим бизнес-логику. Примерами служат Tomcat, JBoss и Node.js. Этот тип серверов управляет сессиями пользователей, кэширует данные и обеспечивает балансировку нагрузки между обработчиками запросов.

Сервер баз данных представляет собой систему управления данными. К наиболее распространённым решениям относятся MySQL, PostgreSQL, Oracle Database и SQL Server. Сервер БД отвечает за соблюдение ACID-свойств транзакций и безопасность хранимых данных.

Файловый сервер обеспечивает централизованное хранение и управление доступом к файлам. \cite{netwrix2024networkdevices} указывает на важность правильной конфигурации файловых сервисов для обеспечения безопасности и контроля версий документов.

Сервер кэширования ускоряет доступ к часто запрашиваемым данным, размещая их в оперативной памяти. Redis и Memcached являются типичными представителями этой категории и позволяют существенно снизить нагрузку на основные серверы баз данных.

\ManualSubsection{Тонкие и толстые клиенты}
\vspace*{-1.5em}
\ManualSubsubsection{Тонкий клиент}

Тонкий клиент представляет собой минималистичное устройство с зависимостью от сервера для обработки данных. \cite{designgurus2024thickthin} определяет характеристики тонких клиентов как ориентированные на минимизацию локальных вычислений.

Тонкие клиенты характеризуются минимальной локальной обработкой, низкими требованиями к оборудованию, централизованным управлением и обновлением, а также требованием постоянного сетевого соединения. К примерам относятся веб-браузеры, терминальные клиенты (RDP, SSH), VDI-станции и облачные приложения.

Преимущества тонких клиентов включают низкую стоимость оборудования, простоту администрирования, высокую безопасность данных (поскольку они не хранятся локально) и низкое энергопотребление. Однако существуют и недостатки: полная зависимость от сети и сервера, возможные проблемы с латентностью, высокие требования к пропускной способности сети и ограниченная функциональность в офлайн-режиме.

\ManualSubsubsection{Толстый клиент}

Толстый клиент представляет собой полноценное устройство, способное выполнять сложные операции локально. \cite{designgurus2024thickthin} рассматривает преимущества и недостатки этого подхода в контексте различных сценариев использования.

Характеристики толстого клиента включают значительную локальную вычислительную мощность, собственную операционную систему и приложения, возможность автономной работы и необходимость отдельного управления обновлениями на каждом устройстве. Примерами служат настольные ПК с установленными приложениями, ноутбуки разработчиков и интегрированные среды разработки (IDE).

К преимуществам относятся автономность и возможность офлайн-работы, высокая производительность для ресурсоёмких задач, меньшая зависимость от сети и возможность расширения через плагины. Недостатками являются высокая стоимость оборудования, сложность администрирования и обновлений, повышенные риски безопасности из-за локального хранения данных, высокое энергопотребление и проблемы синхронизации при офлайн-работе.

\ManualSubsection{Многоуровневые архитектуры}
\vspace*{-1.5em}
\ManualSubsubsection{Двухуровневая архитектура}

Двухуровневая архитектура состоит из клиентского уровня и уровня данных. \cite{ibm2021threetier} сравнивает двух- и трёхуровневые подходы, выявляя их преимущества в различных контекстах.

В данной архитектуре клиент напрямую взаимодействует с базой данных через ODBC, JDBC или другие драйверы. Бизнес-логика распределена между клиентом и хранимыми процедурами БД. Преимуществами являются простота реализации, быстрое развёртывание и низкие начальные затраты. К недостаткам относятся ограниченная масштабируемость, проблемы безопасности из-за прямого доступа к БД, сложность поддержки при росте системы и риск превращения базы данных в узкое место производительности. Типичные применения включают Microsoft Access, простые десктопные приложения и малые CRM-системы.

\ManualSubsubsection{Трёхуровневая архитектура}

Трёхуровневая архитектура разделяет приложение на три логических слоя: представление, приложение и данные. Уровень представления (Presentation Tier) отвечает за пользовательский интерфейс через браузер, мобильное приложение или десктопный клиент. Уровень приложения (Application/Business Logic Tier) является промежуточным слоем, обрабатывающим бизнес-логику, правила, валидацию, управление транзакциями и контроль доступа. Уровень данных (Data Tier) представляет базу данных, отвечающую за хранение и управление данными, при этом клиент не имеет прямого доступа --- все запросы проходят через уровень приложения.

Преимущества трёхуровневой архитектуры включают высокую масштабируемость (каждый уровень масштабируется независимо), повышенную безопасность (клиенты не имеют прямого доступа к БД), модульность и разделение ответственности, возможность независимой разработки и обновления уровней, а также лёгкость поддержки и модификации. Недостатками являются более сложная архитектура, увеличенное время разработки, потенциальная латентность и повышенная сложность развёртывания. Применение включает большинство современных веб-приложений, корпоративные ERP/CRM-системы, облачные приложения и системы онлайн-банкинга.

\ManualSubsection{Сетевые устройства в архитектуре}
\vspace*{-1.5em}
\ManualSubsubsection{Маршрутизатор}

Маршрутизатор направляет сетевые пакеты между различными компьютерными сетями на основе IP-адресов. \cite{cloudflare2024switch} определяет роль маршрутизаторов в современных сетях как ключевую для обеспечения связности.

Маршрутизатор работает на сетевом уровне модели OSI (Layer 3) и читает IP-адрес назначения в каждом пакете, консультируя таблицу маршрутизации для определения оптимального пути. Существует два типа маршрутизации: статическая, при которой маршруты настраиваются администратором вручную, и динамическая, при которой маршруты определяются автоматически через протоколы OSPF, RIP или BGP.

Функции маршрутизатора включают направление трафика между сетями, NAT (Network Address Translation) для преобразования IP-адресов, фильтрацию через ACL (Access Control Lists) и обеспечение качества обслуживания (QoS) для приоритизации трафика.

\ManualSubsubsection{Коммутатор}

Коммутатор соединяет множество устройств в локальной сети (LAN) и пересылает данные между ними на основе MAC-адресов. \cite{cisco2024firewall} указывает на важность коммутаторов в сетевой архитектуре для обеспечения эффективной передачи данных.

Коммутатор работает на канальном уровне (Layer 2) модели OSI. Он получает кадр Ethernet, читает MAC-адрес назначения и отправляет кадр только на соответствующий порт, а не на все порты как устаревший хаб, что существенно повышает эффективность использования сети.

Типы коммутаторов включают неуправляемые (plug-and-play без конфигурации), управляемые (с полной конфигурацией, поддержкой VLAN и SNMP) и Layer 3 коммутаторы, комбинирующие функции коммутатора и маршрутизатора. Функции коммутаторов охватывают создание отдельных доменов коллизий, обучение MAC-адресов и построение таблицы коммутации, создание виртуальных локальных сетей (VLAN) и предотвращение циклов через Spanning Tree Protocol.

\ManualSubsubsection{Межсетевой экран}

Межсетевой экран мониторит и фильтрует входящий и исходящий сетевой трафик на основе правил безопасности. \cite{haproxy2025loadbalancing} подчёркивает важность сетевой фильтрации в современных инфраструктурах для защиты от угроз.

Основные функции межсетевого экрана включают мониторинг входящего и исходящего трафика, фильтрацию на основе правил, динамическую фильтрацию (Stateful Inspection), NAT и маскирование внутренней сетевой топологии, логирование и мониторинг, а также защиту от вторжений (IPS).

Типы межсетевых экранов включают пакетные фильтры (проверяют только заголовки пакетов), Stateful Inspection (отслеживают состояние соединений), Application-Level Gateway или Proxy Firewall (проверяют содержимое на уровне приложений) и Next-Generation Firewall (NGFW), комбинирующие все возможности предыдущих типов с расширенными функциями анализа угроз.

\ManualSubsubsection{Балансировщик нагрузки}

Балансировщик нагрузки распределяет входящий трафик между несколькими серверами для оптимизации использования ресурсов. \cite{designgurus2024thickthin} рассматривает роль балансировки в масштабируемых системах как критически важную для обеспечения высокой доступности.

Алгоритмы балансировки включают Round Robin (последовательное распределение запросов), Least Connections (направление на сервер с наименьшим числом соединений), IP Hash (привязка клиента к серверу по хэшу IP-адреса) и Weighted Round Robin (распределение с учётом весов серверов в зависимости от их мощности).

Преимущества балансировки нагрузки охватывают повышение производительности за счёт распределения нагрузки, обеспечение высокой доступности при отказе отдельных серверов, возможность горизонтального масштабирования и скрытие внутренней топологии инфраструктуры от внешних пользователей.

\newpage
\refstepcounter{section}
\addcontentsline{toc}{section}{\protect\numberline{\thesection}Функции пользователей, администраторов и операторов в информационной системе}
\begin{center}\textbf{\thesection\ ФУНКЦИИ ПОЛЬЗОВАТЕЛЕЙ, АДМИНИСТРАТОРОВ И ОПЕРАТОРОВ В ИНФОРМАЦИОННОЙ СИСТЕМЕ}\end{center}
\vspace*{-1.5em}
\ManualSubsection{Разделение ролей в информационной системе}

Современные информационные системы работают эффективно благодаря чёткому разделению ролей и ответственности. \cite{limoncelli2016practice} подчёркивает важность правильного распределения функций в системном администрировании. Каждая роль в ИС выполняет специфические функции, которые необходимо чётко разграничить для обеспечения безопасности, стабильности и эффективности системы.

\vspace{1.5em}
\noindent Таблица 4 – Сравнение ролей в ИС

{\small
\noindent
\begin{tabular}{|p{0.15\textwidth}|p{0.27\textwidth}|p{0.27\textwidth}|p{0.21\textwidth}|}
\hline
\textbf{Роль} & \textbf{Основные функции} & \textbf{Права доступа} & \textbf{Пример} \\
\noalign{\hrule height 1.2pt}
\hline
Пользователь & Работа с данными, использование готовых инструментов & Ограниченные, только необходимые приложения & Менеджер в CRM \\
\hline
Оператор & Мониторинг, помощь пользователям, резервное копирование & Доступ к логам, простые операции восстановления & Специалист техподдержки \\
\hline
Администратор & Управление инфраструктурой, безопасность, развитие ИС & Полный доступ к системным ресурсам & Системный администратор \\
\hline
\end{tabular}
}

\ManualSubsection{Роль пользователя}

Пользователь представляет собой сотрудника или клиента, который решает повседневные задачи в системе. Это может быть менеджер, работающий с CRM-системой, или студент, использующий образовательный портал. Его деятельность сосредоточена на выполнении бизнес-функций, а не на технической поддержке системы.

Функции пользователя охватывают работу с данными, включая ввод информации, оформление заявок и создание документов. Пользователь активно применяет готовые инструменты системы: поиск, формирование отчётов, доступ к личному кабинету. Важной обязанностью является соблюдение политики безопасности --- сохранение конфиденциальности пароля и осторожность при работе с персональными данными. При возникновении проблем пользователь информирует администратора и обращается в техподдержку.

Права пользователя строго ограничены: он имеет доступ только к необходимым приложениям и данным, видит ограниченную часть информации в зависимости от должности и не может изменять системные параметры или устанавливать программное обеспечение. Такое разграничение прав предотвращает утечки данных, случайные удаления и обеспечивает безопасность системы.

\ManualSubsection{Роль оператора}

Оператор является техническим специалистом, который поддерживает текущую работоспособность системы. Его деятельность направлена на оперативное реагирование на возникающие проблемы и обеспечение бесперебойной работы системы в режиме реального времени.

Задачи оператора включают мониторинг состояния системы с контролем за доступностью и производительностью, помощь пользователям в виде сброса паролей, восстановления доступа и решения простых проблем. Оператор осуществляет резервное копирование, создавая резервные копии данных согласно установленному расписанию. Он также ведёт простое обслуживание --- перезагрузки систем, чистку временных файлов. При возникновении инцидентов оператор фиксирует и регистрирует проблемы для последующего анализа администратором.

Права оператора включают доступ к мониторингу и логам, возможность выполнения простых операций восстановления. При этом оператор не имеет возможности изменять конфигурацию или архитектуру системы, а также управлять правами доступа других пользователей. Наличие квалифицированного оператора позволяет быстро устранять оперативные проблемы, не требуя постоянного вмешательства администратора.

\ManualSubsection{Роль администратора}

Администратор является главным управляющим информационной системой, ответственным за её архитектуру и безопасность. Его деятельность носит стратегический характер и направлена на развитие и защиту ИС.

Функции администратора охватывают управление инфраструктурой, включая установку, конфигурацию и обновление серверов, систем хранения и сетевых устройств. Он осуществляет управление пользователями --- создание и удаление учётных записей, назначение ролей и прав доступа. Защита системы реализуется через внедрение политик безопасности, установку патчей и блокирование атак. Администратор отвечает за развитие ИС --- добавление новых функций и интеграцию с другими системами. В его обязанности входит разработка стратегии резервного копирования и аварийного восстановления, аудит и анализ логов безопасности, планирование и управление изменениями в системе, а также документирование архитектуры и конфигурации.

Права администратора включают полный доступ к системным ресурсам, возможность изменения конфигурации и архитектуры, управление правами доступа всех пользователей и возможность выполнения любых операций в системе. Ответственность администратора распространяется на обеспечение безопасности данных и ресурсов, обеспечение доступности системы для пользователей, соблюдение нормативных требований по информационной безопасности, а также аудит и отчётность по использованию системы.

Важно отметить, что администратор не должен решать текущие проблемы пользователей --- это задача оператора. Администратор должен сосредоточиться на стратегическом развитии и безопасности системы.

\ManualSubsection{Принцип разделения обязанностей}

Правильное распределение ролей реализует принцип разделения обязанностей (Separation of Duties), который является ключевым для безопасности и соответствия регламентам. Согласно этому принципу, один пользователь не должен иметь все права в системе, критические операции должны требовать утверждения несколькими лицами, и один администратор не должен иметь возможность совершить критическую операцию без проверки со стороны другого специалиста.

В качестве практического примера можно рассмотреть финансовую систему, где один администратор настраивает учётные записи пользователей, другой управляет правами доступа к данным, а третий выполняет аудит операций. Такое разделение исключает возможность злоупотреблений и обеспечивает взаимный контроль.

\newpage
\refstepcounter{section}
\addcontentsline{toc}{section}{\protect\numberline{\thesection}Принципы организации доступа к информационным ресурсам}
\begin{center}\textbf{\thesection\ ПРИНЦИПЫ ОРГАНИЗАЦИИ ДОСТУПА К ИНФОРМАЦИОННЫМ РЕСУРСАМ}\end{center}
\vspace*{-1.5em}
\ManualSubsection{Концепция управления доступом}

В условиях цифровой трансформации информация стала ключевым активом, поэтому грамотное управление доступом к информационным ресурсам является критической задачей. \cite{galatenko2011osnovy} определяет управление доступом как основу информационной безопасности.

Организация доступа к информационным ресурсам представляет собой комплекс мер, процедур и правил, обеспечивающих субъектам (пользователям) возможность получения необходимой информации в соответствии с их правами и потребностями, при одновременной защите информации от несанкционированного доступа. Доступ к информационным ресурсам означает возможность субъекта (пользователя, программы, процесса) выполнять определённые операции (чтение, запись, изменение, удаление) над объектом (файлом, базой данных, службой, сетевой папкой).

\ManualSubsection{Триада CIA}

Система контроля доступа обеспечивает три ключевых аспекта информационной безопасности. Конфиденциальность (Confidentiality) гарантирует обеспечение доступа только авторизованным субъектам, исключая возможность ознакомления посторонних лиц с защищаемой информацией. Целостность (Integrity) обеспечивает защиту от несанкционированного изменения или уничтожения данных, гарантируя их точность и непротиворечивость. Доступность (Availability) обеспечивает доступ авторизованным пользователям к информации и ресурсам тогда, когда это необходимо, предотвращая отказы в обслуживании.

\ManualSubsection{Ключевые принципы управления доступом}
\vspace*{-1.5em}
\ManualSubsubsection{Принцип минимальных привилегий}

Пользователям и процессам должны предоставляться только те права, которые абсолютно необходимы для выполнения их обязанностей. \cite{beloglazov2012upravlenie} подробно рассматривает применение этого принципа.

На практике это означает, что обычный пользователь не должен иметь права на установку программного обеспечения. Рядовой сотрудник отдела кадров не должен видеть зарплату всех сотрудников. Веб-приложение должно работать с минимальными правами, необходимыми для выполнения функций. Такой подход резко снижает ущерб от действий и ошибок пользователей, а также от успешных атак, использующих учётные записи с низким уровнем привилегий.

\ManualSubsubsection{Принцип разделения обязанностей}

Критически важные операции должны быть разделены между несколькими пользователями, исключая возможность злоупотреблений одним лицом.

В финансовой системе один пользователь создает платёжное поручение, а другой утверждает его. В IT-инфраструктуре один администратор настраивает службу, а другой управляет учётными записями. В процессе внутреннего контроля разделяются функции автора, проверяющего и утверждающего. Такой подход предотвращает мошенничество и снижает риск ошибок.

\ManualSubsubsection{Принцип «необходимо знать»}

Даже если пользователь имеет доступ к определённому уровню информации, он должен получить доступ только к данным, необходимым для текущей задачи.

Участники проекта имеют доступ к проектной папке, но подпапка с конфиденциальной информацией видна только руководителю. Все работники компании видят корпоративную справку, но зарплату видят только её автор и руководитель. База данных предоставляет доступ только к тем полям, которые необходимы конкретному приложению.

\ManualSubsection{Модели управления доступом}
\vspace*{-1.5em}
\ManualSubsubsection{Дискреционное управление доступом}

При дискреционном управлении доступом (DAC) владелец ресурса самостоятельно решает, кому и какие права предоставить. \cite{sandhu1996role} определяет различные модели и их применение. Классическим примером является система прав на файлы и папки, где пользователь может дать доступ на чтение своему коллеге.

Преимуществами являются гибкость, простота управления в небольших группах и лёгкость делегирования прав. Недостатки включают низкую безопасность, поскольку такую систему легко нарушить по незнанию или умыслу, а также сложность масштабирования для крупных организаций.

\ManualSubsubsection{Мандатное управление доступом}

При мандатном управлении доступом (MAC) доступ жёстко регламентирован политикой безопасности и не может быть изменён владельцем ресурса. Используется в государственных и военных структурах с системой меток безопасности. \cite{bell1973secure} описывает классическую модель Белла-ЛаПадулы.

Правила безопасности устанавливают, что субъект не может читать информацию с более высоким уровнем классификации и не может записывать информацию на более низкий уровень классификации. Преимущества включают высокую степень безопасности и централизованный контроль. Недостатки --- сложность администрирования, негибкость и трудность применения в корпоративной среде.

\ManualSubsubsection{Управление доступом на основе ролей}

Управление доступом на основе ролей (RBAC) является наиболее распространённой моделью в корпоративной среде. \cite{nistRBAC} определяет RBAC как стандарт для корпоративных систем.

Суть модели заключается в том, что пользователям назначаются роли, ролям назначаются права на доступ к ресурсам, и пользователь получает доступ через роль, а не напрямую. Преимущества включают упрощение администрирования (добавление пользователя сводится к назначению роли), масштабируемость (легко управлять правами тысяч пользователей), соответствие бизнес-процессам (роли отражают структуру организации) и упрощение аудита и контроля.

\ManualSubsubsection{Управление доступом на основе атрибутов}

Управление доступом на основе атрибутов (ABAC) представляет собой современную и гибкую модель, принимающую решение о доступе на основе оценки атрибутов.

Атрибуты пользователя включают должность, отдел и уровень безопасности. Атрибуты ресурса охватывают тип файла и метку конфиденциальности. Атрибуты окружения учитывают время суток, местоположение и тип устройства. Преимущества включают максимальную гибкость, контекстную безопасность, поддержку динамических правил и адаптивность к сложным бизнес-процессам.

\ManualSubsection{Протоколы и механизмы реализации}
\vspace*{-1.5em}
\ManualSubsubsection{Аутентификация и авторизация}

LDAP (Lightweight Directory Access Protocol) и Microsoft Active Directory представляют собой централизованное хранилище учётных записей и являются основой RBAC в корпоративных Windows-сетях. \cite{microsoft} предоставляет инструменты управления доступом на основе Active Directory.

Kerberos является сетевым протоколом аутентификации, использующим билеты и защищающим от перехвата паролей. Он описывает версию Kerberos V5 для современных систем.

OAuth 2.0 и OpenID Connect представляют стандарты для авторизации и аутентификации в веб-приложениях и API. \cite{oauth2rfc} определяет фреймворк OAuth 2.0.

SAML (Security Assertion Markup Language) используется для организации единого входа (Single Sign-On, SSO) между разными доменами безопасности. \cite{fstek-order-21} определяет требования к управлению доступом в российских информационных системах.

\ManualSubsubsection{Списки контроля доступа}

ACL (Access Control Lists) представляют собой таблицы, привязанные к объекту, определяющие, какие субъекты и какие операции могут выполнять над ним. ACL является низкоуровневым механизмом реализации политик доступа в файловых системах, сетевых устройствах и базах данных.

\ManualSubsection{Интеграция с администрированием}

Организация доступа неразрывно связана с системным администрированием. Аудит и мониторинг включают ведение журналов доступа и анализ аномалий для выявления подозрительной активности. Управление жизненным циклом учётных записей охватывает создание, изменение и удаление учётных записей при приёме, переводе и увольнении сотрудников. Периодические пересмотры прав обеспечивают проверку соответствия текущим обязанностям. Реагирование на инциденты включает анализ логов доступа при нарушениях безопасности.

Правовое основание для защиты персональных данных в Российской Федерации определяется федеральным законодательством \cite{personaldata}.

\newpage
\refstepcounter{section}
\addcontentsline{toc}{section}{\protect\numberline{\thesection}Основные задачи системного администрирования и поддержания работоспособности ИС}
\begin{center}\textbf{\thesection\ ОСНОВНЫЕ ЗАДАЧИ СИСТЕМНОГО АДМИНИСТРИРОВАНИЯ И ПОДДЕРЖАНИЯ РАБОТОСПОСОБНОСТИ}\end{center}
\vspace*{-1.5em}
\ManualSubsection{Цели и место администрирования}

Системное администрирование представляет собой комплекс практических действий, который делает архитектуру ИС жизнеспособной в реальных условиях. \cite{galatenko_osnovy_ib_2003} определяет основные цели администрирования как обеспечение непрерывной и эффективной работы информационных систем.

Главные цели администрирования включают обеспечение доступности, чтобы системы работали когда они нужны пользователям. Надёжность и целостность данных гарантируют, что данные не теряются и не повреждаются. Безопасность направлена на минимизацию несанкционированного доступа и утечек. Производительность обеспечивает эффективное использование ресурсов. Управляемость и масштабируемость позволяют развивать ИС без хаоса и аварий.

Архитектуру ИС условно делят на уровни: аппаратный (серверы, системы хранения, сеть, инженерная инфраструктура), системный (операционные системы, гипервизоры, СУБД, сервисы каталогов), прикладной (бизнес-приложения, веб-сервисы) и пользовательский (автоматизированные рабочие места сотрудников, мобильные клиенты, браузеры). Системный администратор действует на стыке этих уровней: он разворачивает и настраивает компоненты, контролирует их взаимодействие, а также обеспечивает поддержку в штатных и аварийных ситуациях.

\ManualSubsection{Управление инфраструктурой}

Первый блок задач администратора связан с физической и логической инфраструктурой системы. Установка и конфигурация серверов, сетевого оборудования и систем хранения формируют фундамент информационной системы. Развёртывание и обновление операционных систем, гипервизоров и СУБД обеспечивают программную базу для работы приложений. Планирование ресурсов с запасом под рост нагрузки позволяет избежать проблем производительности в будущем. Управление инженерной инфраструктурой охватывает питание, климат-контроль и пожаротушение.

Без корректно настроенной инфраструктуры даже идеальное программное обеспечение будет работать нестабильно или не запустится вовсе. Поэтому администратор должен уделять первостепенное внимание базовой инфраструктуре.

\ManualSubsection{Управление пользователями и правами доступа}

Администратор напрямую работает с пользователями на этом уровне. Создание, изменение и удаление учётных записей являются повседневными задачами. Назначение и периодический пересмотр прав доступа должны следовать принципу минимально необходимого. Управление группами и ролями организуется в домене, приложениях и СУБД. Особое внимание уделяется контролю при увольнении или переводе сотрудников --- оперативное отключение доступа предотвращает потенциальные угрозы безопасности.

Правильно настроенные права снижают риск утечек, случайных удалений и саботажа, а также упрощают аудит и контроль соответствия требованиям регуляторов.

\ManualSubsection{Резервное копирование и восстановление}

Резервное копирование представляет собой основное средство защиты от потери данных. \cite{cornell_it_incident_request_problem_change_2024} определяет роль резервного копирования в системном администрировании как критически важную.

Задачи администратора в этой области включают определение того, что именно и как часто копировать, с учётом критичности баз данных, файлов и конфигураций. Выбор стратегии охватывает полные, инкрементные и дифференциальные копии. Правило 3-2-1 требует наличия минимум трёх копий данных на двух типах носителей, с одной копией вне основной площадки. Защита резервных копий реализуется через шифрование и контроль доступа. Регулярное тестирование восстановления гарантирует, что резервные копии действительно пригодны для использования.

Наличие проверенной схемы резервного копирования и восстановления позволяет вернуться к работе после серьёзных сбоев, а также уменьшает последствия ошибок администраторов и пользователей.

\ManualSubsection{Мониторинг и управление инцидентами}
\vspace*{-1.5em}
\ManualSubsubsection{Мониторинг и диагностика}

Постоянный контроль состояния серверов, СУБД, сетей и приложений позволяет отслеживать нагрузку, отклик и доступность. Централизованный сбор логов и алертов обеспечивает единую точку анализа состояния системы. Выявление отклонений до того, как пользователи заметят проблему, является признаком зрелой практики мониторинга.

\ManualSubsubsection{Инциденты, проблемы и изменения}

\cite{rezolve_incident_problem_change_2025} определяет различие между этими понятиями в современном ITSM (IT Service Management).

\vspace{1.5em}
\noindent Таблица 5 – Инцидент vs Проблема vs Изменение

{\small
\noindent
\begin{tabular}{|p{0.15\textwidth}|p{0.40\textwidth}|p{0.37\textwidth}|}

\hline
\textbf{Понятие} & \textbf{Определение} & \textbf{Пример} \\
\noalign{\hrule height 1.2pt}
\hline
Инцидент & Неожиданный сбой или ухудшение качества ИТ-сервиса & Недоступность учётной системы \\
\hline
Проблема & Корневая причина одного или нескольких инцидентов & Ошибка конфигурации, дефект ПО \\
\hline
Изменение & Планируемое действие, вносящее правки в конфигурацию & Обновление версии, миграция \\
\hline
\end{tabular}
}

Инцидент представляет собой неожиданный сбой или ухудшение качества ИТ-сервиса, например, недоступность учётной системы. Инцидент требует немедленного реагирования для восстановления сервиса. Проблема является корневой причиной одного или нескольких инцидентов --- это может быть ошибка конфигурации, дефект программного обеспечения или неисправность оборудования. Работа над проблемой направлена на предотвращение повторения инцидентов. Изменение представляет собой планируемое действие, вносящее правки в конфигурацию системы, например, обновление версии, миграция или переразвёртывание.

Роль администратора заключается в быстрой фиксации и классификации инцидента, восстановлении сервиса (пусть временным обходным решением), участии в анализе причин и аккуратной реализации изменений с минимизацией простоев.

Без разделения «чиню прямо сейчас» и «исправляю причину навсегда» организация постоянно живёт в режиме тушения пожаров, что неэффективно и демотивирует команду.

\ManualSubsection{Поддержание работоспособности и профилактика}

Рутинные действия по поддержанию работоспособности часто являются наиболее опасной зоной, где ошибки администратора могут привести к критическим сбоям. Регламентные перезагрузки и обслуживание должны выполняться по чётко определённому графику. Проверка целостности файлов и баз данных позволяет выявить скрытые проблемы. Очистка временных файлов и старых учётных записей поддерживает порядок в системе. Контроль параметров среды включает мониторинг питания, температуры и влажности. Плановая замена оборудования, выходящего за пределы ресурса, предотвращает внезапные отказы.

Профилактический подход значительно эффективнее реактивного: плановые проверки, обновления и тесты восстановления приводят к меньшему количеству аварий, более быстрому восстановлению и прогнозируемой работе системы.

\ManualSubsection{Документирование и взаимодействие}

Для устойчивой работы ИС важны не только технические навыки. Ведение документации по архитектуре, конфигурации и изменениям создаёт базу знаний организации. Создание эксплуатационных регламентов и инструкций стандартизирует процессы. Фиксация инцидентов и решений в базе знаний позволяет учиться на прошлом опыте. Обучение сотрудников базовым правилам работы с ИС снижает количество инцидентов по вине пользователей.

Если знания остаются только в голове администратора, любая его ошибка или увольнение превращаются в серьёзный риск для бизнеса. Документирование является страховкой от таких рисков.

\newpage
\refstepcounter{section}
\addcontentsline{toc}{section}{\protect\numberline{\thesection}Инструменты и практики администрирования в современных вычислительных средах}
\begin{center}\textbf{\thesection\ ИНСТРУМЕНТЫ И ПРАКТИКИ АДМИНИСТРИРОВАНИЯ В СОВРЕМЕННЫХ ВЫЧИСЛИТЕЛЬНЫХ СИСТЕМАХ}\end{center}
\vspace*{-1.5em}
\ManualSubsection{Комплексный набор инструментов администратора}

Современное системное администрирование требует владения широким спектром инструментов. \cite{simplyblock2025infrastructure} определяет основные категории инструментов, необходимых в 2025 году.

\ManualSubsection{Мониторинг}

Мониторинг представляет собой основу проактивного управления инфраструктурой, позволяющую отслеживать состояние систем в реальном времени. Без надлежащего мониторинга администраторы находятся в положении реактивного управления, узнавая о проблемах только от недовольных пользователей.

\ManualSubsubsection{Ключевые метрики}

Доступность серверов (uptime) показывает процент времени, в течение которого система была доступна и функциональна. Использование центрального процессора (CPU) отражает загруженность вычислительных ресурсов, где высокие значения могут указывать на неоптимизированные приложения или DDoS-атаки. Использование оперативной памяти (RAM) критично отслеживать, поскольку её недостаток приводит к использованию медленного дискового пространства (swap). Использование дискового пространства важно контролировать, так как переполнение диска может привести к отказу в работе. Сетевая производительность включает пропускную способность и задержки, влияющие на отклик приложений. Состояние приложений и сервисов показывает, работают ли критические компоненты системы.

\ManualSubsubsection{Инструменты мониторинга}

Zabbix представляет собой комплексное решение с открытым исходным кодом для мониторинга сетей и серверов. Система предоставляет гибкие дашборды и систему алертов, способную отправлять уведомления по различным каналам при срабатывании условий.

Prometheus является системой мониторинга на основе временных рядов, разработанной для облачных окружений и контейнеризации. Система использует pull-модель сбора метрик и прекрасно интегрируется с Kubernetes.

Grafana представляет собой платформу визуализации данных, работающую поверх различных источников данных для создания информативных дашбордов. Её гибкость позволяет объединять данные из множества систем на единой панели управления.

Nagios является надёжным инструментом проверки доступности сетевых сервисов с огромной библиотекой плагинов. Его долгая история использования обеспечивает стабильность и проверенные практики.

\ManualSubsection{Виртуализация и контейнеризация}
\vspace*{-1.5em}
\ManualSubsubsection{Виртуализация}

Виртуализация позволяет запускать несколько виртуальных машин на одном физическом сервере через гипервизор. \cite{scalecomputing2025virtualization} сравнивает виртуализацию и контейнеризацию.

Type 1 (Bare-metal) гипервизоры устанавливаются непосредственно на оборудование без хост-ОС. К ним относятся VMware ESXi, Microsoft Hyper-V и KVM. Они обеспечивают наилучшую производительность и используются в центрах обработки данных.

Type 2 (Hosted) гипервизоры работают как приложение в хост-ОС. Примерами служат VMware Workstation и VirtualBox. Они более удобны для разработки и тестирования, но менее производительны из-за дополнительного слоя операционной системы.

Виртуализацию целесообразно использовать в разнородных окружениях (Windows, Linux, BSD на одном оборудовании), при высоких требованиях к безопасности и изоляции, для legacy-приложений, требующих конкретной ОС, а также при развёртывании on-premises центров обработки данных.

\ManualSubsubsection{Контейнеризация}

Контейнеризация представляет собой более лёгкую форму виртуализации на уровне приложения. Контейнеры разделяют ядро хостовой ОС, но полностью изолируют приложения, библиотеки и зависимости.

Docker является наиболее популярной платформой контейнеризации. Ключевые концепции включают Dockerfile (текстовый файл с инструкциями по созданию образа), Docker-образ (image --- неизменяемый шаблон с приложением и зависимостями), Docker-контейнер (container --- работающий экземпляр образа) и Docker Registry (хранилище образов).

Преимущества Docker включают быстрый запуск контейнеров за секунды или миллисекунды, высокую плотность размещения (сотни контейнеров на одном хосте против десятков виртуальных машин), портативность (контейнер работает идентично в разных средах) и лёгкость (размер образа в сотнях мегабайт против гигабайт для виртуальных машин).

Kubernetes (K8s) представляет собой платформу оркестрации контейнеров, разработанную Google. Если Docker решает проблему упаковки приложения, Kubernetes решает проблему управления множеством контейнеров в распределённой системе. Функции Kubernetes включают автоматическое развёртывание и масштабирование, self-healing (автоматический перезапуск упавших контейнеров), балансировку нагрузки, rolling updates с возможностью отката и управление хранилищем.

\vspace{1.5em}
\noindent Таблица 6 – Виртуализация / Контейнеризация

{\small
\noindent
\begin{tabular}{|p{0.18\textwidth}|p{0.38\textwidth}|p{0.38\textwidth}|}

\hline
\textbf{Аспект} & \textbf{Виртуализация (VM)} & \textbf{Контейнеризация} \\
\noalign{\hrule height 1.2pt}
\hline
Архитектура & Гипервизор + полная ОС для каждой VM & Общее ядро хост-ОС, изолированные приложения \\
\hline
Время запуска & Минуты & Секунды/миллисекунды \\
\hline
Потребление ресурсов & Высокое (несколько GB на VM) & Низкое (сотни MB на контейнер) \\
\hline
Изоляция & Полная (отдельная ОС) & На уровне процессов \\
\hline
Плотность размещения & 10--20 VM на сервер & 100+ контейнеров на сервер \\
\hline
Применение & Legacy-системы, высокая изоляция & Микросервисы, CI/CD, облако \\
\hline
\end{tabular}
}

\ManualSubsection{CI/CD и DevOps-практики}
\vspace*{-1.5em}
\ManualSubsubsection{Определение CI/CD}

Continuous Integration (CI) представляет собой практику частой интеграции кода разработчиков в общий репозиторий с автоматической сборкой и тестированием. Вместо изолированной работы на ветках неделями, разработчики интегрируют изменения часто, что позволяет рано выявлять ошибки.

Continuous Delivery (CD) обеспечивает автоматическую подготовку сборок для развёртывания в production после успешного прохождения всех тестов, хотя само развёртывание требует ручного утверждения.

Continuous Deployment расширяет Continuous Delivery автоматическим развёртыванием успешных сборок в production без ручного вмешательства.

\ManualSubsubsection{Популярные инструменты CI/CD}

Jenkins является одним из самых популярных open-source инструментов CI/CD, отличающимся гибкостью и расширяемостью через плагины. GitLab CI/CD встроен в GitLab и конфигурируется через файл .gitlab-ci.yml в репозитории. GitHub Actions встроены в GitHub и отличаются простотой использования и хорошей интеграцией с экосистемой. Azure DevOps представляет комплексный набор инструментов от Microsoft для управления разработкой.

Преимущества CI/CD включают скорость выхода на рынок (организации с развитым CI/CD развёртываются значительно чаще), качество кода (автоматические тесты выявляют ошибки сразу), снижение рисков (маленькие частые изменения легче отлаживать) и надёжность (автоматизация исключает ошибки ручного развёртывания).

\ManualSubsection{Облачные модели обслуживания}
\vspace*{-1.5em}
\ManualSubsubsection{IaaS (Infrastructure as a Service)}

IaaS предоставляет виртуализированные вычислительные ресурсы через интернет: виртуальные серверы, хранилище и сетевую инфраструктуру. Пользователь управляет операционной системой, установленным ПО, приложениями и данными. Провайдер управляет оборудованием, виртуализацией и сетью на физическом уровне. Эта модель обеспечивает максимальную гибкость и контроль над инфраструктурой. Примерами служат AWS EC2, Google Compute Engine и Microsoft Azure VMs.

\ManualSubsubsection{PaaS (Platform as a Service)}

PaaS предоставляет платформу для разработки, развёртывания и управления приложениями без необходимости управления базовой инфраструктурой. Провайдер управляет всем: операционной системой, middleware и runtime-окружением. Пользователь фокусируется на коде приложения и данных. Встроенные инструменты включают CI/CD, мониторинг и автомасштабирование. Примерами служат Heroku, Google App Engine и AWS Elastic Beanstalk.

\ManualSubsubsection{SaaS (Software as a Service)}

SaaS предоставляет готовые приложения, работающие в облаке провайдера, доступные через веб-браузер по подписке. Провайдер управляет всем стеком: инфраструктурой, ПО, данными, безопасностью и обновлениями. Пользователь только использует приложение. Модель обеспечивает быстрое внедрение и минимальные требования к IT-персоналу. Примерами служат Microsoft 365, Google Workspace и Salesforce CRM.

\ManualSubsection{Современные тренды}
\vspace*{-1.5em}
\ManualSubsubsection{GitOps}

GitOps представляет методологию, где Git-репозиторий является единственным источником истины для инфраструктуры и приложений. Все изменения конфигурации хранятся в Git, а система автоматически синхронизирует текущее состояние с желаемым состоянием, описанным в Git. Изменения вносятся через pull requests с рецензированием, обеспечивая аудитный след и возможность отката.

\ManualSubsubsection{SRE (Site Reliability Engineering)}

SRE представляет подход, разработанный Google, который применяет инженерные принципы к операционной деятельности. Error budgets определяют допустимый downtime: если система не использует весь бюджет ошибок, можно рискнуть развёртыванием новых функций; если бюджет исчерпан, приоритет отдаётся стабильности. Toil reduction направлен на минимизацию однообразной ручной работы и повышение автоматизации. Blameless postmortems проводятся после инцидентов без поиска виноватого, с фокусом на предотвращении подобных ситуаций в будущем.

\ManualSubsubsection{Zero Trust Security}

Zero Trust переворачивает классический подход к безопасности: каждый запрос, независимо от источника, должен быть проверен. Принципы включают проверку каждого запроса (даже от внутренней сети), микросегментацию (разделение сети на малые сегменты с ограниченным доступом), минимальные привилегии (каждый пользователь и устройство получает только необходимые права) и непрерывную верификацию (постоянную проверку состояния устройств).

\newpage
\section*{}
\vspace*{-2.5em}
\begin{center}\textbf{ЗАКЛЮЧЕНИЕ}\end{center}
\phantomsection
\addcontentsline{toc}{section}{Заключение}

Основы архитектуры информационных систем и администрирования представляют собой комплексный набор принципов, практик и технологий, необходимых для построения надёжных, безопасных и масштабируемых систем.

Ключевые компоненты архитектуры --- уровень представления, бизнес-логики и данных --- взаимодействуют через чётко определённые интерфейсы и API. \cite{microsoft2023architecture} \cite{aws2023wellarchitected} определяют современные подходы к проектированию архитектур.

Роли серверов, клиентов и сетевых устройств, правильное распределение функций между пользователями, администраторами и операторами создают основу для эффективной и безопасной работы ИС.

Управление доступом к информационным ресурсам на основе принципов минимальных привилегий и разделения обязанностей обеспечивает конфиденциальность, целостность и доступность данных.

Системное администрирование --- это постоянный процесс поддержания работоспособности, безопасности и развития информационной системы. Современные администраторы должны владеть инструментами мониторинга, виртуализации, контейнеризации, автоматизации и облачных технологий, применяя при этом лучшие практики DevOps, SRE и Infrastructure as Code.

Успешная информационная система --- это результат правильной архитектуры, квалифицированного администрирования и соблюдения принципов безопасности и надёжности на всех уровнях.

\newpage
\begin{center}\textbf{СПИСОК СОКРАЩЕНИЙ}\end{center}
\phantomsection
\addcontentsline{toc}{section}{Список сокращений}

\vspace{1em}

{
\small
\noindent
\begin{tabular}{|l|p{0.75\textwidth}|}
\hline
\textbf{Сокращение} & \multicolumn{1}{c|}{\textbf{Расшифровка}} \\
\noalign{\hrule height 1.2pt}
\hline
ABAC & Attribute-Based Access Control (управление доступом на основе атрибутов)\\
\hline
ACL & Access Control List (список контроля доступа)\\
\hline
API & Application Programming Interface (интерфейс прикладной программы)\\
\hline
CI/CD & Continuous Integration / Continuous Delivery (непрерывная интеграция / доставка)\\
\hline
DAC & Discretionary Access Control (дискреционное управление доступом)\\
\hline
DevOps & Development \& Operations (разработка и эксплуатация)\\
\hline
IaaS & Infrastructure as a Service (инфраструктура как услуга)\\
\hline
ИС & информационная система\\
\hline
LDAP & Lightweight Directory Access Protocol (облегчённый протокол доступа к директориям)\\
\hline
MAC & Mandatory Access Control (мандатное управление доступом)\\
\hline
ОС & операционная система\\
\hline
PaaS & Platform as a Service (платформа как услуга)\\
\hline
RBAC & Role-Based Access Control (управление доступом на основе ролей)\\
\hline
RPO & Recovery Point Objective (целевая точка восстановления)\\
\hline
RTO & Recovery Time Objective (целевое время восстановления)\\
\hline
SaaS & Software as a Service (программное обеспечение как услуга)\\
\hline
SoD & Separation of Duties (разделение обязанностей)\\
\hline
SRE & Site Reliability Engineering (инжиниринг надёжности сайтов)\\
\hline
SSH & Secure Shell (защищённая оболочка)\\
\hline
SSL/TLS & Secure Sockets Layer / Transport Layer Security (протоколы шифрования)\\
\hline
SSO & Single Sign-On (единый вход)\\
\hline
СУБД & система управления базой данных\\
\hline
VPN & Virtual Private Network (виртуальная приватная сеть)\\
\hline
VM & Virtual Machine (виртуальная машина)\\
\hline
\end{tabular}
}

\clearpage
\nocite{*}
\printbibliography[heading=bibliography]

\end{document}
