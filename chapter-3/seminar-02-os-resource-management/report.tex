% реферат по теме 'Операционные системы и управление ресурсами вычислительной техники'
\documentclass{SibFU-docs}

% доп. пакеты
\usepackage{graphicx}
\usepackage{float}
\usepackage{hyperref}
\usepackage{caption}
\usepackage{subcaption}
\usepackage{amsmath}
% предотвартим ошибку с \Bbbk перед загрузкой пакета amssymb
\let\Bbbk\relax
\usepackage{amssymb}
\usepackage{booktabs}
\usepackage{tabularx}
\usepackage{longtable}
\usepackage{enumitem}
\usepackage{multirow}

% гиперссылки
\hypersetup{
    colorlinks=true,
    linkcolor=black,
    citecolor=blue,
    urlcolor=blue
}

% библиография
\addbibresource{report.bib}

% титульный лист
\begin{document}
\setlist[itemize]{left=1.3cm, itemsep=1pt, topsep=2pt} % для выравнивания списков
\setlist[enumerate]{left=1.3cm, itemsep=1pt, topsep=2pt}
\renewcommand{\contentsname}{}
\mycovertitle
{Гуманитарный институт}
{Кафедра прикладной информатики в искусстве и интерактивном медиа}
{Операционные системы и управление ресурсами вычислительной техники}
{ст. преподаватель И.~Р.~Нигматуллин}
{Логинова Ю.В., Фиряго О.А., Захаров И.М., Якубовская В.Г.,\\Дегтярева В.В., Арсенян А.В.}
 
\begin{center}
\bfseries РЕФЕРАТ
\end{center}
\noindent

Реферат по теме: «Операционные системы и управление ресурсами вычислительной техники» содержит 31 страниц и 40 использованных источников.

Ключевые слова: ОПЕРАЦИОННАЯ СИСТЕМА, КЛАССИФИКАЦИЯ, АРХИТЕКТУРА, МНОГОПОЛЬЗОВАТЕЛЬСКАЯ СИСТЕМА, МНОГОЗАДАЧНОСТЬ, УПРАВЛЕНИЕ, РЕСУРС, ПРОЦЕСС, ФАЙЛОВАЯ СИСТЕМА, БЕЗОПАСНОСТЬ ВЫЧИСЛИТЕЛЬНЫХ СИСТЕМ, АППАРАТНОЕ ОБЕСПЕЧЕНИЕ, СРЕДСТВА МОНИТОРИНГА, ОПТИМИЗАЦИЯ.

В работе исследуются основные принципы функционирования операционных систем и методы управления ресурсами вычислительной техники. Рассмотрены назначение и ключевые функции операционных систем, механизмы управления процессами, памятью и файловыми системами, а также взаимодействие операционной системы с аппаратным обеспечением. Приведены особенности многопользовательских и многозадачных систем, классификация операционных систем по архитектуре и назначению, а также средства мониторинга производительности и оптимизации работы. Отдельное внимание уделено вопросам сетевой архитектуры и безопасности вычислительных систем.

Основные выводы:

\begin{enumerate}
    \item Операционная система как фундаментальный посредник между пользователем и аппаратным обеспечением, обеспечивая эффективное управление ресурсами (процессор, память, устройства ввода-вывода) и предоставляя удобный интерфейс для работы с компьютером.
    
    \item Взаимодействие с аппаратным обеспечением через многоуровневую абстракцию (драйверы, HAL). Обчеспечение портируемости приложений за счёт стандартных интерфейсов.
    
    \item Классификация операционных систем по архитектуре (монолитная, микроядерная, гибридная) и назначению (настольные, серверные, мобильные, встраиваемые, реального времени) определяет их функциональность, производительность и область применения.

    \item Средства мониторинга производительности (диспетчер задач, системные мониторы, утилиты командной строки) и методы оптимизации (виртуальная память, управление процессами, кэширование) позволяют поддерживать стабильность и эффективность системы при различных рабочих нагрузках.
\end{enumerate}

Рекомендации:

\begin{itemize}[label=-]
    \item Спроектировать архитектуру операционной системы с разделением на ядро, системные модули и прикладной уровень для обеспечения надёжности и безопасности.
    \item Реализовать систему управления процессами с динамическими приоритетами и виртуальной памятью для поддержки многозадачной работы.
    \item Внедрить журналируемые файловые системы и унифицированные интерфейсы для взаимодействия с оборудованием. 
    \item Использовать средства мониторинга ресурсов и разграничения доступа в многопользовательской среде.
\end{itemize}

\newpage

\thispagestyle{empty}
\begin{center}
\bfseries СОДЕРЖАНИЕ
\end{center}
\vspace{-6pt}
\tableofcontents
\clearpage

% фантомные боли для секций, чтобы не было нумераций кроме тела реферата
\section*{}
\vspace*{-2.5em}
\begin{center}\textbf{ВВЕДЕНИЕ}\end{center}
\phantomsection
\addcontentsline{toc}{section}{Введение}

Операционные системы (ОС) являются фундаментальным компонентом вычислительной техники, обеспечивая взаимодействие между пользователем, программным обеспечением и аппаратными ресурсами. Их эффективное функционирование определяет производительность, стабильность и безопасность всей компьютерной системы. Грамотное управление ресурсами вычислительной техники на уровне операционной системы — ключевой фактор для решения современных вычислительных задач.

Современное состояние проблемы. Развитие операционных систем находится в постоянной динамике, где классические архитектурные подходы (монолитная, микроядерная) сосуществуют с гибридными решениями. Функционал управления процессами и памятью эволюционирует в сторону большей эффективности и изоляции, а модели многопользовательского доступа усложняются для соответствия современным требованиям безопасности. Инструменты мониторинга трансформируются от базовых системных утилит к комплексным платформам аналитики в реальном времени.

Актуальность обусловлена повсеместным использованием операционных систем во всех сферах деятельности, ростом требований к производительности и энергоэффективности вычислительных систем, усилением угроз информационной безопасности и необходимостью оптимизации использования дорогостоящих аппаратных ресурсов.

Работы заключается в комплексном рассмотрении взаимосвязи между архитектурными принципами построения ОС, механизмами управления её ключевыми ресурсами (процессы, память, файлы) и практическими аспектами их мониторинга и оптимизации в единой логической цепочке.

Цель работы: исследование архитектурных принципов, ключевых функций и механизмов управления ресурсами в современных операционных системах, а также методов оценки и повышения их эффективности.

Задачи: определить назначение и базовые функции ОС в компьютерной среде; проанализировать механизмы управления процессами, памятью и файловыми системами; исследовать принципы взаимодействия ОС с аппаратным обеспечением; классифицировать ОС по архитектуре и целевому назначению; изучить основы организации многопользовательских и многозадачных вычислений; рассмотреть современные средства мониторинга производительности и оптимизации работы системы.

Методы: анализ архитектуры и компонентов ОС, сравнительный анализ классификационных признаков и механизмов управления, анализ документации производителей и академических источников, практический анализ показателей производительности.

\newpage
% вручную увеличить номер секции и записать в оглавление с номером,
% чтобы в теле не печатался автоматический номер, но в tableofcontents он остался
\refstepcounter{section}
\addcontentsline{toc}{section}{\protect\numberline{\thesection} Назначение и функции операционной системы в компьютерной среде}
\vspace*{-2.5em}
\begin{center}\textbf{\thesection\ НАЗНАЧЕНИЕ И ФУНКЦИИ ОПЕРАЦИОННОЙ СИСТЕМЫ В КОМПЬЮТЕРНОЙ СРЕДЕ}\end{center}
\vspace*{-1.5em}
\ManualSubsection{Определение операционной системы}

Операционная система (ОС) представляет собой комплекс программ, который является фундаментальным компонентом вычислительной системы, загружаемым при её включении и обеспечивающим возможность работы всех других программ \cite{silberschatz2018operating}. Без операционной системы компьютер представляет собой лишь набор неорганизованных аппаратных компонентов, неспособных к выполнению полезных задач. Изначально разработанная как система управления вводом-выводом, эволюция ОС привела к её становлению в качестве универсального посредника между пользователем и вычислительными ресурсами \cite{denning1971third}. Визуально её интерфейс может быть представлен графической оболочкой с элементами управления, однако её сущность заключается в программном обеспечении, осуществляющем базовое управление аппаратурой.

\ManualSubsubsection{Функция посредника между пользователем и аппаратным обеспечением}

Ключевая роль операционной системы заключается в обеспечении интерфейса между пользователем и сложным аппаратным обеспечением компьютера. Пользователь взаимодействует с системой через удобные абстракции — графические значки, команды, сенсорные жесты, — в то время как аппаратные компоненты, такие как центральный процессор (CPU), оперативная память (ОЗУ) и накопители, функционируют на уровне электрических сигналов и машинных кодов. Операционная система выполняет роль транслятора, принимая высокоуровневые запросы пользователя и преобразуя их в последовательности низкоуровневых команд, исполняемых аппаратурой \cite{tanenbaum2015modern}. Эта абстракция освобождает пользователя от необходимости понимания деталей работы оборудования для выполнения стандартных задач.

\ManualSubsubsection{Управление ресурсами компьютера}

Операционная система выступает в качестве диспетчера, ответственного за эффективное и справедливое распределение ограниченных ресурсов вычислительной системы между конкурирующими процессами. К таким ресурсам относится процессорное время, объём оперативной памяти, пространство на накопителях и пропускная способность устройств ввода-вывода \cite{stallings2018operating}. Основной механизм управления, известный как многозадачность, позволяет ОС создавать видимость одновременного выполнения нескольких программ, переключая контекст процессора между ними и выделяя каждой необходимые ресурсы. Это предотвращает ситуации, когда одна программа может монополизировать ресурсы, приводя к отказу в обслуживании для других задач или к нестабильности всей системы.

\ManualSubsection{Обеспечение работы прикладных программ}

Операционная система предоставляет стабильную и унифицированную платформу для выполнения прикладных программ. Вместо того чтобы требовать от каждого приложения прямого управления аппаратными устройствами, ОС предлагает стандартизированный набор сервисов и интерфейсов программирования приложений (API). Эти API служат абстрактным слоем, через который программы запрашивают доступ к памяти, файловым операциям, сетевым соединениям или устройствам отображения \cite{silberschatz2018operating}. Такой подход не только значительно упрощает процесс разработки программного обеспечения, но и обеспечивает переносимость программ в рамках одной и той же операционной среды, а также повышает безопасность и стабильность системы, контролируя доступ приложений к критическим ресурсам.

\ManualSubsubsection{Организация данных}

Для удобного и надёжного хранения информации операционная система реализует файловую систему, которая является структурой для организации, именования, хранения и извлечения данных на долговременных носителях. Она предоставляет пользователю абстрактное представление данных в виде файлов, сгруппированных в иерархические каталоги (папки), скрывая сложности физического размещения информации на диске (HDD или SSD) \cite{tanenbaum2015modern}. Файловая система отвечает за управление свободным пространством, отслеживание местоположения фрагментов файлов, обеспечение целостности данных и контроль прав доступа к ним, формируя основу для работы с пользовательской и системной информацией.


\ManualSubsection{Безопасность и управление доступом}

В многопользовательских средах операционная система реализует комплексные механизмы разграничения доступа и обеспечения безопасности. Это достигается через систему учётных записей и профилей пользователей, где для каждого пользователя могут быть определены индивидуальные настройки рабочей среды, права доступа к файлам и директориям, а также разрешения на выполнение определённых действий (например, установку программ) \cite{stallings2018operating}. ОС обеспечивает фундаментальные процедуры безопасности: аутентификацию для проверки подлинности пользователя, авторизацию для контроля его запросов к ресурсам и изоляцию процессов друг от друга для предотвращения взаимного вмешательства. Эти базовые механизмы служат фундаментом для защиты данных от несанкционированного доступа. Помимо управления правами пользователей, современные операционные системы реализуют более тонкие механизмы безопасности, такие как обязательный контроль доступа (Mandatory Access Control — MAC), используемый для защиты систем с высокими требованиями к секретности \cite{silberschatz2018operating}. Важным аспектом также является управление привилегиями процессов, где ОС делит их на пользовательский и привилегированный (ядерный) режимы, строго ограничивая возможность обычных приложений выполнять критически важные операции с оборудованием или памятью. Для защиты целостности системы применяются механизмы цифровой подписи и проверки драйверов устройств, предотвращающие загрузку непроверенного кода в ядро. Эти механизмы в совокупности создают многоуровневую модель защиты, которая является неотъемлемой частью архитектуры любой современной операционной системы.

\newpage
\refstepcounter{section}
\addcontentsline{toc}{section}{\protect\numberline{\thesection}Управление процессами, памятью и файловыми системами}
\begin{center}\textbf{\thesection\ УПРАВЛЕНИЕ ПРОЦЕССАМИ, ПАМЯТЬЮ И ФАЙЛОВЫМИ СИСТЕМАМИ}\end{center}
\vspace*{-1.5em}
\ManualSubsection{Управление процессами: выполнение программ}

При запуске программы операционная система создаёт процесс — сущность, включающую исполняемый код, данные и контекст выполнения. Каждому процессу присваивается уникальный идентификатор (Process ID, PID), позволяющий системе однозначно его идентифицировать. Информация о всех активных процессах хранится в специальной структуре ядра, называемой таблицей процессов \cite{silberschatz2018operating}. Поскольку количество доступных процессорных ядер, как правило, меньше числа выполняемых программ, ключевой задачей ОС является реализация многозадачности. Достигается это работой планировщика процессов, который, переключая процессор между задачами с высокой частотой, создаёт иллюзию их параллельного выполнения \cite{tanenbaum2015modern}. Состояние процесса динамично изменяется, проходя этапы готовности, выполнения и ожидания.

Для корректного взаимодействия параллельных процессов ОС предоставляет механизмы межпроцессного взаимодействия (IPC), такие как разделяемая память, каналы или очереди сообщений. При работе с общими ресурсами для предотвращения конфликтов и состояний гонки используются примитивы синхронизации, в число которых входят семафоры и мьютексы, теоретическая основа которых была заложена в работе \cite{hoare1974monitors}. В современных системах, таких как Windows, эта модель расширена понятием потоков (threads), которые являются более легковесными единицами выполнения внутри процесса \cite{microsoft-processes-threads}. Стандартизированный интерфейс для управления процессами и потоками определён в стандартах POSIX \cite{posix-standard}.

\ManualSubsection{Управление памятью: организация данных в ОЗУ}

Управление оперативной памятью является критически важной функцией ОС, обеспечивающей изоляцию и эффективное распределение ресурсов между конкурирующими процессами. Фундаментальным концептом здесь выступает виртуальная память. Каждому процессу выделяется собственное виртуальное адресное пространство, которое отображается на физическую память и пространство на диске через механизм подкачки (swapping) \cite{denning1970virtual}. Это создаёт у процесса иллюзию эксклюзивного владения обширным, непрерывным адресным пространством, одновременно защищая память одного процесса от вмешательства другого \cite{microsoft-virtual-memory}.

Физическая память разделена на фиксированные блоки — страницы (frames), а виртуальное пространство — на блоки аналогичного размера (pages). При обращении к виртуальному адресу, страница которого отсутствует в ОЗУ, возникает страничное прерывание (page fault), и требуемая страница подгружается с диска. Для выбора страницы-кандидата на выгрузку при нехватке памяти применяются алгоритмы замещения, такие как Least Recently Used (LRU) или First-In-First-Out (FIFO) \cite{stallings2018operating}. Побочным эффектом длительной работы системы с памятью является фрагментация, которая подразделяется на внутреннюю (неэффективное использование выделенного блока) и внешнюю (наличие множества небольших разрозненных свободных участков).

\ManualSubsection{Файловые системы: хранение информации на дисках}

Файловая система абстрагирует физическую организацию данных на накопителях, предоставляя пользователю и приложениям логическое представление в виде иерархии файлов и каталогов. Её основные задачи включают именование, организацию хранения, обеспечение надёжности и эффективного доступа к данным. При создании файла система не только записывает его содержимое в свободные блоки на диске, но и создаёт запись с метаданными. В UNIX-подобных системах для этого служит структура inode, хранящая атрибуты файла и указатели на блоки данных \cite{bovet2005understanding}. В файловой системе NTFS аналогичную функцию выполняют записи главной файловой таблицы (Master File Table, MFT) \cite{silberschatz2018operating}.

Для повышения производительности ОС активно использует кэширование: часто запрашиваемые данные с диска хранятся в оперативной памяти, что сокращает количество медленных операций ввода-вывода. Современные файловые системы, такие как NTFS, ext4 или APFS, реализуют механизм журналирования (journaling) для повышения отказоустойчивости. Перед внесением изменений в основную структуру файловой системы они сначала фиксируются в специальном журнале, что позволяет восстановить целостность после внезапного сбоя \cite{linux-filesystems}. Стоит отметить, что различные файловые системы предлагают разные компромиссы между производительностью, надёжностью и функциональностью. Например, некоторые системы используют метод копирования при записи (Copy-on-Write) для предотвращения повреждения данных, в то время как другие оптимизированы для работы с большими файлами или высокой степени параллелизма операций. Выбор файловой системы является важным аспектом конфигурации, влияющим на общую стабильность и отзывчивость операционной системы \cite{tanenbaum2015modern}.

\ManualSubsection{Взаимосвязь компонентов системы}

Три рассмотренные подсистемы — управление процессами, памятью и файлами — функционируют не изолированно, а в тесной взаимосвязи, формируя целостную среду выполнения. Последовательность запуска программы наглядно демонстрирует эту интеграцию: файловая система считывает исполняемый файл с диска, подсистема управления памятью выделяет для него виртуальное адресное пространство и загружает код, после чего диспетчер процессов создаёт структуры данных процесса и ставит его в очередь на выполнение \cite{tanenbaum2015modern}. В ходе работы процессы постоянно инициируют обращения к файлам, что требует координации между менеджером памяти (который кэширует данные) и файловой подсистемой. Нехватка оперативной памяти приводит к активизации механизма подкачки, который, в свою очередь, генерирует нагрузку на диск.

Эта взаимозависимость создаёт сложные причинно-следственные связи при настройке и оптимизации системы. Например, увеличение размера дискового кэша может ускорить операции ввода-вывода, но одновременно сократить объём оперативной памяти, доступной для исполняющихся процессов, что потенциально увеличит частоту страничных прерываний и снизит общую производительность \cite{silberschatz2018operating}. Аналогично, агрессивная политика планировщика процессов, направленная на максимальную утилизацию процессора, может приводить к частым переключениям контекста, каждое из которых требует сохранения и восстановления состояния регистров и данных процесса в памяти. Таким образом, операционная система выступает как сложный координатор, балансирующий запросы и ресурсы между всеми компонентами для обеспечения стабильной и эффективной работы вычислительной системы в целом. Понимание этих глубинных взаимосвязей является ключевым для системного администрирования и тонкой настройки производительности \cite{stallings2018operating}.

\newpage
\refstepcounter{section}
\addcontentsline{toc}{section}{\protect\numberline{\thesection}Взаимодействие операционной системы с аппаратным обеспечением}
\begin{center}\textbf{\thesection\ ВЗАИМОДЕЙСТВИЕ ОПЕРАЦИОННОЙ СИСТЕМЫ С АППАРАТНЫМ ОБЕСПЕЧЕНИЕМ}\end{center}
\vspace*{-1.5em}
\ManualSubsection{Роль операционной системы как интерфейса и менеджера ресурсов}

Современный компьютер представляет собой комплексную систему, образуемую аппаратными компонентами и программным обеспечением. Аппаратное обеспечение включает физические устройства, такие как процессор, оперативная память и накопители. Программное обеспечение состоит из инструкций, управляющих этими устройствами. Операционная система (ОС) выступает ключевым программным элементом, выполняющим функции посредника между пользователем, прикладными программами и аппаратурой, обеспечивая абстракцию и эффективное распределение вычислительных ресурсов \cite{studfile2023os}.

Основное назначение ОС в данном контексте — предоставление унифицированных интерфейсов, скрывающих сложные детали работы аппаратуры от приложений. Эта идея абстракции позволяет программистам работать с обобщёнными понятиями, такими как «файл» или «процесс», не вдаваясь в специфику конкретных устройств. Параллельно ОС реализует стратегии управления ресурсами, распределяя процессорное время, память и доступ к устройствам ввода-вывода между множеством конкурирующих процессов, что предотвращает конфликты и обеспечивает стабильность работы всей системы \cite{cyberleninka2022interaction}.

\ManualSubsubsection{Ключевые механизмы взаимодействия}

Взаимодействие операционной системы с аппаратной частью реализуется через ряд специализированных механизмов. Драйверы устройств представляют собой программные модули, детально «знающие» конкретное оборудование. Они выполняют трансляцию общих команд ОС в специфические инструкции, понятные определённой модели устройства, выступая обязательным связующим звеном \cite{vstack2023os}. Системные вызовы служат безопасным способом для прикладных программ запросить услуги ядра ОС, связанные с доступом к ресурсам. Когда программе требуется выполнить операцию ввода-вывода, она инициирует системный вызов, передавая управление привилегированному ядру ОС, которое взаимодействует с драйверами.

Прерывания — это аппаратные сигналы, посредством которых устройства уведомляют процессор о наступлении важного события. При получении такого сигнала процессор приостанавливает текущую задачу и передаёт управление заранее определённому обработчику прерывания в ядре ОС. Этот асинхронный механизм позволяет системе эффективно реагировать на внешние события, не прибегая к постоянному опросу устройств. Для разгрузки центрального процессора от рутинных операций копирования данных используется механизм прямого доступа к памяти (DMA). Контроллер DMA позволяет периферийным устройствам обмениваться данными с оперативной памятью напрямую, минимизируя участие CPU \cite{cyberleninka2022interaction}.

\ManualSubsection{Управление основными аппаратными ресурсами}

Взаимодействие с процессором осуществляется через планировщик задач, который определяет очерёдность и длительность выполнения процессов. Работа планировщика активируется прерываниями от системного таймера. Важным аспектом является управление режимами работы процессора, где ОС обеспечивает переключение между привилегированным режимом для выполнения кода ядра и пользовательским режимом для работы приложений, что является основой защиты системы.

Управление оперативной памятью возложено на менеджер памяти. Он отвечает за выделение и освобождение памяти для процессов, отслеживание её использования и реализацию механизмов защиты, изолирующих память одного процесса от другого. Для расширения адресуемого пространства ОС использует концепцию виртуальной памяти, при которой часть данных, не помещающихся в физическую память, временно выгружается на диск, создавая у программ иллюзию работы с большим объёмом ОЗУ \cite{studfile2023os}.

Взаимодействие с устройствами ввода-вывода строится на связке драйверов и системы прерываний. ОС предоставляет абстракции, такие как файловая система для накопителей, которая организует данные в виде иерархии файлов и каталогов, скрывая от пользователя физическую структуру секторов. Управление шинами и периферийными устройствами начинается с взаимодействия с микропрограммным обеспечением материнской платы (BIOS/UEFI) на этапе загрузки, после чего ОС через драйверы шин обнаруживает и конфигурирует подключённое оборудование \cite{vstack2023os}.

\ManualSubsection{Процесс загрузки как комплексный пример взаимодействия}

Процесс начальной загрузки компьютера наглядно демонстрирует последовательное и сложное взаимодействие программного обеспечения с аппаратными компонентами \cite{habr2023boot}. После включения питания управление получает код, записанный в постоянной памяти материнской платы (BIOS или UEFI), который выполняет самопроверку оборудования. Далее этот код находит на загрузочном диске специальную программу — загрузчик, и загружает её в оперативную память.

Загрузчик, в свою очередь, находит на диске ядро операционной системы, загружает его в память и передаёт ему управление. Ядро инициализирует критически важные подсистемы: загружает необходимые драйверы устройств, настраивает менеджер памяти, запускает планировщик процессов и системные службы. Завершающим этапом является запуск пользовательского интерфейса и назначенных для автоматического запуска приложений. Каждый шаг этой цепочки представляет собой результат координированной работы программных инструкций и аппаратных механизмов \cite{lecture2021boot}.

\newpage
\refstepcounter{section}
\addcontentsline{toc}{section}{\protect\numberline{\thesection}Классификация операционных систем по архитектуре и назначению}
\begin{center}\textbf{\thesection\ КЛАССИФИКАЦИЯ ОПЕРАЦИОННЫХ СИСТЕМ ПО АРХИТЕКТУРЕ И НАЗНАЧЕНИЮ}\end{center}
\vspace*{-1.5em}

\ManualSubsection{Классификация по архитектуре ядра}

Архитектура ядра является фундаментальным признаком, определяющим внутреннюю организацию операционной системы, принципы взаимодействия её компонентов и границы между привилегированным режимом работы и пространством пользователя \cite{silberschatz2018operating}. Различия в архитектуре напрямую влияют на ключевые характеристики системы: производительность, надёжность, безопасность и сопровождаемость.

\ManualSubsubsection{Монолитная архитектура}

Монолитная архитектура характеризуется объединением всех основных служб операционной системы — управления процессами, памятью, файловой системой и драйверами устройств — в единое целое, работающее в адресном пространстве ядра. Компоненты взаимодействуют путём прямых вызовов функций, что минимизирует накладные расходы и обеспечивает высокую производительность \cite{bovet2005understanding}. Классическими примерами являются UNIX и Linux. Однако данный подход усложняет модернизацию и отладку отдельных компонентов, а сбой в драйвере может привести к краху всей системы \cite{stallings2018operating}.

\ManualSubsubsection{Микроядерная архитектура}

Микроядерная архитектура следует принципу минимизации кода, выполняемого в привилегированном режиме. Ядро предоставляет лишь элементарные механизмы: базовое управление процессами, виртуальной памятью и межпроцессное взаимодействие. Все остальные службы, включая драйверы и файловые системы, реализуются как пользовательские процессы \cite{liedtke1995microkernel}. Это повышает надёжность и стабильность, так как ошибка в сервисе не затрагивает ядро. Примерами служат QNX, применяемая в системах реального времени \cite{qnx-microkernel}, и учебная система MINIX \cite{tanenbaum2006minix}. Основным компромиссом является потенциальное снижение производительности из-за частого переключения контекста.

\ManualSubsubsection{Гибридная архитектура}

Гибридная архитектура представляет собой компромиссный подход, объединяющий черты монолитной и микроядерной моделей. Критически важные для производительности компоненты, такие как часть драйверов или подсистема графического вывода, остаются в пространстве ядра, в то время как другие службы могут быть вынесены в пользовательское пространство. Это позволяет достичь баланса между быстродействием, стабильностью и модульностью. Наиболее известным примером операционной системы с гибридной архитектурой является Microsoft Windows \cite{microsoft-windows-architecture}.

\ManualSubsection{Классификация по назначению и сфере применения}

Целевое назначение операционной системы определяет набор предъявляемых к ней требований и, как следствие, её архитектурные и функциональные особенности. Данная классификация позволяет выделить системы, оптимизированные для конкретных условий эксплуатации.

\ManualSubsubsection{Операционные системы общего назначения}

Операционные системы общего назначения, такие как Microsoft Windows, macOS и дистрибутивы Linux, предназначены для широкого спектра задач на персональных компьютерах, рабочих станциях и серверах. Их ключевыми характеристиками являются универсальность, поддержка разнообразного аппаратного и программного обеспечения, развитый графический интерфейс и эффективная многозадачность \cite{tanenbaum2015modern}. Они оптимизированы для обеспечения высокой интерактивной производительности в условиях динамически меняющейся рабочей нагрузки.

\ManualSubsubsection{Операционные системы реального времени}

Операционные системы реального времени проектируются для применения в системах, где критически важно соблюдение жёстких временных ограничений. Они гарантируют выполнение задач в предсказуемые сроки. Различают системы жёсткого реального времени, где невыполнение срока является недопустимым отказом (например, в управлении промышленными установками), и системы мягкого реального времени, где временные задержки допустимы, но нежелательны. Типичными представителями являются QNX и VxWorks \cite{stallings2018operating}.

\ManualSubsubsection{Встраиваемые операционные системы}

Встраиваемые операционные системы функционируют в рамках специализированных устройств с ограниченными вычислительными ресурсами, таких как маршрутизаторы, медицинские приборы или системы «умного дома». Они характеризуются компактностью, модульностью, низким энергопотреблением и часто поддерживают только строго необходимый для устройства набор функций. Примерами могут служить различные варианты Embedded Linux, FreeRTOS или специализированные версии QNX \cite{silberschatz2018operating}.

\newpage
\refstepcounter{section}
\addcontentsline{toc}{section}{\protect\numberline{\thesection}Основы многопользовательских и многозадачных вычислений}
\begin{center}\textbf{\thesection\ ОСНОВЫ МНОГОПОЛЬЗОВАТЕЛЬСКИХ И МНОГОЗАДАЧНЫХ ВЫЧИСЛЕНИЙ}\end{center}
\vspace*{-1.5em}

\ManualSubsection{От пакетной обработки к интерактивным вычислениям}

В 1950-х годах господствующей моделью использования вычислительной техники являлась пакетная обработка, при которой задания, подготовленные на перфокартах, выполнялись последовательно, исключая любое интерактивное взаимодействие с пользователем. Этот подход, характеризуемый длительными простоями центрального процессора во время операций ввода-вывода, привёл к осознанию необходимости более эффективного использования дорогостоящих вычислительных ресурсов. Концептуальный прорыв произошёл в конце 1950-х годов, когда Джон Маккарти в своей внутренней памятной записке MIT сформулировал идею разделения времени (time-sharing) \cite{mccarthy1959programs}. Данная концепция предполагала возможность обслуживания нескольких пользователей одним компьютером, создавая у каждого иллюзию монопольного доступа к вычислительной мощности. Практическая реализация этих идей была осуществлена в 1961 году с появлением экспериментальной системы CTSS (Compatible Time-Sharing System) в MIT, которая считается первой работоспособной системой разделения времени \cite{computerhistory1961}.

\ManualSubsection{Концепция «компьютерной утилиты» и проект Multics}

Развитие идей разделения времени привело к формированию концепции «компьютерной утилиты» (computer utility), рассматривавшей вычислительные ресурсы как общедоступный сервис, аналогичный электричеству. Наиболее амбициозным проектом, направленным на воплощение этой концепции, стал Multics (Multiplexed Information and Computing Service), инициированный в 1965 году консорциумом в составе MIT, General Electric и Bell Labs \cite{multicians2025history}. Целью проекта было создание высоконадёжной, безопасной и масштабируемой операционной системы, способной непрерывно обслуживать сотни пользователей. Хотя коммерческий успех Multics был ограниченным, его архитектурные принципы, такие как иерархическая файловая система, кольцевая защита и динамическое связывание, оказали глубокое влияние на последующее развитие вычислительных систем, особенно на операционную систему Unix \cite{organick1972multics}.

\ManualSubsection{Эволюция технологий удалённого доступа}

Распространение систем разделения времени было бы невозможно без развития инфраструктуры удалённого доступа. Первоначально в качестве клиентских устройств использовались электромеханические телетайпы, такие как Teletype Model 33, которые печатали результаты на бумаге. Связь с центральным компьютером осуществлялась через последовательные интерфейсы или с использованием модемов, преобразующих цифровые сигналы для передачи по телефонным линиям. К 1963 году система CTSS уже поддерживала одновременную работу около тридцати пользователей, часть из которых подключалась удалённо \cite{corbato1962experimental}. Эта технология заложила основу для модели «терминал-сервер», которая эволюционировала от текстовых терминалов к графическим рабочим столам, оставаясь актуальной в современных корпоративных средах.

\ManualSubsection{Многозадачность и многопользовательность: терминологическое разграничение}

Важно различать понятия многозадачности (multitasking) и многопользовательности (multiuser capability), которые описывают различные аспекты работы системы. Многозадачность относится к способности операционной системы выполнять несколько процессов или потоков, переключая между ними ресурс процессора. Многопользовательская система, в свою очередь, обеспечивает одновременную работу нескольких независимых пользователей в изолированных средах, что является фундаментальным признаком систем разделения времени. Таким образом, любая многопользовательская система по определению является многозадачной, однако обратное утверждение неверно: современная настольная операционная система может эффективно управлять множеством фоновых задач, оставаясь при этом ориентированной на одного пользователя.

\ManualSubsection{Сервисные бюро и модель «вычисления как услуги»}

В 1960-х годах технология разделения времени породила новую бизнес-модель — сервисные бюро (service bureaus). Эти компании, приобретая мощные мейнфреймы, такие как IBM System/360, продавали вычислительное время клиентам, которые не могли себе позволить собственное оборудование. Клиенты, часто представлявшие средний бизнес, подключались к центральному компьютеру бюро через телефонные линии и модемы для выполнения задач, связанных с расчётом заработной платы, бухгалтерским учётом или научными исследованиями \cite{ibm2024timesharing}. Эта модель стала прообразом современной концепции «вычисления как услуги», демонстрируя экономическую эффективность централизации вычислительных ресурсов и их распределённого использования.

\ManualSubsection{Облачные технологии как наследники time-sharing}

Современные облачные вычисления являются прямым идейным наследником систем разделения времени, реализуя концепцию «компьютерной утилиты» на качественно новом технологическом уровне. Модель SaaS (Software as a Service) напрямую соответствует изначальному видению доступа к приложениям как к централизованно предоставляемому сервису \cite{cacm2023saas}. Ключевым архитектурным принципом, унаследованным от time-sharing и доведённым до совершенства в облаках, является мультитенантность (multi-tenancy), при которой один экземпляр приложения и его инфраструктура обслуживают множество изолированных клиентов. Это требует сложных механизмов обеспечения справедливого распределения ресурсов, безопасности и приватности данных, что является центральной темой в разработке облачных систем \cite{aws2025fairness}.

\ManualSubsection{Терминальные серверы и инфраструктура виртуальных рабочих столов}

Промежуточным звеном между классическими системами разделения времени и облачными сервисами выступают технологии терминальных серверов и виртуальных рабочих столов (VDI). Такие решения, как Microsoft Remote Desktop Services, позволяют множеству пользователей одновременно запускать сессии или полноценные виртуальные машины на централизованных серверах, получая к ним доступ с лёгких клиентских устройств через сеть \cite{microsoft2024rds}. Это обеспечивает централизованное управление, безопасность данных и снижение стоимости владения клиентским парком. Архитектура VDI, основанная на гипервизорах, предоставляет каждому пользователю полностью изолированную и персонализируемую рабочую среду, что является эволюцией идеи индивидуального пользовательского контекста в системах разделения времени \cite{azure2024vdi}. Оба подхода продолжают развивать ключевые принципы многопользовательских вычислений, адаптируя их к требованиям современной корпоративной ИТ-инфраструктуры.

\newpage
\refstepcounter{section}
\addcontentsline{toc}{section}{\protect\numberline{\thesection}Средства мониторинга производительности и оптимизации работы системы}
\begin{center}\textbf{\thesection\ СРЕДСТВА МОНИТОРИНГА ПРОИЗВОДИТЕЛЬНОСТИ И ОПТИМИЗАЦИИ РАБОТЫ СИСТЕМЫ}\end{center}
\vspace*{-1.5em}
\ManualSubsection{Введение в мониторинг и оптимизацию}

Мониторинг производительности является непрерывным процессом сбора, анализа и визуализации данных о функционировании системы, что служит основой для её оптимизации, то есть улучшения таких характеристик, как скорость работы, эффективность использования ресурсов и надёжность. Фундаментальная идея заключается в том, что эффективная оптимизация невозможна без предварительного измерения ключевых показателей системы.

\ManualSubsection{Ключевые метрики системного мониторинга}

Процесс мониторинга начинается с определения набора измеримых показателей, которые характеризуют состояние компонентов инфраструктуры. Для центрального процессора (CPU) критически важна метрика загрузки, отражающая степень использования вычислительных ядер; её длительное приближение к стам процентов указывает на исчерпание ресурсов. Мониторинг оперативной памяти (RAM) должен учитывать не только общий объём использованной памяти, но и доступную для немедленного выделения, поскольку современные операционные системы активно используют часть памяти для кэширования данных. Для дискового подсистемы (Disk I/O) ключевыми являются показатели пропускной способности, количества операций ввода-вывода в секунду (IOPS) и задержки выполнения операций, так как именно диск часто становится узким местом, замедляющим работу всей системы. Сетевые метрики, такие как объём передаваемых данных и частота потери пакетов, необходимы для оценки пропускной способности каналов связи между серверами.

\ManualSubsection{Метрики уровня приложения}

Помимо инфраструктурных метрик, обязательным является контроль показателей самого прикладного программного обеспечения. К ним относится время отклика на запрос пользователя, причём анализ должен учитывать не только среднее значение, но и высокие процентили (95-й, 99-й), которые выявляют редкие, но критически важные замедления. Пропускная способность (throughput), измеряемая в запросах в секунду, определяет общую производительность системы под нагрузкой. Частота ошибок, выраженная в процентах неудачных операций, служит важнейшим индикатором стабильности приложения \cite{prometheus_basics}.

\ManualSubsection{Основные системные средства мониторинга}

Начальный этап анализа состояния системы часто реализуется с помощью встроенных утилит командной строки Unix-подобных систем. Такие инструменты, как \texttt{top}, \texttt{htop} и \texttt{free}, предоставляют информацию о загрузке процессора и использовании памяти в реальном времени. Утилиты \texttt{iostat} и \texttt{vmstat} дают детальную картину активности дискового ввода-вывода и свопинга. Однако эти средства предоставляют лишь моментальный снимок состояния и не предназначены для долгосрочного хранения истории метрик или централизованного наблюдения за распределённой инфраструктурой.

\ManualSubsection{Профессиональные средства мониторинга}

Для построения полноценной системы наблюдения требуется использование специализированных платформ. Prometheus, построенная на модели активного опроса (pull), является стандартом де-факто для сбора и хранения метрик в виде временных рядов. Её мощный язык запросов PromQL позволяет выполнять сложный анализ данных и настраивать правила для генерации алертов \cite{prometheus_basics}. Для визуализации собранных данных широко применяется Grafana, которая позволяет создавать комплексные информационные панели (дашборды), объединяющие графики, диаграммы и индикаторы состояния из различных источников данных, включая Prometheus \cite{grafana_visualizations}.

Обработка логов, являющихся ключевым источником информации для отладки, эффективно осуществляется с помощью стека ELK (Elasticsearch, Logstash, Kibana). Logstash выполняет сбор и первичную обработку логов, Elasticsearch обеспечивает их индексирование и быстрый поиск, а Kibana предоставляет веб-интерфейс для анализа и построения визуализаций на основе лог-данных \cite{elk_stack_overview}. В архитектурах, основанных на микросервисах, для отслеживания пути единого запроса через множество сервисов применяется распределённый трейсинг. Инструмент Jaeger реализует эту задачу, представляя запрос в виде трейса (trace), состоящего из взаимосвязанных спэнов (span), каждый из которых соответствует операции в рамках отдельного сервиса \cite{jaeger_tracing}. Для хранения метрик с высокой частотой обновления используются специализированные базы данных временных рядов, такие как InfluxDB, оптимизированные для работы с потоками данных, помеченными временными метками \cite{influxdb_timeseries}.

\ManualSubsection{Облачные платформы и открытые стандарты}

Крупные организации часто отдают предпочтение всеобъемлющим облачным платформам мониторинга, таким как Datadog или New Relic, которые предлагают интеграцию сбора метрик, логов и трейсов в едином интерфейсе, минимизируя затраты на развёртывание и поддержку собственной инфраструктуры наблюдения. Для обеспечения совместимости и упрощения инструментирования приложений создан открытый стандарт OpenTelemetry, который предоставляет унифицированные API и SDK для сбора телеметрии (метрик, логов, трейсов) с возможностью последующей отправки в любую поддерживаемую систему мониторинга.

\ManualSubsection{Практический подход к внедрению}

Выбор конкретного набора инструментов определяется масштабом и сложностью системы. Для малых проектов и стартапов оптимальным решением является комбинация Prometheus и Grafana. По мере роста системы и усложнения архитектуры к ним добавляется стек ELK для анализа логов и Jaeger для трейсинга микросервисов. Для корпоративных распределённых систем часто целесообразен переход на управляемые облачные платформы. Ключевым принципом является итеративность: внедрение мониторинга следует начинать с базовых измерений, постепенно наращивая глубину и охват наблюдаемости по мере развития самой системы.

\newpage
\section*{}
\vspace*{-2.5em}
\begin{center}\textbf{ЗАКЛЮЧЕНИЕ}\end{center}
\phantomsection
\addcontentsline{toc}{section}{Заключение}

Операционная система является ключевым компонентом архитектуры информационных систем, обеспечивающим связь между пользователем, прикладным программным обеспечением и аппаратными ресурсами. Через механизмы управления процессами, памятью и файловыми системами ОС формирует среду, в которой возможна надёжная и эффективная работа приложений.

Рассмотренные в работе функции операционной системы, её взаимодействие с аппаратным обеспечением и архитектурные подходы к построению ядер (монолитная, микроядерная, гибридная архитектуры) показывают, что выбор конкретной ОС определяется требованиями к надёжности, быстродействию, масштабу и области применения. Классификация по назначению подчёркивает различия между системами общего назначения, ОС реального времени и встраиваемыми решениями.

Анализ эволюции многопользовательских и многозадачных вычислений демонстрирует, как идеи разделения времени привели к появлению современных моделей удалённого доступа, терминальных серверов, инфраструктуры виртуальных рабочих столов и облачных вычислений. Эти подходы основаны на эффективном совместном использовании ресурсов и строгой изоляции пользовательских сред.

Особое внимание уделено средствам мониторинга и оптимизации: от встроенных утилит операционных систем до специализированных платформ сбора метрик, логов и распределённого трейсинга. Именно они позволяют объективно оценивать состояние ресурсов, выявлять узкие места и принимать обоснованные решения по настройке и развитию инфраструктуры.

Таким образом, операционная система выступает центральным элементом в архитектуре современной вычислительной среды, определяющим уровень её надёжности, безопасности и производительности. Понимание принципов работы ОС, её классификации, механизмов взаимодействия с аппаратурой, а также практик мониторинга и оптимизации является необходимой основой для профессиональной деятельности системных администраторов и архитекторов информационных систем.

\newpage
\begin{center}\textbf{СПИСОК СОКРАЩЕНИЙ}\end{center}
\phantomsection
\addcontentsline{toc}{section}{Список сокращений}

\vspace{1em}

{
\small
\noindent
\begin{tabular}{|l|p{0.75\textwidth}|}
\hline
\textbf{Сокращение} & \multicolumn{1}{c|}{\textbf{Расшифровка}} \\
\noalign{\hrule height 1.2pt}
\hline
API & Application Programming Interface (интерфейс прикладного программирования)\\
\hline
BIOS & Basic Input/Output System (базовая система ввода-вывода)\\
\hline
CPU & Central Processing Unit (центральный процессор)\\
\hline
CTSS & Compatible Time-Sharing System (совместимая система разделения времени)\\
\hline
DMA & Direct Memory Access (прямой доступ к памяти)\\
\hline
FIFO & First-In-First-Out (первым пришёл — первым ушёл)\\
\hline
HDD & Hard Disk Drive (накопитель на жёстких магнитных дисках)\\
\hline
HAL & Hardware Abstraction Layer (слой аппаратных абстракций)\\
\hline
I/O & Input/Output (ввод-вывод)\\
\hline
IPC & Inter-Process Communication (межпроцессное взаимодействие)\\
\hline
OS & Operating System (операционная система)\\
\hline
IOPS & Input/Output Operations Per Second (операций ввода-вывода в секунду)\\
\hline
LRU & Least Recently Used (наименее недавно использованный)\\
\hline
MFT & Master File Table (главная файловая таблица)\\
\hline
MIT & Massachusetts Institute of Technology (Массачусетский технологический институт)\\
\hline
NTFS & New Technology File System (файловая система новой технологии)\\
\hline
POSIX & Portable Operating System Interface (переносимый интерфейс операционных систем)\\
\hline
RAM & Random Access Memory (оперативная память)\\
\hline
SaaS & Software as a Service (программное обеспечение как услуга)\\
\hline
SSD & Solid-State Drive (твердотельный накопитель)\\
\hline
UEFI & Unified Extensible Firmware Interface (унифицированный расширяемый интерфейс прошивки)\\
\hline
VDI & Virtual Desktop Infrastructure (инфраструктура виртуальных рабочих столов)\\
\hline
\end{tabular}
}

\clearpage
\nocite{*}
\printbibliography[heading=bibliography]

\end{document}
