\documentclass{SibFU-docs}

% доп. пакеты
\usepackage{graphicx}
\usepackage{float}
\usepackage{hyperref}
\usepackage{caption}
\usepackage{subcaption}
\usepackage{amsmath}
% предотвартим ошибку с \Bbbk перед загрузкой пакета amssymb
\let\Bbbk\relax
\usepackage{amssymb}
\usepackage{booktabs}
\usepackage{tabularx}
\usepackage{longtable}
\usepackage{enumitem}
\usepackage{multirow}

% гиперссылки
\hypersetup{
    colorlinks=true,
    linkcolor=black,
    citecolor=blue,
    urlcolor=blue
}

% библиография
\addbibresource{report.bib}

% титульный лист
\begin{document}
\setlist[itemize]{left=1.3cm, itemsep=1pt, topsep=2pt} % для выравнивания списков
\setlist[enumerate]{left=1.3cm, itemsep=1pt, topsep=2pt}
\renewcommand{\contentsname}{}
\mycovertitle
{Гуманитарный институт}
{Кафедра прикладной информатики в искусстве и интерактивном медиа}
{Дополненная и виртуальная реальность}
{ст. преподаватель И.~Р.~Нигматуллин}
{Пачковская А.В., Финочка С.А., Захаров И.М., Логинова Ю.В.,\\Иванова С.Ю., Тимофеева А.А., Дегтярева В.В., Шикайкова К.С.}
 
\begin{center}
\bfseries РЕФЕРАТ
\end{center}
\noindent

Реферат по теме: «Дополненная и виртуальная реальность» содержит 35 страниц и 23 использованных источников.

Ключевые слова: ДОПОЛНЕННАЯ РЕАЛЬНОСТЬ, ВИРТУАЛЬНАЯ РЕАЛЬНОСТЬ, СМЕШАННАЯ РЕАЛЬНОСТЬ, ИММЕРСИВНЫЕ ТЕХНОЛОГИИ, ОБЪЕКТЫ, ТРЕХМЕРНОЕ МОДЕЛИРОВАНИЕ, ФОТОГРАММЕТРИЯ, ОБЛАКО ТОЧЕК, ПОЛИГОНАЛЬНАЯ МОДЕЛЬ, ТЕКСТУРИРОВАНИЕ И РИГГИНГ, ПРИНЦИПЫ РАБОТЫ УСТРОЙСТВ, ИНТЕРФЕЙС, ДВИЖКИ РАЗРАБОТКИ, ГУМАНИТАРНЫЕ ИССЛЕДОВАНИЯ.

В работе исследуются принципы, технологии и приложения иммерсивных сред — дополненной (AR), виртуальной (VR) и смешанной реальности (MR). Рассмотрены методы создания и цифрового представления трехмерных объектов (3D-моделирование, фотограмметрия, полигональные модели, текстурирование), а также технические основы иммерсивных устройств (трекинг головы и рук, пространственное позиционирование 6DoF, распознавание маркеров, сенсоры). Особое внимание уделено инструментам разработки (Unity, Unreal Engine, ARCore, ARKit, WebXR) и проектированию пользовательского опыта (UX/UI) в AR/VR. Проанализированы практические применения технологий в образовании (виртуальные лаборатории, образовательные симуляции, мультимодальное обучение), культуре (виртуальные музеи, цифровые реконструкции) и социальной сфере (метавселенные, социальные VR-платформы), а также этические аспекты их влияния на восприятие и медиакультуру.

Основные выводы:

\begin{enumerate}
    \item Иммерсивные технологии (AR/VR/MR) создают принципиально новые интерфейсы взаимодействия человека с цифровой информацией, проецируя виртуальные объекты в реальное пространство или полностью погружая пользователя в синтетическую среду, что формирует расширенный пользовательский опыт (XR).
    
    \item Технической основой иммерсивных систем являются движки реального времени и AR-фреймворки (ARKit, ARCore), которые обеспечивают трекинг положения в пространстве (6DoF), распознавание маркеров и пространственное позиционирование цифрового контента через комплекс сенсоров и инерциальных датчиков**.
    
    \item Создание контента для иммерсивных сред опирается на pipeline 3D-моделирования (от фотограмметрии и облаков точек до полигональных моделей, текстурирования и риггинга), что позволяет добиваться высокого уровня реализма и интерактивности цифрового представления 3D-моделей.

    \item Применение иммерсивных технологий выходит за рамки развлечений, находя мощные реализации в образовательных симуляциях, виртуальных музеях, гуманитарных исследованиях и цифровых реконструкциях, способствуя развитию медиакультуры и мультимодального обучения, но одновременно требуя разработки этических рамок и изучения влияния VR/AR на восприятие.
\end{enumerate}

Рекомендации:

\begin{itemize}[label=-]
    \item Для образовательных и культурных проектов использовать виртуальные лаборатории и историческую визуализацию на базе движков разработки (Unity, Unreal), применяя геймификацию для повышения вовлеченности.
    \item Разрабатывать пользовательский интерфейс в AR/VR с учетом специфики UX-дизайна иммерсивных сред, обеспечивая интуитивное взаимодействие с виртуальными объектами.
    \item Внедрять кросс-платформенные решения с использованием WebXR и WebAR для широкой доступности контента без необходимости установки специализированного ПО.
    \item При создании социальных VR-платформ и проектов для метавселенных уделять первостепенное внимание вопросам этики иммерсивных технологий, приватности и психологического комфорта пользователей.
\end{itemize}

\newpage

\thispagestyle{empty}
\begin{center}
\bfseries СОДЕРЖАНИЕ
\end{center}
\vspace{-6pt}
\tableofcontents
\clearpage

% фантомные боли для секций, чтобы не было нумераций кроме тела реферата
\section*{}
\vspace*{-2.5em}
\begin{center}\textbf{ВВЕДЕНИЕ}\end{center}
\phantomsection
\addcontentsline{toc}{section}{Введение}

Иммерсивные технологии, включающие в себя дополненную (AR), виртуальную (VR) и смешанную реальность (MR), формируют новую парадигму взаимодействия человека с цифровой информацией, интегрируя виртуальные объекты в физическое пространство или создавая полностью синтетические среды. Их развитие и внедрение определяют современные тенденции в цифровой культуре, образовании, музейном деле и социальных коммуникациях, предлагая расширенный пользовательский опыт (XR). Создание и управление такими средами требуют комплексного подхода, объединяющего методы 3D-моделирования, пространственного позиционирования и проектирования пользовательских интерфейсов.

Современное состояние проблемы. Технологический ландшафт иммерсивных сред находится в стадии активного развития: от специализированных приложений к платформенным решениям. AR-фреймворки, такие как ARCore и ARKit, и мощные движки реального времени (Unity, Unreal Engine) становятся стандартом для разработки. При этом сохраняется актуальность фундаментальных задач: повышения реализма цифрового представления 3D-моделей (через фотограмметрию, текстурирование), совершенствования трекинга (6DoF, отслеживание головы и рук) и разработки интуитивного UX-дизайна. Область применения технологий стремительно расширяется от виртуальных музеев и образовательных симуляций до социальных VR-платформ и концепций метавселенных, что выдвигает на первый план вопросы этики и влияния на человеческое восприятие.

Актуальность работы обусловлена стремительной конвергенцией иммерсивных технологий в повседневную жизнь и профессиональные сферы. Возникает потребность не только в технической реализации, но и в осмысленном проектировании содержательного контента — от исторической визуализации и цифровых реконструкций до виртуальных лабораторий для мультимодального обучения. Развитие стандартов, таких как WebXR, делает технологии более доступными, усиливая необходимость в методических и культурологических исследованиях их применения.

Работа заключается в системном исследовании полного цикла создания иммерсивной среды: от технических основ (сенсоры, трекинг, движки) и контент-пайплайна (3D-моделирование, полигональные модели, облака точек) до принципов проектирования пользовательского опыта и анализа конкретных прикладных областей в гуманитарной сфере.

Цель работы: исследовать принципы, технологии и инструменты создания иммерсивных сред (AR/VR/MR), а также проанализировать методики и практики их эффективного применения в образовании, культуре и научной визуализации.

Задачи: определить ключевые компоненты и принципы работы иммерсивных устройств (VR-шлемов, AR-устройств); проанализировать методы создания и интеграции трехмерных объектов (3D-моделирование, фотограмметрия, текстурирование и риггинг); исследовать технологии пространственного взаимодействия (трекинг, распознавание маркеров, 6DoF); изучить современные движки и фреймворки для разработки (Unity, Unreal Engine, ARCore, ARKit, WebAR); рассмотреть особенности проектирования пользовательского интерфейса и UX в иммерсивных средах; проанализировать прикладное применение технологий в виртуальных музеях, образовательных симуляциях и гуманитарных исследованиях; оценить социокультурные аспекты и этические вопросы развития иммерсивных технологий.

Методы: анализ технологических решений и документации к платформам разработки (ARCore, ARKit, WebXR), сравнительный анализ методов создания 3D-контента, проектирование пользовательских сценариев взаимодействия, анализ кейсов применения в сфере культуры и образования, изучение академических источников по медиакультуре и цифровой этике.

\newpage
% вручную увеличить номер секции и записать в оглавление с номером,
% чтобы в теле не печатался автоматический номер, но в tableofcontents он остался
\refstepcounter{section}
\addcontentsline{toc}{section}{\protect\numberline{\thesection} Представление трехмерных объектов как формы цифровой информации}
\vspace*{-2.5em}
\begin{center}\textbf{\thesection\ ПРЕДСТАВЛЕНИЕ ТРЕХМЕРНЫХ ОБЪЕКТОВ КАК ФОРМЫ ЦИФРОВОЙ ИНФОРМАЦИИ}\end{center}
\vspace*{-1.5em}
\ManualSubsection{Понятие цифрового представления 3D-объектов}

Цифровое представление трёхмерных объектов стало критически важным для целого ряда областей: от инженерного проектирования (CAD/CAM) и компьютерной анимации до 3D-печати и виртуальной реальности. Эти модели служат основой для создания симуляций, виртуальных миров, прототипирования и научной визуализации. Фундаментально любая цифровая 3D-модель состоит из двух основных компонентов: геометрии, определяющей форму объекта в пространстве, и атрибутов, описывающих визуальные свойства его поверхности, таких как материалы и текстуры \cite{shirley2009}. Это разделение позволяет эффективно управлять сложностью сцены в процессе рендеринга.

\ManualSubsection{Методы представления геометрии}

Существует несколько базовых методов цифрового описания геометрии, каждый из которых оптимизирован для конкретных задач. Доминирующим подходом в интерактивных приложениях, играх и анимации являются полигональные сетки. Данный метод основан на аппроксимации поверхности объекта множеством плоских многоугольников, чаще всего треугольников. Цифровое представление полигональной модели может храниться в виде списка вершин, списка граней или в более сложных структурах данных, таких как «крылатое» представление, обеспечивающее эффективный обход поверхности. Основное преимущество полигональных сеток заключается в их гибкости, возможности управления уровнем детализации и высокой совместимости с графическим конвейером реального времени.

Для задач, требующих математической точности и параметрического контроля, например, в промышленном дизайне и машиностроении, применяются параметрические поверхности. Наиболее совершенной их формой являются NURBS (Non-Uniform Rational B-Splines). В цифровом виде такая поверхность описывается набором контрольных точек, их весов и узловых векторов. Это представление обеспечивает высокую точность и гладкость, однако может быть менее эффективным для рендеринга сложных органических форм по сравнению с полигональными сетками.

В научной и медицинской визуализации стандартом де-факто стало воксельное представление, аналогичное пиксельному, но в трёхмерном пространстве. Пространство разбивается на регулярную сетку из кубических элементов (вокселей), каждый из которых содержит значение, например, плотность или цвет. Формат DICOM, используемый для данных компьютерной томографии, по сути, является воксельным набором \cite{DICOM2023}. Хотя такое представление идеально для работы с объёмными данными сканирования, оно требует значительных вычислительных ресурсов и памяти.

Ещё одним методом, непосредственно получаемым в результате 3D-сканирования, является облако точек. Оно представляет собой неструктурированный набор точек в пространстве, каждая из которых имеет координаты и, возможно, дополнительные атрибуты, такие как цвет. Облака точек точно фиксируют геометрию реального объекта, но для их дальнейшего использования в графических приложениях часто требуется конвертация в полигональную сетку.

\ManualSubsection{Представление визуальных атрибутов}

Для придания 3D-геометрии реалистичного внешнего вида используются визуальные атрибуты, главным из которых является материал. Современный стандарт в индустрии — физически корректный рендеринг (Physically Based Rendering, PBR). Данная модель, популяризированная работами \cite{burley2012}, описывает поверхность через набор физически интерпретируемых параметров, таких как базовый цвет (albedo), шероховатость (roughness) и металличность (metallic). Это обеспечивает единообразный и предсказуемый визуальный результат при различном освещении. PBR-моделирование базируется на законах оптики и энергосбережения, что позволяет материалам взаимодействовать со светом в виртуальной среде аналогично их поведению в реальном мире, устраняя необходимость в субъективной ручной настройке отражений и бликов.

Детализация поверхности достигается за счёт наложения текстурных карт — растровых изображений, проецируемых на геометрию. Ключевой процесс, позволяющий выполнить это корректно, — UV-развёртка. В её ходе трёхмерная поверхность модели «разрезается» и проецируется на двумерную плоскость с координатами U и V, создавая карту для размещения текстур. Это позволяет избежать искажений и эффективно использовать текстурное пространство. Для создания иллюзии сложного рельефа без фактического изменения геометрии применяются карты нормалей (normal maps), которые манипулируют векторами поверхности при расчёте освещения. Более ресурсоёмкие, но и более точные карты смещения (displacement maps) физически деформируют сетку модели, создавая реальный геометрический рельеф, что особенно важно для кинематографического рендеринга и крупных планов.

Кроме того, современный пайплайн включает создание дополнительных карт, таких как карта окружающей окклюзии (ambient occlusion), определяющая затенение в углублениях, и карта высот (height map). Интеграция этих атрибутов в единый материал осуществляется внутри графических движков (Unreal Engine, Unity) или специализированных пакетов для создания материалов (Substance Designer), где задаются сложные взаимосвязи между различными текстурами и параметрами шейдера.

\ManualSubsection{Форматы файлов и процесс создания}

Выбор формата файла для хранения 3D-модели зависит от её назначения. Для 3D-печати широко используется формат STL, описывающий геометрию как неструктурированную треугольную сетку. В компьютерной графике распространён открытый формат OBJ, поддерживающий геометрию, текстуры и материалы. Для точного обмена инженерными данными между CAD-системами применяется стандартизированный формат STEP (ISO 10303), способный хранить параметрическую геометрию, сборки и допуски. В последние годы набирает популярность формат glTF (GL Transmission Format), позиционируемый как «JPEG для 3D». Его ключевое преимущество заключается в оптимальной структуре для передачи и быстрой загрузки 3D-сцен в веб-среду и приложения реального времени, так как он может содержать в одном файле всю необходимую информацию: меши, материалы, текстуры, анимации и даже сцены.

Процесс создания цифровых 3D-объектов может быть ручным, автоматическим или алгоритмическим. Ручное моделирование выполняется художниками в специализированном ПО, таком как Blender или Maya, для создания произвольных форм, и включает в себя техники полигонального моделирования, скульптинга и ретопологии для оптимизации сетки. Автоматическое оцифровывание реальных объектов осуществляется с помощью 3D-сканеров, результатом работы которых являются облака точек, или методами фотограмметрии, где модель реконструируется из серии фотографий с разных ракурсов. Полученные таким образом данные требуют последующей обработки: очистки от шума, сшивания сканов и создания чистой полигональной топологии. Алгоритмическая, или процедурная, генерация применяется для создания сложных структур, таких как ландшафты, архитектурные формы или растительность, по заданным правилам и параметрам. Этот подход, реализуемый в инструментах вроде Houdini или встроенных генераторах ландшафта игровых движков, позволяет создавать масштабные и вариативные объекты с минимальными ручными усилиями, что незаменимо для построения больших виртуальных миров.

\newpage
\refstepcounter{section}
\addcontentsline{toc}{section}{\protect\numberline{\thesection}Основы трехмерного моделирования и фотограмметрии для создания виртуальных объектов}
\begin{center}\textbf{\thesection\ ОСНОВЫ ТРЕХМЕРНОГО МОДЕЛИРОВАНИЯ И ФОТОГРАММЕТРИИ ДЛЯ СОЗДАНИЯ ВИРТУАЛЬНЫХ ОБЪЕКТОВ}\end{center}
\vspace*{-1.5em}
\ManualSubsection{Трехмерное моделирование}

Трёхмерное моделирование представляет собой процесс создания цифровых объектов, существующих в виртуальном пространстве с координатами по ширине, высоте и глубине. Этот процесс является фундаментальным для генерации контента в иммерсивных средах, таких как виртуальная и дополненная реальность. Существует несколько методологий моделирования, каждая из которых служит для решения определённого круга задач \cite{blender2024}.

Полигональное моделирование является наиболее распространённым подходом в реальном времени. Его принцип основан на построении поверхности объекта из плоских многоугольников — чаще всего треугольников или четырёхугольников. Совокупность таких полигонов, соединённых рёбрами и вершинами, формирует сетку, которая и определяет форму объекта. Чем плотнее сетка, тем более детализированной и гладкой выглядит модель, однако это напрямую влияет на вычислительную нагрузку при рендеринге. Оптимизация сетки, или ретопология, направлена на сокращение числа полигонов при сохранении визуального качества, что критически важно для приложений, работающих в режиме реального времени.

Параметрическое моделирование, напротив, ориентировано на инженерные и архитектурные задачи. В его основе лежит создание объектов не путём прямого манипулирования геометрией, а через задание управляющих параметров, зависимостей и ограничений. Изменение любого параметра автоматически обновляет всю модель, сохраняя логические связи между её элементами. Это обеспечивает высокую точность и удобство для итеративного проектирования, позволяя быстро генерировать вариации объекта. Наконец, 3D-скульптинг имитирует процесс лепки из цифрового материала. Начиная с базовой формы, художник с помощью специализированных кистей и инструментов вытягивает, сглаживает и детализирует поверхность, что позволяет создавать сложные органические формы с высоким уровнем художественной выразительности, что широко используется в кинематографе и игровой индустрии. Данный метод особенно эффективен для создания персонажей, существ и объектов с нерегулярной, сложной поверхностью.

\ManualSubsection{Фотограмметрия}

Фотограмметрия — это технология получения точных метрических данных и трёхмерных моделей реальных объектов на основе серии их фотографических изображений, сделанных с разных ракурсов \cite{remondino2011}. В отличие от абстрактного моделирования, фотограмметрия позволяет создать цифровую копию физического объекта с высокой степенью достоверности, сохраняя его точные пропорции, геометрию и текстуру поверхности. Эта технология находит применение в самых разных областях: от цифрового сохранения культурного наследия и создания контента для виртуальных музеев до инженерных обмеров и разработки реалистичных активов для компьютерных игр и кинопроизводства.

Ключевым принципом работы является метод «Structure-from-Motion» (SfM). Специализированное программное обеспечение, такое как Agisoft Metashape или Meshroom, анализирует перекрывающиеся фотографии, автоматически определяя положение камеры в момент каждого снимка и идентифицируя общие точки на объекте \cite{westoby2012, meshroom2024}. На основе этих данных алгоритмы реконструируют плотное облако точек в трёхмерном пространстве, которое затем преобразуется в полигональную сетку. Исходные фотографии используются для генерации текстур, которые «натягиваются» на полученную модель, обеспечивая её фотографическую реалистичность. Качество итоговой модели напрямую зависит от количества и качества исходных снимков, а также от правильности настроек алгоритмов обработки.

Технологический процесс требует соблюдения ряда условий для достижения качественного результата. Объект должен быть равномерно освещён для минимизации шумов и засветов, а фон по возможности — контрастным и не детализированным. Фотографии делаются с максимальным перекрытием (обычно 60–80 процентов), чтобы обеспечить достаточное количество общих точек для сопоставления. В качестве оборудования может использоваться как профессиональная зеркальная камера, так и современный смартфон, однако стабильность и качество оптики напрямую влияют на детализацию итоговой модели \cite{agisoft2024}. В зависимости от масштаба задачи применяются различные методы: наземная съёмка со штатива для архитектурных объектов и археологических артефактов, аэрофотограмметрия с БПЛА для ландшафтов, картографии и крупных сооружений, либо студийная съёмка на калиброванном поворотном столе для небольших предметов, где требуется максимальная точность.

\ManualSubsection{Виртуальные объекты и процесс их создания}

Виртуальный объект является фундаментальным элементом цифровых иммерсивных сред. Это цифровая конструкция, обладающая геометрией, текстурой и, зачастую, интерактивными свойствами, которая может быть как абстрактной формой, так и точной репликой физического предмета \cite{remondino2011}. Процесс создания такого объекта на стыке фотограмметрии и 3D-моделирования представляет собой последовательный технологический пайплайн, объединяющий сильные стороны обоих подходов для достижения баланса между реализмом и производительностью.

На первом, подготовительном, этапе производится всесторонняя фотосъёмка реального объекта в контролируемых условиях. Далее изображения импортируются в программное обеспечение для фотограмметрии, где происходит генерация высокополигональной модели и текстур. Поскольку такая модель, часто содержащая миллионы полигонов, непригодна для использования в реальном времени, следующим критическим шагом является её оптимизация. Выполняется ретопология — процесс создания новой, чистой и лёгкой полигональной сетки с правильной топологией поверх высокодетализированной сканированной модели. Для переноса визуальной сложности исходной сканированной поверхности на оптимизированную сетку используется процедура запекания карт нормалей, амбьента и диффузной текстуры. Этот процесс позволяет низкополигональной модели визуально имитировать детализацию высокополигонального оригинала. Финальным этапом является экспорт модели в стандартные форматы (например, FBX или glTF), готовые к импорту в игровые движки или среды визуализации. Такой синтез технологий позволяет эффективно переносить объекты материального мира в цифровое пространство для их последующего использования в образовательных симуляциях, цифровых архивах, виртуальных выставках и интерактивных медиапроектах, обеспечивая высокий уровень аутентичности и погружения.

\newpage
\refstepcounter{section}
\addcontentsline{toc}{section}{\protect\numberline{\thesection}Принципы функционирования технологий дополненной и виртуальной реальности}
\begin{center}\textbf{\thesection\ ПРИНЦИПЫ ФУНКЦИОНИРОВАНИЯ ТЕХНОЛОГИЙ ДОПОЛНЕННОЙ И ВИРТУАЛЬНОЙ РЕАЛЬНОСТИ}\end{center}
\vspace*{-1.5em}

\ManualSubsection{Основные принципы и концептуальные различия технологий}

Технологии виртуальной (VR) и дополненной реальности (AR) представляют собой смежные, но концептуально различные подходы к модификации человеческого восприятия посредством цифровых средств. Виртуальная реальность характеризуется полным замещением физического окружения пользователя сгенерированной компьютером синтетической средой, что обеспечивает эффект тотального погружения и изоляции от внешних стимулов \cite{jsdrm2018}. В противоположность этому, дополненная реальность функционирует на принципе суперпозиции, налагая цифровые информационные слои и объекты на непрерывно воспринимаемое реальное пространство, тем самым обогащая его без разрыва контекстуальной связи пользователя с физическим миром \cite{funreality2023}. Это фундаментальное различие формирует диссонирующие требования к архитектуре систем, методам пространственного отслеживания и психофизиологическому воздействию на пользователя.

\ManualSubsection{Архитектура и технические основы VR-систем}

Архитектурный каркас систем виртуальной реальности конструируется вокруг задачи генерации и поддержания когерентной иммерсивной среды с минимальной латентностью. Ключевым компонентом является система трекинга, реализующая отслеживание шести степеней свободы (6DoF). Она интегрирует данные инерциальных измерительных блоков (IMU), включающих гироскопы и акселерометры, для определения угловой скорости и линейного ускорения головы пользователя. Для повышения точности позиционирования в ограниченном пространстве (room-scale) применяются системы внешнего трекинга на основе оптических камер, инфракрасных маяков или технологии внутрипросветного сканирования (Lighthouse) \cite{sber2025}. Полученные пространственные координаты используются графическим конвейером, который осуществляет рендеринг двух перспективно смещённых стереоизображений с частотой кадров, превышающей порог в 90 Гц. Такая частота является критической для минимизации явления симуляционной болезни (киберболезни), возникающей при рассинхронизации зрительного и вестибулярного восприятия. Дополнительный вклад в иммерсивность вносит система пространственного аудио, закрепляющая виртуальные источники звука в трёхмерной системе координат, что усиливает иллюзию присутствия.

\ManualSubsection{Архитектура и технические основы AR-систем}

Архитектура систем дополненной реальности существенно сложнее ввиду необходимости одновременного решения задач компьютерного зрения, пространственного картирования и композитинга в реальном времени. Фундаментальной основой является алгоритм одновременной локализации и картирования (SLAM – Simultaneous Localization and Mapping). Алгоритм в реальном времени анализирует видеопоток с камеры устройства, выделяя характерные особенности окружения, детектируя плоскости (пол, стены) и оценивая глубину сцены. На основе этих данных строится и постоянно уточняется трёхмерная карта пространства, которая служит основой для привязки и устойчивого позиционирования виртуальных объектов \cite{funreality2023}. После определения позиции камеры и структуры окружения графический движок производит рендеринг виртуальных объектов с учётом перспективы, освещения реальной сцены и окклюзии (взаимного перекрытия реальными и цифровыми объектами). Полученное изображение композитируется с видеопотоком, что требует значительных вычислительных ресурсов для обеспечения фотореалистичности и стабильности. Современные мобильные AR-платформы, такие как ARKit и ARCore, эффективно оптимизируют этот процесс, используя специализированные нейронные процессоры и аппаратные ускорители, что позволяет развертывать сложные AR-приложения на смартфонах и планшетах.

\ManualSubsection{Эволюционный путь и исторический контекст технологий}

Генезис технологий иммерсивного взаимодействия уходит корнями в научные изыскания XIX века, когда Чарльз Уитстон в 1838 году сконструировал стереоскоп, продемонстрировав базовый принцип бинокулярного зрения. Однако терминологическое и концептуальное оформление произошло значительно позднее. Термин «виртуальная реальность» был введён в культурный оборот французским драматургом Антоненом Арто в 1938 году. Практическая реализация началась с создания Мортоном Хейлигом мультисенсорного симулятора «Сенсорама» (1957), а знаковым прорывом стал разработанный Айвеном Сазерлендом в 1968 году головной дисплей «Дамоклов меч», заложивший основы современных VR-шлемов \cite{jsdrm2018}. Понятие «дополненная реальность» было формализовано в 1990 году исследователями компании Boeing при разработке систем для помощи сборщикам авиационных конструкций. Современный этап, характеризующийся коммерциализацией и массовым распространением, начался в 2012 году с успешной краудфандинговой кампании Oculus Rift, что стимулировало возрождение интереса к VR. Параллельно, выход в 2017 году программных комплексов ARKit и ARCore создал унифицированную технологическую базу для разработки мобильных AR-приложений, открыв эпоху повсеместного проникновения дополненной реальности в потребительский сектор и бизнес-процессы.

\ManualSubsection{Анализ практических приложений в профессиональных и образовательных сферах}

Области применения технологий VR и AR эволюционировали от узкоспециализированных решений к массовым инструментам, трансформирующим ключевые секторы экономики и социальной сферы. В медицинской отрасли VR используется для создания высокоточных хирургических симуляторов, позволяющих проводить репетиции сложных оперативных вмешательств на анатомически корректных цифровых моделях без риска для пациента. Кроме того, VR-терапия успешно применяется для лечения тревожно-фобических расстройств методом контролируемой экспозиции. AR, в свою очередь, выполняет функцию интраоперационной навигации, проецируя данные КТ или МРТ непосредственно на операционное поле, что минимизирует инвазивность и повышает точность манипуляций. В образовательном процессе данные технологии реализуют принципы визуализации и интерактивности. VR позволяет осуществлять виртуальные экспедиции в недоступные среды (космос, исторические эпохи) и безопасно моделировать опасные лабораторные эксперименты. AR обогащает традиционные учебные материалы, преобразуя статические изображения в интерактивные трёхмерные модели, доступные для исследования и манипулирования \cite{jsdrm2018}. В промышленном производстве AR-решения интегрированы в процессы сборки, технического обслуживания и контроля качества, предоставляя работникам контекстные инструкции и доступ к удалённой экспертной поддержке в реальном времени. VR применяется для проектирования производственных линий, создания цифровых двойников и обучения персонала работе со сложным оборудованием в безопасной виртуальной среде \cite{sber2025}.

\ManualSubsection{Критический анализ существующих проблем и перспективных направлений развития}

Несмотря на значительные достижения, дальнейшее развитие технологий VR и AR сдерживается рядом системных ограничений. Для VR доминирующей проблемой остаётся феномен симуляционной болезни, обусловленный сенсорным конфликтом и латентностью системы. Борьба с ней требует дальнейшего увеличения частоты обновления дисплеев, внедрения техник рендеринга с фовеальной фиксацией (foveated rendering) и совершенствования алгоритмов прогнозирования движения (predictive tracking). Технологические барьеры AR связаны с точностью и устойчивостью работы алгоритмов SLAM в динамически меняющихся условиях, а также с энергопотреблением и тепловыделением мобильных устройств. Актуальной задачей является достижение фотореалистичности и физической достоверности цифровых объектов, интегрированных в реальную среду, что требует развития методов реалистичного освещения, затенения и окклюзии.

Перспективные векторы развития концентрируются вокруг конвергенции технологий в рамках расширенной реальности (XR) и их интеграции с другими прорывными направлениями. Ожидается переход к устройствам смешанной реальности в форм-факторе повседневных очков с применением технологий волноводных и голографических дисплеев. Важным трендом является развитие нейроинтерфейсов, позволяющих осуществлять взаимодействие со средой на основе регистрации биоэлектрической активности мозга. Формирование концепции метавселенных актуализирует исследования в области создания устойчивых, взаимосвязанных виртуальных миров, что влечёт за собой необходимость разработки новых экономических моделей, социальных протоколов и правовых норм. Разрешение данных вопросов станет определяющим фактором для устойчивого и социально ответственного внедрения технологий дополненной и виртуальной реальности в повседневную жизнь и профессиональную деятельность.

\newpage
\refstepcounter{section}
\addcontentsline{toc}{section}{\protect\numberline{\thesection}Среды и инструменты для разработки контента в AR/VR}
\begin{center}\textbf{\thesection\ СРЕДЫ И ИНСТРУМЕНТЫ ДЛЯ РАЗРАБОТКИ КОНТЕНТА В AR/VR}\end{center}
\vspace*{-1.5em}
\ManualSubsection{Игровые движки: Unity и Unreal Engine}

Выбор движка для создания иммерсивного контента является стратегическим решением, определяющим производительность, качество визуализации и целевую платформу. В индустрии доминируют два решения: Unity от Unity Technologies и Unreal Engine от Epic Games, каждое из которых предлагает уникальный подход к разработке. Unity позиционируется как универсальная и доступная среда с компонентной архитектурой Entity-Component-System (ECS) и программированием, что обеспечивает пологую кривую обучения и быструю итерацию прототипов. Движок славится своей кроссплатформенностью, поддерживая развёртывание на более чем 25 платформах, включая мобильные VR-устройства, что делает его предпочтительным выбором для проектов, ориентированных на широкую аудиторию и быстрый вывод на рынок \cite{Unity2025Engine}.

В противоположность этому, Unreal Engine зарекомендовал себя как промышленный стандарт для создания высококачественного фотореалистичного контента. Его архитектура, основанная на объектно-ориентированном программировании на C++ и визуальном скриптинге Blueprints, предоставляет разработчикам максимальный контроль над производительностью и графическим конвейером. Такие технологии, как система глобального освещения Lumen и управление геометрией Nanite, позволяют достигать кинематографического качества в реальном времени, что критически важно для иммерсивных симуляций и AAA-проектов в VR \cite{UnrealEngine2025Engine}. Таким образом, выбор между Unity и Unreal Engine представляет собой компромисс между скоростью разработки и доступностью, с одной стороны, и максимальной графической точностью и производительностью — с другой.

\ManualSubsection{Платформы и фреймворки дополненной реальности}

Технологическая основа дополненной реальности строится на сложных алгоритмах компьютерного зрения, главным из которых является одновременная локализация и картографирование (SLAM). Эти алгоритмы позволяют устройствам понимать и отслеживать своё положение в трёхмерном пространстве без использования специальных маркеров. Современные фреймворки предоставляют разработчикам абстракции над этими низкоуровневыми процессами, предлагая готовые функции для обнаружения плоскостей, оценки освещения и пространственного позиционирования контента. Подходы делятся на маркерные, требующие заранее подготовленного визуального паттерна, и безмаркерные, работающие в произвольной среде, причём последние становятся отраслевым стандартом благодаря своей универсальности.

Ключевыми платформами в этой области являются ARKit от Apple для iOS-устройств и ARCore от Google для Android и iOS. Обе реализуют визуально-инерциальный SLAM, обеспечивая высокоточное отслеживание и устойчивое закрепление цифровых объектов в реальном мире \cite{ARCore2025GoogleAR}. Для кросс-платформенной разработки в среде Unity существует абстрактный слой AR Foundation, унифицирующий работу с ARKit и ARCore. В корпоративном сегменте широко применяется Vuforia Engine от PTC, который предлагает гибридный подход, сочетающий превосходное отслеживание изображений и объектов с возможностями SLAM, что делает его популярным для промышленных и маркетинговых решений \cite{Vuforia2025PTC}. Выбор фреймворка определяется целевыми платформами, требованиями к точности и спецификой проекта — от массовых мобильных приложений до специализированных индустриальных кейсов.

\ManualSubsection{Конвейер создания и оптимизации 3D-контента}

Создание трёхмерных активов для иммерсивных сред представляет собой многоэтапный процесс, начинающийся с моделирования и заканчивающийся интеграцией в игровой движок. На первом этапе, 3D-моделировании, применяются такие инструменты, как Blender, Maya или ZBrush. Blender, являясь свободным и открытым пакетом, предоставляет полный спектр функций для моделирования, скульптинга и анимации, что делает его крайне популярным среди независимых разработчиков и образовательных проектов \cite{Blender2025ThreeDCreation}. Критически важным аспектом является оптимизация геометрии, которая включает создание низкополигональных (low-poly) моделей для реального времени и использование карт нормалей (normal maps) для передачи деталей с высокополигональных (high-poly) оригиналов без затрат производительности.

Последующие этапы — текстурирование и настройка материалов — определяют визуальное восприятие объекта. Современный стандарт, физически корректный рендеринг (PBR), реализуемый в таких программах, как Substance Painter, использует набор текстурных карт (альбедо, шероховатость, металличность), чтобы материалы реалистично взаимодействовали с виртуальным или дополненным освещением. Завершающей фазой конвейера является риггинг, анимация и финальная оптимизация перед экспортом в формат FBX или glTF. Итоговая интеграция в движок требует дополнительной настройки: наложения коллайдеров для физического взаимодействия, настройки шейдеров под конкретный рендер-пайплайн и импорта анимаций. Этот сложный пайплайн обеспечивает баланс между художественной выразительностью и техническими требованиями к частоте кадров в VR.

\ManualSubsection{Методы взаимодействия и ввода в иммерсивных средах}

Способы взаимодействия пользователя с виртуальным миром эволюционировали от классических контроллеров к более естественным методам, таким как отслеживание рук и пальцев. Физические контроллеры, оснащённые трекингом с шестью степенями свободы (6DoF), тактильной отдачей и набором кнопок, остаются золотым стандартом для игровых и профессиональных VR-приложений благодаря своей надёжности, точности и тактильному отклику. Они обеспечивают чёткий и предсказуемый ввод, что необходимо для сложных манипуляций и длительных сессий. Однако их недостатком является снижение уровня иммерсивности, поскольку пользователь осознаёт, что держит в руках искусственный инструмент, а не взаимодействует с миром напрямую.

Альтернативой выступает отслеживание рук (hand tracking), при котором камеры на гарнитуре в реальном времени вычисляют положение суставов кисти. Этот метод, реализованный в устройствах вроде Meta Quest и Apple Vision Pro, обеспечивает беспрецедентную естественность и интуитивность, особенно для новичков и социальных приложений. Однако он предъявляет высокие требования к вычислительным ресурсам, может страдать от ошибок при окклюзии (когда одна рука закрывает другую) и не предоставляет физической тактильной обратной связи. Продвинутым развитием этой технологии является отслеживание пальцев, часто реализуемое с помощью haptic-перчаток, которое открывает возможности для симуляции тонких моторных навыков, таких как игра на музыкальном инструменте или хирургические манипуляции, но остаётся нишевым и дорогостоящим решением.

\ManualSubsection{Критерии производительности и комфорта пользователя}

Успех иммерсивного опыта в решающей степени зависит от технических параметров, непосредственно влияющих на физиологический комфорт пользователя. Ключевым из них является частота кадров (FPS), где минимально допустимым для VR считается показатель в 90 кадров в секунду, а комфортный порог начинается со 120 FPS. Низкая частота кадров приводит к задержке между движением головы и визуальным откликом, что является основной причиной киберболезни (cybersickness) — состояния, характеризующегося тошнотой и дезориентацией. Для достижения стабильно высокого FPS применяется комплекс методов оптимизации, включая уровни детализации (LOD), отсечение невидимых объектов (occlusion culling) и батчинг вызовов отрисовки (draw call batching).

Не менее важным компонентом иммерсии является пространственный звук, который создаёт акустическую трёхмерную картину, соответствующую визуальной сцене. Технология, основанная на Head-Related Transfer Function (HRTF), позволяет точно позиционировать звуковые источники вокруг пользователя, значительно усиливая чувство присутствия. Третьим критическим фактором выступает общая латентность системы «движение–фотон» (motion-to-photon latency), которая должна составлять менее 20 миллисекунд. Эта задержка складывается из времени обработки данных сенсоров, рендеринга кадра и отклика дисплея. Превышение порогового значения латентности приводит к рассогласованию между вестибулярным и зрительным восприятием, что также провоцирует укачивание. Таким образом, создание комфортного VR-опыта требует глубокого понимания и тонкой балансировки всех этих взаимосвязанных технических и психофизиологических аспектов.

\newpage
\refstepcounter{section}
\addcontentsline{toc}{section}{\protect\numberline{\thesection}Применение AR и VR в гуманитарных исследованиях и образовании}
\begin{center}\textbf{\thesection\ ПРИМЕНЕНИЕ AR И VR В ГУМАНИТАРНЫХ ИССЛЕДОВАНИЯХ И ОБРАЗОВАНИИ}\end{center}
\vspace*{-1.5em}

\ManualSubsection{Введение и педагогические основания}

Технологии виртуальной (VR) и дополненной реальности (AR) открывают принципиально новые горизонты в гуманитарных науках и образовании, трансформируя абстрактные концепции и исторические нарративы в интерактивные, погружающие среды. Их применение позволяет преодолеть традиционный барьер между теоретическим знанием и эмпирическим опытом, формируя у обучающегося психологическое состояние «присутствия» («sense of presence») внутри изучаемого контекста. Это создает основу для глубокого кинестетического и эмоционального усвоения материала, что особенно значимо для дисциплин гуманитарного цикла. Современная педагогическая наука все чаще обращается к иммерсивным технологиям как к инструментам, способным обеспечить мультимодальное обучение, сочетающее визуальный, аудиальный и кинестетический каналы восприятия.

\ManualSubsection{Применение в исторических и культурологических исследованиях}

Историческая наука и культурология переживают методологическую трансформацию, обусловленную возможностями виртуальных и дополненных сред. Технологии позволяют не только визуализировать, но и интерактивно исследовать утраченные или труднодоступные объекты культурного наследия. В частности, практика создания виртуальных исторических реконструкций, например, разрушенных памятников архитектуры, предоставляет исследователям уникальный инструмент для анализа пространственных отношений, масштаба и архитектурных особенностей, переводя работу с двумерными чертежами и описаниями в трехмерное иммерсивное пространство \cite{VRArchaeology2024}. Этот процесс носит междисциплинарный характер, объединяя экспертизу историков, археологов и IT-специалистов на всех этапах — от сбора и анализа источников до трехмерного моделирования и текстурирования.

Параллельно развивается практика создания виртуальных музеев, которые обеспечивают доступ к коллекциям и экспозициям, преодолевая географические и физические ограничения. Подобные проекты, как отмечается в обзорах, позволяют детально изучать произведения искусства, манипулируя виртуальными артефактами, что невозможно в условиях традиционного музейного посещения \cite{VirtualMuseumsPedsovet}. Это направление расширяет границы медиакультуры, создавая новые формы цифрового взаимодействия с культурным наследием.

\ManualSubsection{Применение в лингвистике и межкультурной коммуникации}

В области лингводидактики и изучения межкультурной коммуникации иммерсивные технологии предлагают решение ключевой проблемы — создания аутентичной языковой среды. AR-приложения, такие как интерактивные карточки, совмещают визуальные образы, текстовые обозначения и звуковое сопровождение, что способствует лучшему запоминанию лексики через мультисенсорное восприятие \cite{AReducationVarwin}. Более сложные VR-решения моделируют полноценные коммуникативные ситуации, помещая пользователя в виртуальные сценарии реального мира, где необходимо использовать язык для решения практических задач. Это формирует не только языковые навыки, но и культурную компетенцию, позволяя отрабатывать невербальные аспекты общения и социальные нормы в контролируемой, но реалистичной среде.

\ManualSubsection{Применение в социальных науках и развитие эмпатии}

Одним из наиболее социально значимых направлений является использование VR для моделирования сложных социальных ситуаций и развития эмпатии. Технология позволяет пользователю «воплотиться» в опыт другого человека, смоделировав, например, перспективу человека с ограниченными возможностями или представителя иной социальной группы. Такой иммерсивный опыт, основанный на симуляции жизненных обстоятельств, приводит к активации как когнитивного компонента эмпатии (понимание чужой перспективы), так и эмоционального (сопереживание). Исследования показывают, что подобные симуляции могут оказывать долгосрочное влияние на установки и способствовать снижению предубеждений, открывая новые инструменты для социального образования и формирования толерантности.

\ManualSubsection{Интеграция в образовательный процесс и индивидуализация}

Внедрение VR и AR в образовательный процесс кардинально меняет парадигму взаимодействия «преподаватель — студент», смещая акцент в сторону активного, исследовательского обучения. Иммерсивные среды позволяют учащимся «оказываться» внутри изучаемых процессов — будь то историческое событие, архитектурный памятник или биологическая система, что резко повышает уровень вовлеченности и эмоциональной значимости материала. Нейропедагогические данные подтверждают, что подобный опыт способствует формированию более прочных нейронных связей и улучшению долговременной памяти за счет одновременной активации множественных сенсорных каналов.

Ключевым преимуществом технологий является возможность персонализации обучения. Адаптивные системы, построенные на движках реального времени и алгоритмах анализа данных, могут отслеживать прогресс учащегося, выявлять паттерны ошибок и автоматически подстраивать сложность контента, темп и способы подачи информации. Это позволяет выстраивать индивидуальные образовательные траектории, оптимально соответствующие когнитивному стилю и потребностям каждого обучающегося, что особенно важно в условиях разнородных учебных групп.

\ManualSubsection{Технологические, экономические и этические аспекты внедрения}

Успешная интеграция иммерсивных технологий в гуманитарную сферу требует учета комплекса технологических и инфраструктурных факторов. Выбор между устройствами VR (например, Oculus Quest) и AR (Microsoft HoloLens или мобильные решения на базе ARKit/ARCore) определяется конкретными педагогическими задачами, бюджетом и необходимой мобильностью. Архитектура образовательной платформы должна обеспечивать низкую латентность для VR, точный трекинг для AR, а также включать системы управления контентом, аналитики и защиты данных.

Основным барьером для массового внедрения остается высокая стоимость оборудования и разработки качественного образовательного контента, требующего междисциплинарных команд. Однако анализ окупаемости показывает, что долгосрочные выгоды — такие как повышение вовлеченности, улучшение усвоения материала и возможность безопасной отработки навыков в симулируемых средах — могут компенсировать первоначальные инвестиции, особенно в корпоративном и профессиональном образовании.

Крайне важным является учет этических аспектов и приватности. Сбор биометрических и поведенческих данных в иммерсивных средах требует строгого соблюдения принципов информированного согласия, минимизации данных и прозрачности алгоритмов. Необходимо также обеспечивать равенство доступа к технологиям, предотвращая углубление цифрового разрыва, и гарантировать психологическую безопасность пользователей, избегая киберболезни и потенциально травмирующего контента через возрастную классификацию и дозирование сеансов.

\newpage
\refstepcounter{section}
\addcontentsline{toc}{section}{\protect\numberline{\thesection}Тенденции развития иммерсивных технологий и их влияние на современную культуру}
\begin{center}\textbf{\thesection\ ТЕНДЕНЦИИ РАЗВИТИЯ ИММЕРСИВНЫХ ТЕХНОЛОГИЙ И ИХ ВЛИЯНИЕ НА СОВРЕМЕННУЮ КУЛЬТУРУ}\end{center}
\vspace*{-1.5em}
\ManualSubsection{Введение и актуальность}

Иммерсивные технологии, охватывающие виртуальную (VR), дополненную (AR) и смешанную реальность (MR), представляют собой один из наиболее динамично развивающихся сегментов цифровой экономики, радикально трансформирующий взаимодействие человека с информацией и культурным контентом. Прогнозируемый объём инвестиций в эту область на ближайшее десятилетие оценивается в 520 миллиардов долларов, что свидетельствует о её переходе от статуса экспериментальной новинки к роли неотъемлемого компонента профессиональной, образовательной и социокультурной сфер \cite{PeekPro2025}. Изучение данных технологий становится критически важным для понимания формирующихся моделей социальной коммуникации, новых форм художественного выражения и этических вызовов, связанных со стиранием границ между физическим и цифровым опытом.

\ManualSubsection{Определение и классификация иммерсивных технологий}

Термин «иммерсивные технологии» описывает цифровые среды, обеспечивающие глубокое погружение пользователя за счёт интуитивного взаимодействия с контентом и создания устойчивого ощущения присутствия. Ключевым обобщающим понятием является расширенная реальность (XR, Extended Reality), которое служит зонтичным термином для всего спектра соответствующих технологий \cite{XRSI2022}. Виртуальная реальность (VR) предполагает полную замену физического окружения цифровым, что требует специализированных гарнитур и контроллеров. В отличие от неё, дополненная реальность (AR) накладывает цифровые объекты на реальный мир через экран устройства, сохраняя связь пользователя с физическим окружением. Смешанная реальность (MR) представляет собой синтез двух предыдущих подходов, где виртуальные объекты способны взаимодействовать с физическим пространством в реальном времени. Дополнительный уровень иммерсии обеспечивается такими технологиями, как пространственные вычисления, тактильная обратная связь и пространственный звук, которые позволяют системам понимать трёхмерное пространство и передавать пользователю физические ощущения от взаимодействия с цифровыми объектами.

\ManualSubsection{Ключевые технологические тренды (2025–2026 гг.)}

Основным драйвером развития иммерсивных сред в текущий период является глубокая интеграция с системами искусственного интеллекта (AI). Генеративный ИИ позволяет создавать динамичный контент и адаптивные виртуальные среды в реальном времени, что значительно снижает затраты на разработку. Одновременно ИИ используется для персонализации опыта, анализа поведения пользователя и создания неигровых персонажей (NPC) с естественными поведенческими паттернами \cite{DigitalOne2025}. Параллельно наблюдается эволюция цифровых аватаров в сторону гиперреализма и постоянства (персистентности), что формирует основу для новых форм социальной идентичности и цифровой экономики в метавселенных. Другим значимым трендом является коммерциализация и совершенствование устройств смешанной реальности, которые, будучи менее инвазивными, чем классические VR-гарнитуры, становятся предпочтительным форматом для повседневного использования. Развитие сетей связи пятого и шестого поколений (5G/6G) решает проблему задержки, делая возможными сложные многопользовательские сессии и потоковую передачу контента высокого качества. Перспективным, хотя и находящимся на ранних стадиях, направлением является разработка нейроинтерфейсов (интерфейсов «мозг–компьютер»), которые в будущем могут обеспечить прямое управление виртуальной средой.

\ManualSubsection{Влияние на культурные практики и институции}

Влияние иммерсивных технологий на культурную сферу проявляется, прежде всего, в трансформации доступа и взаимодействия с искусством и наследием. Виртуальные музеи и туры, такие как проект Британского музея или опыт Лувра в Абу-Даби, демократизируют доступ к мировым коллекциям, позволяя удалённым аудиториям детально изучать артефакты. AR-приложения в физических музеях обогащают экспозицию, накладывая дополнительную информацию, реконструкции или интерактивные сценарии на реальные объекты. В метавселенных возникают принципиально новые выставочные пространства, свободные от физических ограничений, где искусство может становиться интерактивным и изменяемым. В сфере развлечений иммерсивные технологии переопределяют нарративный опыт, создавая интерактивное кино и театр, где зритель становится активным участником сюжета. Платформы вроде Roblox, с аудиторией в десятки миллионов пользователей, формируют новые социальные практики и культурные миры, где среднее время пребывания значительно превышает показатели традиционных социальных сетей, что свидетельствует о высокой степени вовлечённости, обеспечиваемой эффектом пространственного присутствия \cite{PeekPro2025}.

\ManualSubsection{Образовательные и социальные приложения}

В образовательном контексте иммерсивные технологии обеспечивают переход от пассивного восприятия к активному, экспериментальному обучению. Студенты могут проводить виртуальные химические опыты, исследовать исторические локации или изучать анатомию в трёхмерном пространстве, что приводит к повышению уровня запоминания информации. Профессиональное обучение, особенно в высокорисковых областях, таких как медицина или промышленность, получает мощный инструмент в виде симуляторов, позволяющих отрабатывать сложные процедуры без угрозы для жизни или дорогостоящего оборудования \cite{NextBrick2024}. В социальной сфере технологии XR способствуют формированию гибридных форматов работы через создание виртуальных офисов, где преодолеваются географические барьеры. Однако исследования указывают на двойственный эффект социальных VR-платформ: в то время как социально активные пользователи укрепляют связи, для лиц с низкой самооценкой и изоляцией существует риск усугубления тревожности и депрессии, что требует внимательного изучения психосоциальных последствий.

\ManualSubsection{Этические вызовы и вопросы регулирования}

Бурное развитие иммерсивных технологий сопровождается рядом серьёзных этических и правовых вызовов. Наиболее острым является вопрос конфиденциальности, поскольку устройства XR способны собирать обширный массив биометрических данных, включая движения глаз, мимику и жесты, которые являются индикаторами внимания и эмоционального состояния. Случай с компанией Charlotte Tilbury, оштрафованной на миллионы долларов за сбор биометрии без согласия, иллюстрирует правовые риски и необходимость прозрачных регуляторных рамок. Психологические риски включают потенциальное развитие зависимости, дезориентацию при переходе между реальностями и размывание границ между реальными и виртуальными воспоминаниями. Проблема цифрового неравенства усугубляется высокой стоимостью оборудования и требованиями к инфраструктуре, что может привести к новым формам социальной эксклюзии. Интеграция ИИ порождает вопросы предвзятости алгоритмов и эмоциональной манипуляции. Ответом на эти вызовы должно стать формирование комплексного законодательства, регулирующего сбор биометрических данных, обеспечивающего психологическую безопасность пользователей, особенно несовершеннолетних, и гарантирующего защиту от дискриминации в виртуальных пространствах.

\newpage
\section*{}
\vspace*{-2.5em}
\begin{center}\textbf{ЗАКЛЮЧЕНИЕ}\end{center}
\phantomsection
\addcontentsline{toc}{section}{Заключение}

Иммерсивные технологии, включающие в себя дополненную, виртуальную и смешанную реальность, вышли далеко за рамки инструментов для игр и развлечений, превратившись в мощный механизм трансформации цифровой культуры и создания принципиально новых способов взаимодействия человека с информацией. В ходе исследования был проанализирован полный цикл создания иммерсивных сред: от технических принципов работы VR-шлемов и AR-устройств, основанных на сложных алгоритмах трекинга и пространственного позиционирования (6DoF), до современных методов генерации контента, таких как трёхмерное моделирование, фотограмметрия и текстурирование. Особое внимание было уделено инструментальной экосистеме разработки, включающей как мощные движки Unity и Unreal Engine, так и специализированные фреймворки вроде ARKit и ARCore, которые сегодня позволяют создавать иммерсивные приложения практически для любых платформ, включая мобильные устройства и web-браузеры через WebXR.

Анализ практического применения этих технологий в гуманитарных науках, образовании и культурной сфере показал их значительный потенциал для решения реальных задач. Виртуальные музеи и цифровые реконструкции памятников истории делают культурное наследие доступным для глобальной аудитории, а образовательные симуляции и виртуальные лаборатории создают условия для безопасного и глубокого усвоения сложного материала через мультимодальное восприятие. Социальные VR-платформы и нарождающиеся концепции метавселенных указывают на формирование принципиально новых цифровых пространств для коммуникации, работы и творчества. Однако это же расширение функциональности порождает серьёзные вызовы, связанные с защитой приватности пользователей, этикой сбора биометрических данных, цифровым неравенством и психологическими рисками чрезмерного погружения.

Таким образом, иммерсивные технологии утвердились в качестве одного из ключевых направлений развития цифровой среды XXI века. Их дальнейшая эволюция будет определяться не только успехами в области компьютерного зрения, графического рендеринга и миниатюризации устройств, но и способностью общества сформировать адекватные этические, правовые и педагогические рамки для их ответственного использования. Для будущих специалистов в области цифровых гуманитарных наук, медиа и образования понимание принципов работы, возможностей и ограничений технологий AR/VR становится необходимым условием для создания содержательных, инклюзивных и психологически безопасных иммерсивных проектов, способных обогатить человеческий опыт, а не заменить его.

\newpage
\begin{center}\textbf{СПИСОК СОКРАЩЕНИЙ}\end{center}
\phantomsection
\addcontentsline{toc}{section}{Список сокращений}

\vspace{1em}

{
\small
\noindent
\begin{tabular}{|l|p{0.75\textwidth}|}
\hline
\textbf{Сокращение} & \multicolumn{1}{c|}{\textbf{Расшифровка}} \\
\noalign{\hrule height 1.2pt}
\hline
3D & Three-Dimensional (трёхмерный)\\
\hline
AR & Augmented Reality (дополненная реальность)\\
\hline
VR & Virtual Reality (виртуальная реальность)\\
\hline
MR & Mixed Reality (смешанная реальность)\\
\hline
SLAM & Simultaneous Localization and Mapping (одновременная локализация и картографирование)\\
\hline
6DoF & Six Degrees of Freedom (шесть степеней свободы)\\
\hline
IMU & Inertial Measurement Unit (инерциальный измерительный блок)\\
\hline
FPS & Frames Per Second (кадров в секунду)\\
\hline
PBR & Physically Based Rendering (физически корректный рендеринг)\\
\hline
LOD & Level of Detail (уровень детализации)\\
\hline
UI & User Interface (пользовательский интерфейс)\\
\hline
UX & User Experience (пользовательский опыт)\\
\hline
HRTF & Head-Related Transfer Function (акустическая передаточная функция, связанная с головой)\\
\hline
NPC & Non-Player Character (неигровой персонаж)\\
\hline
NFT & Non-Fungible Token (невзаимозаменяемый токен)\\
\hline
SfM & Structure-from-Motion (восстановление структуры по движению)\\
\hline
STL & Stereolithography (стереолитография, формат файла для 3D-печати)\\
\hline
OBJ & Wavefront .obj file (формат файла геометрического определения)\\
\hline
BPL & Blueprint (визуальный язык скриптинга в Unreal Engine)\\
\hline
SfM & Structure-from-Motion (восстановление структуры по движению)\\
\hline
CAD & Computer-Aided Design (автоматизированное проектирование)\\
\hline
UV & Название координатной системы для текстур (U и V — аналоги осей X и Y в 2D)\\
\hline
\end{tabular}
}

\clearpage
\nocite{*}
\printbibliography[heading=bibliography]

\end{document}
