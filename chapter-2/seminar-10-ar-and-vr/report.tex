% реферат по теме 'Дополненная и виртуальная реальность'
\documentclass{SibFU-docs}

% доп. пакеты
\usepackage{graphicx}
\usepackage{float}
\usepackage{hyperref}
\usepackage{caption}
\usepackage{subcaption}
\usepackage{amsmath}
% предотвартим ошибку с \Bbbk перед загрузкой пакета amssymb
\let\Bbbk\relax
\usepackage{amssymb}
\usepackage{booktabs}
\usepackage{tabularx}
\usepackage{longtable}
\usepackage{enumitem}
\usepackage{multirow}

% гиперссылки
\hypersetup{
    colorlinks=true,
    linkcolor=black,
    citecolor=blue,
    urlcolor=blue
}

% библиография
\addbibresource{report.bib}

% титульный лист
\begin{document}
\makecovertitle{Гуманитарный институт}{Кафедра прикладной информатики в искусстве и интерактивных медиа}{Дополненная и виртуальная реальность: технологии, конвейеры и применение}{ГФ25-02Б}{Захаров И.М., Финочка С.А., Пачковская А.В., Иванова С.Ю., Логинова Ю.В., Шикайкова К.С., Дегтярева В.В.}{Нигматуллин И.Р., ст. преп. каф. ИТвКиКИ}

% олавление
\tableofcontents
\clearpage

% введение
\section{Введение}
\noindent
В этом реферате собраны и структурированы материалы по теме «Дополненная и виртуальная реальность». Цель работы — представить технически корректное, иллюстрированное и практико-ориентированное изложение ключевых понятий, методов и инструментов, применяемых в индустрии и научных исследованиях на 2024–2025 годы.

В работе представлены:
\begin{itemize}[noitemsep]
    \item обзор методов представления 3D-данных и форматов обмена;
    \item разбор этапов создания 3D-контента (моделирование, фотограмметрия, ретопология, текстурирование);
    \item архитектура AR/VR-систем, алгоритмы трекинга и SLAM, требования к задержке и частоте кадров;
    \item сравнение инструментов разработки и практические рекомендации для выбора движка и SDK;
    \item дизайн взаимодействия (контроллеры, трекинг рук, отслеживание пальцев), UX и эргономика;
    \item примеры применения в медицине, образовании и промышленности, а также обсуждение этических вопросов.
\end{itemize}

\section{Краткий обзор содержания источников}
\noindent
Документ охватывает следующие области и задачи:
\begin{itemize}[noitemsep]
    \item Представление 3D-объектов и форматы данных (полигональные сети, NURBS, воксели, облака точек).
    \item Методы создания объектов: моделирование, фотограмметрия, сканирование.
    \item Основы AR/VR: аппаратная и программная архитектура, SLAM, датчики, задержки (латентность) и FPS.
    \item Инструменты разработки: Unity, Unreal, ARKit, ARCore, Vuforia, WebXR.
    \item Дизайн взаимодействия: контроллеры, отслеживание рук и пальцев, тактильная обратная связь.
    \item Примеры применения: медицина, образование, промышленность, развлечения.
\end{itemize}

\section{Представление 3D-данных}
\subsection{Геометрические представления}
\noindent
Современные 3D-системы используют разные подходы к хранению геометрии объекта. В интерактивных приложениях доминируют полигональные сети, тогда как в инженерной практике широко распространены параметрические поверхности и форматы STEP/IGES. В научной визуализации часто применяются воксели и облака точек.

\subsubsection{Полигональные сети}
Полигональные сети (mesh) — надёжный и универсальный способ представления поверхностей. Ниже приведена таблица, суммирующая их ключевые преимущества и ограничения.

\begin{table}[H]
\centering
\begin{tabular}{p{0.32\textwidth}p{0.32\textwidth}p{0.32\textwidth}}
	\toprule
Аспект & Преимущества & Ограничения \\
\midrule
Прозрачность концепции & Простота рендеринга, хорошая поддержка GPU & Большое число полигонов для детализированных форм \\
Аппаратная совместимость & Поддержка движков (Unity, Unreal) & Не всегда удобны для CAE/точного инжиниринга \\
Инструменты оптимизации & LOD, ретопология, нормал карты & Требуют ручной работы и автоматических пайплайнов \\
Масштабирование & Эффективны при отрисовке на GPU & Плохая поддержка для финитных элементов и точных инженерных расчётов \\
\bottomrule
\end{tabular}
\caption{Сравнение полигональных сетей}
\end{table}

\paragraph{Качество и компромиссы} 
При выборе представления модели проектировщик всегда балансирует между точностью, вычислительной стоимостью и требованиями к взаимодействию. Для VR/AR-приложений в реальном времени важны минимальное число вызовов отрисовки (draw calls) и эффективное использование текстур, тогда как для оффлайн-рендеринга или CAD приоритетом остаётся математическая точность (см. \cite{shirley2009}).

\subsubsection{Параметрические поверхности и NURBS}
\noindent
Параметрические поверхности удобны для инженерной работы и обеспечивают высокую точность описания форм, однако для их редактирования и обмена данными обычно используются специальные форматы (STEP/IGES). NURBS особенно полезны там, где требуется точное описание кривых и гладких поверхностей.

\subsubsection{Воксели и облака точек}
\noindent
Воксели удобны для объёмных данных (медицинская визуализация), облака точек — для сканирования и реконструкции сцены.

\section{Методы создания 3D-контента}
\subsection{Ручное моделирование и скульптинг}
\noindent
Классические инструменты (Blender, Maya, 3ds Max, ZBrush) позволяют художникам и инженерам создавать сложные модели. Скульптинг идёт от общего к частному, с постепенным добавлением деталей.

\subsection{Фотограмметрия и 3D-сканирование}
\noindent
Фотограмметрия продолжает оставаться одним из наиболее доступных методов точной реконструкции объектов. Классический рабочий процесс включает подготовку сцены, серийную съёмку, выравнивание изображений, построение облака точек, реконструкцию поверхности, создание текстур и последующую ретопологию.

\noindent
Опыт практических проектов показывает, что на качество результата существенно влияют перекрытие снимков, равномерность освещения и точность калибровки камеры. Подробные рабочие циклы и рекомендации приведены в профильных обзорах и руководствах \cite{agisoft2024,meshroom2024,realitycapture2024}.

\subsubsection{Особенности обработки большого облака точек}
\begin{itemize}
    \item фильтрация шума и удаление выбросов (outlier removal);
    \item выравнивание (registration) нескольких сканов посредством ICP (Iterative Closest Point);
    \item конвертация облака точек в полигональную поверхность (surface reconstruction) с последующей ретопологией.
\end{itemize}

Один из стандартных этапов — оптимизация (bundle adjustment), которая минимизирует суммарную квадратичную ошибку проекций: 
\[\min_{X, P} \sum_{i,j} w_{ij} \| x_{ij} - \pi(P_i, X_j) \|^2,\]
где $X_j$ — 3D-точки, $P_i$ — параметры камеры, $x_{ij}$ — наблюдаемые 2D-координаты, $\pi$ — проекционная функция, а $w_{ij}$ — веса наблюдений. Этот метод подробно разобран в работах по фотограмметрии \cite{remondino2011,kersten2013}.

\subsection{Программная генерация и AI}
\noindent
Современные методы включают процедурную генерацию и применение ИИ для ускорения ретопологии, создания текстур и генерации 3D-моделей по описанию.

\section{Основы AR/VR: архитектура и трекинг}
\subsection{Аппаратная часть и датчики}
\noindent
Ключевые сенсоры: гироскопы, акселерометры, магнитометры, RGB/IR-камеры, Time-of-Flight сенсоры. Комбинация данных от датчиков и видеопотока позволяет решать задачи SLAM и трекинга.

\subsection{SLAM и позиционирование}
\noindent
Реализация SLAM включает обнаружение ключевых точек, сопоставление и оптимизацию (bundle adjustment). Для мобильных устройств применяется визуально-инерциальный подход (VIO). Ниже приведены основные этапы алгоритма VIO:
\begin{enumerate}[noitemsep]
    \item извлечение и сопоставление признаков (feature detection and matching);
    \item краткосрочное оценивание движения (odometry) на основе IMU и оптических потоков;
    \item построение и обновление карты (mapping) — добавление бондов и ключевых кадров;
    \item глобальная оптимизация (pose graph optimization / bundle adjustment).
\end{enumerate}

В мобильных AR-решениях (ARKit, ARCore) применяют гибридный подход, совмещающий визуальные сведения с инерциальными датчиками для повышения устойчивости трекинга \cite{funreality2023,agisoft2024}.

\subsection{Важность латентности и FPS}
\noindent
Низкая латентность и стабильная частота кадров (FPS) критичны для комфортного пользовательского опыта. Для VR часто ориентируются на 90–120 FPS; для AR требование зависит от класса устройства. При этом важно учитывать составные части латентности (motion‑to‑photon):
\begin{itemize}
    \item задержка сенсоров (sensor latency) — время, за которое IMU и гироскопы дают показания;
    \item задержка вычислений (processing latency) — время обработки трекинга, физики и рендера;
    \item задержка дисплея (display latency) — время обновления пикселя на экране гарнитуры.
\end{itemize}

В целом цель проектирования — минимизировать суммарную латентность. На практике для этого применяют техники предсказания движения (prediction) и репроекции кадра (reprojection): предсказание положения головы или контроллера и геометрическая корректировка последнего сгенерированного кадра \cite{jsdrm2018,burley2012}.

\subsubsection{Пример расчёта:}
Пусть сенсор дает обновление за 2 ms, обработка трекинга занимает 6 ms, рендеринг кадра — 10 ms, а дисплей добавляет ещё 4 ms. Тогда суммарная motion-to-photon задержка примерно:
\[ L_{total} = 2 + 6 + 10 + 4 = 22\ \text{ms},\]
что находится на границе комфортного диапазона; целью разработки является снижение до <15 ms для комфорта большинства пользователей.

\section{Инструменты разработки и интеграция}
\subsection{Unity, Unreal и Web}
\noindent
\begin{sloppypar}
Unity и Unreal традиционно занимают разные ниши: Unity удобен для мобильных решений и оперативного прототипирования, тогда как Unreal — для проектов, где критичен высокий уровень визуального качества. Для веба чаще используют WebXR в связке с A-Frame или Three.js. В пайплайнах экспорт контента обычно производится в форматы FBX или glTF/GLB.
\end{sloppypar}

\subsubsection{Выбор движка: критерии}
\noindent
При выборе движка для XR-проекта следует оценивать следующие факторы: целевые платформы, требуемый уровень графического качества, навыки и предпочтения команды (например, \texttt{C\#} для Unity или \texttt{C++}/Blueprints для Unreal) и потребности в интеграции с инструментами автоматизации и развёртывания. Взвешивание этих факторов помогает выбрать оптимальный стек для конкретной задачи.

Недавние исследования предлагают формализованные критерии выбора между Unity и Unreal для XR-проектов \cite{arXiv2025VREngineComparison,DailyDev2024UnityVsUnreal}.

\subsection{Рабочие пайплайны и форматы обмена}
\noindent
В промышленной разработке распространен следующий упрощённый пайплайн обмена артефактами:
\begin{enumerate}[noitemsep]
    \item художественное моделирование (Blender, ZBrush) — экспорт высокополигональных моделей (high-poly) в OBJ/FBX;
    \item ретопология и запекание нормалей (bake) — генерация низкополигональных моделей (low-poly) и карт normal/ao/metal/roughness;
    \item упаковка контента в glTF/GLB для web- и мобильных приложений или FBX для движков.
\end{enumerate}

Процесс подробно описан в профильных гайдлайнах и руководствах по инструментам \cite{unreal2024,blender2024,agisoft2024}.

\section{Дизайн взаимодействия в VR и AR}
\subsection{Контроллеры, трекинг рук и пальцев}
\noindent
Выбор способа ввода определяется сценарием использования: контроллеры сохраняют преимущество в точности для игр и симуляций; отслеживание рук (hand tracking) повышает естественность взаимодействия в социальных и повседневных приложениях; для профессиональных сценариев востребовано отслеживание пальцев и профессиональные тактильные устройства.

\subsubsection{Рекомендации UX}
\begin{itemize}[noitemsep]
    \item избегать резких ускорений и перемещений, использовать телепортацию или сглаженные траектории;
    \item давать визуальные якоря и статичные объекты для ориентирования пользователя;
    \item предоставлять конфигурационные варианты для пользователей с разной чувствительностью к укачиванию;
    \item минимизировать необходимость ввода точных жестов в длительных сессиях — они утомляют.
\end{itemize}

\subsubsection{Оценка эффективности методов ввода}
Ниже приведено рекомендательное сопоставление (баллы 0–10) основных методов ввода на основании обзора литературы и практики:
\begin{table}[H]
\centering
\begingroup
\setlength{\extrarowheight}{2pt}
\begin{tabularx}{\linewidth}{@{}Xrrrrr@{}}
    	\toprule
    Метод & Точность & Надежность & Интуитивность & Доступность & Комфорт \\
    \midrule
    Контроллеры & 9 & 9 & 6 & 7 & 8 \\
    Hand Tracking & 6 & 6 & 9 & 9 & 5 \\
    Finger Tracking & 10 & 8 & 9 & 3 & 7 \\
    \bottomrule
\end{tabularx}
\endgroup
\caption{Сравнение методов ввода в VR/AR}
\end{table}

Сочетание этих методов часто даёт лучший результат: например, гибрид контроллер+hand-tracking улучшает интуитивность без потери точности в задачах требующих сложной манипуляции.

\subsection{UX и эргономика}
\noindent
VR предъявляет высокие требования к эргономике: вес шлема, распределение нагрузки, время сессии и способы перемещения по сцене (телепортация, скользящее движение).

\section{Применение технологий}
\subsection{Медицина}
\noindent
VR применяется в обучении хирургов, симуляции операций и реабилитации. AR используется для навигации в операционной и визуализации данных пациента.

\subsubsection{Кейс: подготовка хирургических бригад}
В нескольких университетах были проведены контрольные исследования, показывающие, что тренировки в VR снижают количество ошибок в реальных операциях и ускоряют время выполнения процедур. Практика включает моделирование анатомии пациента по данным КТ/МРТ (воксельные данные) и воспроизведение сценариев операций \cite{verhoeven2012,remondino2011}.

\subsection{Образование}
\noindent
VR даёт возможность организовать интерактивные лабораторные занятия; AR делает учебники интерактивными и помогает визуализировать сложные объекты. Существуют успешные кейсы внедрения AR в школьное и высшее образование, которые подтверждают повышение вовлечённости и результативности обучения \cite{funreality2023,3dtoday2024}.

\section{Проблемы и перспективы}
\subsection{Киберболезнь и латентность}
\noindent
Проблема киберболезни частично решается аппаратно (высокая частота обновления, малая латентность) и программно — за счёт репроекции (reprojection), предсказания движения (prediction) и специальных режимов комфорта (comfort modes).

\subsection{Этические и социальные вопросы}
\noindent
Интенсивное использование AR/VR предполагает сбор больших объёмов персональных и биометрических данных. Это вызывает вопросы приватности и возможной эксплуатации этих данных в коммерческих и политических целях. Необходимо разрабатывать и внедрять прозрачные политики обработки данных и механизмы согласия пользователей \cite{jsdrm2018}.

\subsection{Экономические и организационные барьеры}
\noindent
Высокая стоимость контента и устройств, нехватка специалистов, юридические риски и необходимость адаптации образовательных программ — ключевые препятствия быстрому распространению технологий.

\section{Рекомендации для практиков}
\begin{itemize}
    \item начинать прототипы с простых сценариев и проверять UX на ранних этапах;
    \item использовать кросс-платформенные форматы (glTF/GLB) для удобства развёртывания;
    \item инвестировать в автоматизацию пайплайнов ретопологии и запекания карт;
    \item предусматривать fallback-режимы для устройств с низкой производительностью.
\end{itemize}

\appendix
\section{Глоссарий}
\noindent
Ниже приведены краткие определения терминов, использованных в тексте:
\begin{description}[noitemsep]
    \item[AR] Дополненная реальность (Augmented Reality) — наложение цифровых объектов на изображение реального мира.
    \item[VR] Виртуальная реальность (Virtual Reality) — полное погружение пользователя в симулированную среду.
    \item[MR] Смешанная реальность (Mixed Reality) — гибрид AR и VR с возможностью взаимодействия виртуальных объектов с реальным окружением.
    \item[SLAM] Simultaneous Localization and Mapping — алгоритмы одновременного картографирования и локализации.
    \item[PBR] Physically Based Rendering — подход к созданию материалов, моделирующий физику отражения света.
\end{description}

\section{Список используемых инструментов и сервисов}
\noindent
Краткий перечень инструментов, упоминаемых в работе:
\begin{itemize}[noitemsep]
    \item Blender, Maya, 3ds Max, ZBrush — инструменты моделирования и скульптинга;
    \item Agisoft Metashape, Meshroom, RealityCapture — ПО для фотограмметрии;
    \item Unity, Unreal Engine, ARKit, ARCore, Vuforia — SDK и движки для AR/VR/ MR;
    \item glTF/GLB, FBX, OBJ, STEP — форматы обмена 3D-данными;
    \item Biber/biblatex — формирование библиографии в LaTeX.
\end{itemize}

\section{Литературный обзор}
\noindent
В последние годы появилось множество работ, посвящённых практическим и теоретическим аспектам AR/VR. Общая часть базируется на фундаментальных текстах по компьютерной графике \cite{shirley2009} и PBR \cite{burley2012}, а также на многочисленных практических гайдах по фотограмметрии и интеграции контента \cite{remondino2011,agisoft2024,meshroom2024}.

Ниже приводится тематическая сводка литературы по ключевым направлениям:
\begin{itemize}[noitemsep]
    \item 3D-репрезентации и обмен форматов: обсуждаются преимущества glTF как формата для web-рендеринга и FBX для движков \cite{unreal2024,blender2024}.
    \item Фотограмметрия и реконструкция: классические методы SfM и современные DL-решения для восстановления текстурированной геометрии \cite{westoby2012,nguyen2020}.
    \item Алгоритмы трекинга и SLAM: обзор визуально-инерциальных подходов и их устойчивость на мобильных платформах \cite{funreality2023,jsdrm2018}.
    \item Инструменты разработки: сравнительный анализ Unity и Unreal для XR-проектов, а также инструменты для web-AR (WebXR, A-Frame) \cite{DailyDev2024UnityVsUnreal,ZapWorks2021WebXR}.
\end{itemize}

\section{Технические детали рендеринга и материалов}
\subsection{Physically Based Rendering (PBR)}
\noindent
PBR-модель опирается на базовые энергетические законы: суммарный отражённый поток не может превышать падающий. Современные графические пайплайны применяют модели распределённой функции отражения (BRDF), например, модель Cook-Torrance. Для практического применения используются каналы: albedo, normal, metallic, roughness, ao. Формула освещённости для диффузной составляющей при ламбертовском отражении:
\[ L_d = \frac{\rho}{\pi} \sum_{i} L_i (n \cdot l_i), \]
где $\rho$ — альбедо, $L_i$ — интенсивность источника, $n$ — нормаль, $l_i$ — направление на источник. Для зеркальной составляющей применяется модель microfacet с функцией распределения нормалей $D(\omega)$ и функцией геометрии $G$.

\subsection{Оптимизации шейдеров для мобильных устройств}
\begin{itemize}[noitemsep]
    \item использование менее дорогих алгебраических операций (замена \texttt{pow} на более быстрые приближённые реализации);
    \item предварительное запекание освещения (pre-baked lighting) и использование light probes для статического окружения;
    \item редукция числа текстурных выборок путём комбинирования карт в атласах и применение форматов сжатия текстур (ETC2, ASTC);
    \item раннее отбрасывание глубины (early-z) и независимая от порядка обработка прозрачности (order-independent transparency) при необходимости.
\end{itemize}

\section{Углублённый разбор SLAM и фильтрации датчиков}
\subsection{Фильтрация показаний IMU}
\noindent
Одним из ключевых компонентов устойчивого VIO является фильтрация шума в измерениях акселерометра и гироскопа. Часто применяются расширенный фильтр Калмана (EKF) или его негативные варианты (UKF), а также скользящие средние и фильтры низких частот на этапе предобработки. Простейшая модель обновления ориентации через интегрирование гироскопа:
\[ q_{t+1} = q_t \otimes \exp\left(\frac{1}{2} \Delta t \, \omega_t\right), \]
где $q$ — кватернион ориентации, $\omega_t$ — вектор угловой скорости, $\otimes$ — кватернионное умножение.

\subsection{Pose Graph Optimization}
\noindent
Для глобальной согласованности карты используется оптимизация графа поз (pose graph). Минимизация ошибки обычно формулируется как ненелинейная оптимизационная задача, решаемая методом Левенберга–Марквардта.

\subsection{Псевдокод упрощённого VIO-пайплайна}
\begin{verbatim}
Initialize state (pose, velocity, biases)
for each frame do
    read IMU samples since last frame
    propagate state by IMU integration
    capture image frame
    extract features and match to map
    if enough matches then
        construct measurement residuals
        run optimization (local BA)
    end
    occasionally add keyframe and marginalize
end
\end{verbatim}

\section{Пример реализации: мини-проект прототипа AR-просмотра мебели}
\noindent
Ниже приведён план прототипа, который можно реализовать за 4–6 недель в малой команде (1–2 разработчика + 1 художник):
\begin{enumerate}[noitemsep]
    \item Сбор требований и выбор платформы (mobile iOS/Android или standalone)
    \item Подготовка 3D-аксетов: low-poly модели мебели, карты PBR
    \item Выбор SDK: ARCore/ARKit через Unity AR Foundation
    \item Реализация плейсмента: обнаружение плоскостей, якорение, сохранение позиции
    \item UX: контроль масштаба, поворот жестами, возврат в начальное состояние
    \item Тестирование на 3–5 устройствах с разной производительностью
    \item Сбор метрик: время на плейсмент, число сбоев трекинга, user satisfaction
\end{enumerate}

\section{Методология оценки UX и производительности}
\noindent
Методология оценки должна сочетать объективные метрики (FPS, латентность, число сбоев трекинга) и субъективные данные — опросы пользователей и стандартизированные шкалы комфорта (например, SSQ). Для принятия решений по UX полезно проводить A/B‑тестирование альтернативных режимов.

\section{Экономические аспекты и оценка стоимости разработки}
\noindent
Оценка стоимости XR-проекта должна учитывать стоимость создания контента (3D‑художники, аниматоры), часы разработки, лицензии на движки и SDK, расходы на тестирование на реальных устройствах и маркетинг. Малые пилотные проекты часто укладываются в диапазон 10–20 тыс. USD, в то время как масштабные промышленные внедрения требуют существенно больших инвестиций.

\section{Обсуждение и рекомендации для исследователей}
\noindent
Исследователям рекомендуется сосредоточиться на следующих направлениях:
\begin{itemize}[noitemsep]
    \item улучшение энергоэффективных алгоритмов трекинга для мобильных устройств;
    \item адаптивные модели освещения и материализации для смешанной реальности;
    \item этически-ориентированные протоколы сбора биометрических данных;
    \item универсальные форматы обмена и метаданные для совместимости контента между платформами.
\end{itemize}

\section{Заключение}
\noindent
В данной работе было подробно рассмотрено состояние технологий дополненной и виртуальной реальности, методы создания контента, алгоритмы трекинга и практические рекомендации по внедрению. Технологии XR продолжают развиваться быстрыми темпами, и междисциплинарная кооперация — ключ к созданию безопасных, доступных и полезных приложений.

\clearpage
\printbibliography[heading=bibliography]

\end{document}
