% реферат: Табличные данные и базы данных (реляционные и нереляционные, СУБД)

\documentclass{SibFU-docs}

% для предотвращения СТРАШНОЙ ошибки с OT1
\fontencoding{T2A}\selectfont

% графика и математика
\usepackage{graphicx}
\usepackage{amsmath}

% листинги кода (SQL, только ASCII!)
\usepackage{listings}
\lstset{basicstyle=\ttfamily\small,breaklines=true,frame=single}

% таблицы
\usepackage{booktabs}
\usepackage{longtable}
\usepackage{multirow}
\usepackage{array}

% подписи и float-окружения
\usepackage{caption}
\usepackage{float}

% гиперссылки
\usepackage{hyperref}
\hypersetup{
    colorlinks=true,
    linkcolor=black,
    citecolor=blue,
    urlcolor=blue
}

% библиография (класс уже подключил biblatex, просто добавляем источник)
\addbibresource{report.bib}

\begin{document}

% ----------------- титульный лист -----------------
\makecovertitle%
{Гуманитарный институт}%
{Кафедра прикладной информатики в искусстве и интерактивных медиа}%
{ТАБЛИЧНЫЕ ДАННЫЕ И БАЗЫ ДАННЫХ (РЕЛЯЦИОННЫЕ И НЕРЕЛЯЦИОННЫЕ, СУБД)}%
{ГФ25-02Б}%
{Тетерина П.А., Иванова С.Ю., Логинова Ю.В., Ситников А.В., Захаров И.М., Арсенян А.В.}

% ----------------- аннотация -----------------
\begin{abstract}
Данный реферат представляет собой комплексное исследование принципов организации, структурирования и хранения данных в современных системах управления базами данных (СУБД). Работа охватывает теоретические основы реляционной модели, заложенные Э.~Ф.~Коддом, процессы нормализации баз данных, сравнительный анализ реляционных (SQL) и нереляционных (NoSQL) подходов к организации данных, функциональные возможности СУБД, роль языка SQL как универсального инструмента взаимодействия с данными, концепции целостности и надёжности хранения информации, а также современные облачные технологии и инструменты для работы с базами данных. Особое внимание уделено практическим аспектам применения баз данных в контексте разработки интерактивных медиа и цифровых проектов. Работа подготовлена студентами первого курса направления «Прикладная информатика в искусстве и интерактивных медиа» Сибирского федерального университета.
\end{abstract}

\tableofcontents
\newpage

% ----------------- тело реферата -----------------
\section{Введение}

Современная цифровая экономика характеризуется экспоненциальным ростом объёмов данных. По оценкам исследовательских компаний, к 2025 году объём мировых данных превысит 175 зеттабайт. В этих условиях эффективное хранение, структурирование и управление данными становится критически важной задачей для организаций любого масштаба — от стартапов до транснациональных корпораций.

Базы данных (БД) и системы управления базами данных (СУБД) являются фундаментом современных информационных систем. Они обеспечивают надёжное хранение данных, быстрый доступ к информации, поддержку целостности и согласованности данных при параллельном доступе множества пользователей. Область применения баз данных чрезвычайно широка: от банковских систем и систем бронирования авиабилетов до социальных сетей, интернет-магазинов и мультимедийных платформ.

Теоретические основы реляционных баз данных были заложены в 1970 году Эдгаром Коддом в его основополагающей работе «A Relational Model of Data for Large Shared Data Banks»~\cite{codd1970}. Реляционная модель данных и язык SQL остаются доминирующими технологиями в области управления данными уже более полувека. Классические учебники по базам данных~\cite{date2003, elmasri2016} подробно описывают принципы проектирования и функционирования реляционных СУБД.

Однако с начала XXI века, особенно с развитием веб-приложений и требований к обработке больших данных (Big Data), появились альтернативные подходы к организации данных — нереляционные базы данных (NoSQL). Эти системы отказываются от жёстких табличных схем в пользу более гибких моделей, обеспечивая горизонтальную масштабируемость и высокую производительность~\cite{sadalage2012}.

Современные распределённые системы ставят новые вызовы перед разработчиками баз данных: необходимость обеспечения согласованности данных при репликации, механизмы консенсуса~\cite{ongaro2014, lamport1998}, глобально распределённые транзакции~\cite{corbett2012}. Облачные провайдеры предлагают управляемые сервисы баз данных (DBaaS), снимая операционную нагрузку с разработчиков приложений.

Целью данного реферата является систематизация знаний о принципах организации баз данных, сравнение различных подходов к хранению данных, анализ функциональных возможностей СУБД и рассмотрение современных инструментов для работы с данными. При подготовке реферата были использованы классические академические работы~\cite{codd1970, date2003, elmasri2016}, исследования в области транзакций и восстановления~\cite{haerder1983, gray1993}, документация современных СУБД~\cite{postgresql_docs, mongo_docs, clickhouse_docs} и облачных сервисов, а также спецификации стандартов защиты данных~\cite{gdpr}.

Структура работы включает шесть основных разделов, последовательно раскрывающих ключевые аспекты организации баз данных: от фундаментальных принципов структурирования и нормализации до современных облачных технологий и инструментов администрирования.

\section{Основные принципы структурирования и нормализации баз данных}

\subsection{Фундаментальные концепции баз данных}

База данных представляет собой упорядоченный набор структурированной информации о конкретной предметной области. Система управления базами данных (СУБД) — это программный комплекс, предоставляющий средства для создания баз данных, выполнения операций над данными (добавление, изменение, удаление, поиск) и управления доступом пользователей.

Качественная база данных должна удовлетворять следующим требованиям:
\begin{itemize}
  \item \textbf{Производительность:} обеспечение быстрого ответа на запросы пользователей, эффективное использование ресурсов системы;
  \item \textbf{Минимизация избыточности:} устранение дублирования данных, что экономит дисковое пространство и упрощает обновление информации;
  \item \textbf{Целостность данных:} поддержание точности и непротиворечивости информации на протяжении всего жизненного цикла;
  \item \textbf{Безопасность:} защита данных от несанкционированного доступа, обеспечение конфиденциальности и соблюдение нормативных требований;
  \item \textbf{Удобство использования:} простота выполнения операций для конечных пользователей и администраторов.
\end{itemize}

Центральной проблемой проектирования баз данных является поиск баланса между производительностью системы и лёгкостью сопровождения. Денормализованные структуры данных могут обеспечивать более быстрый доступ к информации, но усложняют процедуры обновления и повышают риск возникновения аномалий данных.

\subsection{Модели данных}

Данные в базе организуются в соответствии с определённой моделью данных — набором концепций и правил для описания структуры, взаимосвязей и ограничений целостности. Исторически сложились следующие основные модели:

\textbf{Иерархическая модель} представляет данные в виде древовидной структуры, где каждый узел может иметь только одного родителя. Эта модель эффективна для представления данных с естественной иерархией (например, организационные структуры), но недостаточно гибка для описания сложных связей.

\textbf{Сетевая модель} расширяет иерархическую, позволяя узлам иметь несколько родителей. Это обеспечивает большую выразительность, но усложняет навигацию по данным и требует от программистов детального знания физической структуры базы.

\textbf{Реляционная модель}, предложенная Коддом~\cite{codd1970}, стала революционным шагом в развитии технологий баз данных. Данные представляются в виде набора двумерных таблиц (отношений), состоящих из строк (кортежей) и столбцов (атрибутов). Ключевые преимущества реляционной модели включают:
\begin{itemize}
  \item простоту и интуитивность табличного представления;
  \item строгую математическую основу (реляционная алгебра и реляционное исчисление);
  \item независимость данных: физическая реализация отделена от логического представления;
  \item декларативный язык запросов SQL, позволяющий описывать \textit{что} необходимо получить, а не \textit{как} это сделать.
\end{itemize}

\subsection{Основные понятия реляционной модели}

В реляционной модели используются следующие фундаментальные понятия:

\textbf{Атрибут (столбец)} определяет тип данных и представляет определённую характеристику описываемых объектов. Например, в таблице «Студенты» атрибутами могут быть: идентификатор студента, фамилия, имя, дата рождения, группа.

\textbf{Кортеж (строка)} представляет собой конкретный набор значений атрибутов, описывающий один объект предметной области.

\textbf{Отношение (таблица)} — это множество кортежей с одинаковой структурой атрибутов. Важно, что в реляционной модели порядок строк и столбцов не имеет значения.

\textbf{Первичный ключ} — атрибут или комбинация атрибутов, уникально идентифицирующий каждый кортеж в отношении. Первичный ключ не может содержать пустых (NULL) значений.

\textbf{Внешний ключ} — атрибут или набор атрибутов в одном отношении, который ссылается на первичный ключ другого отношения, обеспечивая связь между таблицами и поддерживая ссылочную целостность.

\subsection{Процесс нормализации}

Нормализация — это формальная процедура декомпозиции отношений с целью устранения избыточности данных и предотвращения аномалий обновления, вставки и удаления. Нормализация основывается на теории функциональных зависимостей между атрибутами~\cite{date2003}.

\textbf{Первая нормальная форма (1НФ)} требует, чтобы значения всех атрибутов были атомарными (неделимыми). Недопустимо хранение списков значений или повторяющихся групп в одной ячейке таблицы. Например, поле «Телефоны» не должно содержать несколько номеров через запятую; вместо этого следует создать отдельную таблицу для телефонов с внешним ключом.

\textbf{Вторая нормальная форма (2НФ)} применима к отношениям с составным первичным ключом. Отношение находится во 2НФ, если оно находится в 1НФ и каждый неключевой атрибут функционально полно зависит от первичного ключа (то есть зависит от всего ключа целиком, а не от его части). Нарушение 2НФ приводит к частичной зависимости и дублированию данных.

\textbf{Третья нормальная форма (3НФ)} исключает транзитивные зависимости неключевых атрибутов. Отношение находится в 3НФ, если оно находится во 2НФ и ни один неключевой атрибут не зависит от другого неключевого атрибута. Например, если в таблице «Сотрудники» хранится код отдела и название отдела, возникает транзитивная зависимость (название зависит от кода, а код — от первичного ключа сотрудника). Правильным решением будет вынести информацию об отделах в отдельную таблицу.

\textbf{Нормальная форма Бойса-Кодда (НФБК)} усиливает требования 3НФ: отношение находится в НФБК, если для каждой нетривиальной функциональной зависимости детерминант является потенциальным ключом. НФБК устраняет определённые аномалии, которые могут сохраняться в 3НФ при наличии перекрывающихся потенциальных ключей.

\textbf{Четвёртая нормальная форма (4НФ)} устраняет многозначные зависимости. Отношение находится в 4НФ, если оно находится в НФБК и не содержит нетривиальных многозначных зависимостей. Многозначная зависимость возникает, когда атрибут определяет множество значений других атрибутов независимо друг от друга.

\textbf{Пятая нормальная форма (5НФ)} или проекционно-соединительная нормальная форма устраняет зависимости соединения. Отношение находится в 5НФ, если каждая зависимость соединения в нём определяется потенциальным ключом.

На практике большинство баз данных нормализуются до 3НФ, что обеспечивает хороший баланс между устранением избыточности и сложностью структуры. Дальнейшая нормализация применяется в специфических случаях для решения конкретных проблем.

\subsection{Практические аспекты проектирования}

При проектировании баз данных следует учитывать не только формальные критерии нормализации, но и практические соображения:
\begin{itemize}
  \item производительность запросов: иногда контролируемая денормализация может ускорить выполнение часто используемых запросов;
  \item объём данных: для больших таблиц даже небольшая избыточность может привести к значительному расходу дискового пространства;
  \item сложность запросов: чрезмерная нормализация может потребовать многочисленных соединений таблиц, усложняя запросы и снижая производительность.
\end{itemize}

Правильное проектирование схемы базы данных является критически важным этапом разработки информационной системы и во многом определяет её дальнейшую эффективность и способность к масштабированию.

\section{Отличие реляционных и нереляционных моделей данных}

\subsection{Контекст возникновения NoSQL}

Термин «NoSQL» получил широкое распространение в конце 2000-х годов, хотя отдельные нереляционные системы существовали и ранее. Движущей силой развития NoSQL-технологий стали потребности крупных интернет-компаний (Google, Amazon, Facebook), столкнувшихся с ограничениями традиционных реляционных СУБД при работе с петабайтами данных и миллионами одновременных пользователей~\cite{sadalage2012}.

NoSQL часто расшифровывается как «Not Only SQL», подчёркивая дополняющий, а не заменяющий характер этих технологий по отношению к реляционным базам данных. Нереляционные СУБД отказываются от жёсткой табличной схемы и строгих транзакционных гарантий в пользу гибкости, высокой доступности и горизонтальной масштабируемости.

\subsection{Фундаментальные различия}

\textbf{Схема данных.} Реляционные СУБД требуют предварительного определения структуры данных (DDL-схемы) до начала работы. Любые изменения схемы (добавление столбцов, изменение типов данных) могут быть сложными и затратными операциями, особенно для больших таблиц. NoSQL-системы, напротив, используют динамические схемы (schema-less или schema-flexible подходы), позволяя добавлять новые поля без модификации существующих записей.

\textbf{Масштабирование.} Реляционные СУБД традиционно используют вертикальное масштабирование (scale-up): увеличение мощности сервера путём добавления процессоров, памяти, более быстрых дисков. Этот подход имеет физические и экономические ограничения. NoSQL-системы спроектированы для горизонтального масштабирования (scale-out): распределение данных между множеством серверов (узлов кластера). Добавление новых узлов увеличивает общую производительность системы линейно или близко к линейно.

\textbf{Транзакционные гарантии.} Реляционные СУБД строго следуют принципам ACID (Atomicity, Consistency, Isolation, Durability), обеспечивая строгую согласованность данных. NoSQL-системы часто следуют принципу BASE (Basically Available, Soft state, Eventually consistent), жертвуя немедленной согласованностью ради доступности и производительности.

\subsection{Типы нереляционных баз данных}

Нереляционные СУБД неоднородны и включают несколько специализированных типов, каждый из которых оптимизирован для определённых сценариев использования.

\textbf{Документные базы данных} (MongoDB~\cite{mongo_docs}, CouchDB, Couchbase) хранят данные в виде документов — самодостаточных структур данных в форматах JSON, BSON или XML. Каждый документ может иметь собственную структуру, что обеспечивает гибкость схемы. Документные БД хорошо подходят для систем управления контентом, каталогов товаров, пользовательских профилей.

\textbf{Базы данных «ключ-значение»} (Redis, Amazon DynamoDB, Riak) реализуют простейшую модель: значение ассоциируется с уникальным ключом и может быть извлечено по этому ключу. Эта модель обеспечивает максимальную производительность и применяется для кэширования, хранения сессий пользователей, реализации очередей сообщений.

\textbf{Колоночные базы данных} (Apache Cassandra, HBase, ClickHouse~\cite{clickhouse_docs}) организуют данные не по строкам, а по столбцам. Это позволяет эффективно выполнять аналитические запросы, агрегирующие данные по определённым атрибутам, и обеспечивает высокую степень сжатия данных. Колоночные СУБД применяются в системах аналитики больших данных, хранилищах данных (data warehouses), системах мониторинга и сбора метрик.

\textbf{Графовые базы данных} (Neo4j, Amazon Neptune, ArangoDB) представляют данные в виде графа: узлы соответствуют сущностям, рёбра — связям между ними. Графовые БД эффективны для задач, требующих анализа связей: социальные сети, системы рекомендаций, обнаружение мошенничества, анализ зависимостей в программном обеспечении.

\subsection{Сравнительный анализ}

Выбор между реляционной и нереляционной моделью зависит от требований конкретного проекта. Реляционные СУБД предпочтительны для:
\begin{itemize}
  \item приложений с чётко определённой структурой данных;
  \item систем, требующих строгих транзакционных гарантий (банковские системы, системы бронирования);
  \item сложных аналитических запросов с множественными соединениями таблиц;
  \item проектов, где важна стандартизация (SQL является международным стандартом).
\end{itemize}

Нереляционные СУБД эффективны для:
\begin{itemize}
  \item приложений с высокой нагрузкой и требованиями к горизонтальному масштабированию;
  \item систем с динамически изменяющейся структурой данных;
  \item проектов, где приоритетом является доступность, а не строгая согласованность;
  \item специализированных задач (полнотекстовый поиск, графовая аналитика).
\end{itemize}

Современная архитектура приложений часто основана на принципе полиглотной персистентности (polyglot persistence) — использовании различных типов хранилищ данных для различных компонентов системы в соответствии с их специфическими требованиями.

\section{Основные функции и возможности систем управления базами данных}

\subsection{Определение и назначение СУБД}

Система управления базами данных — это комплекс программных средств, обеспечивающих создание, модификацию и управление базами данных. СУБД выполняет роль посредника между пользователями (или приложениями) и физическим хранилищем данных, предоставляя высокоуровневый интерфейс для работы с информацией.

Без СУБД управление данными было бы сопоставимо с поиском нужной книги в огромной библиотеке без системы каталогов. СУБД обеспечивает структурированное хранение, быстрый поиск, контроль целостности и безопасный многопользовательский доступ к данным.

\subsection{Базовые операции CRUD}

Фундаментальными операциями для работы с данными являются:
\begin{itemize}
  \item \textbf{Create (создание):} добавление новых записей в базу данных;
  \item \textbf{Read (чтение):} извлечение информации согласно заданным критериям;
  \item \textbf{Update (обновление):} модификация существующих записей;
  \item \textbf{Delete (удаление):} удаление записей из базы данных.
\end{itemize}

В реляционных СУБД эти операции реализуются средствами языка SQL через операторы INSERT, SELECT, UPDATE и DELETE соответственно. Для определения структуры данных используются операторы языка DDL (Data Definition Language): CREATE, ALTER, DROP.

\subsection{Управление транзакциями и свойства ACID}

Транзакция — это последовательность операций, которая должна быть выполнена как единое целое. Классическим примером является перевод денежных средств между счетами: операция списания со счёта отправителя и зачисления на счёт получателя должны выполниться обе, либо ни одна из них.

Свойства ACID обеспечивают надёжность транзакционной обработки~\cite{gray1993, haerder1983}:

\textbf{Atomicity (атомарность):} транзакция выполняется полностью или не выполняется вовсе. В случае ошибки все изменения откатываются.

\textbf{Consistency (согласованность):} транзакция переводит базу данных из одного целостного состояния в другое, соблюдая все определённые ограничения целостности.

\textbf{Isolation (изолированность):} параллельно выполняющиеся транзакции не влияют друг на друга. Результат параллельного выполнения транзакций должен быть эквивалентен их последовательному выполнению.

\textbf{Durability (долговечность):} после успешного завершения транзакции её результаты сохраняются в системе даже в случае последующего сбоя оборудования.

Для реализации этих свойств СУБД используют различные механизмы: журналирование операций (write-ahead logging), контрольные точки (checkpoints), блокировки (locks) и многоверсионность (MVCC — Multi-Version Concurrency Control). MVCC позволяет читателям не блокировать писателей и наоборот, значительно повышая параллелизм в системах с высокой нагрузкой~\cite{postgresql_docs}.

\subsection{Индексация и оптимизация запросов}

Индекс — это вспомогательная структура данных, ускоряющая поиск информации в таблице, подобно предметному указателю в конце учебника. Индексы критически важны для производительности, но требуют баланса: каждый индекс ускоряет чтение, но замедляет операции записи и занимает дополнительное дисковое пространство.

Основные типы индексов:
\begin{itemize}
  \item \textbf{B-деревья:} универсальные сбалансированные деревья поиска, эффективные для точного и диапазонного поиска, поддерживают сортировку;
  \item \textbf{Хеш-индексы:} обеспечивают очень быстрый поиск точных совпадений, но не поддерживают диапазонные запросы;
  \item \textbf{LSM-деревья:} оптимизированы для интенсивной записи, применяются в системах с высокой частотой обновлений;
  \item \textbf{Специализированные индексы:} полнотекстовые (GIN в PostgreSQL), пространственные (R-деревья для геоданных), индексы для JSON-данных.
\end{itemize}

Оптимизатор запросов в СУБД анализирует различные способы выполнения запроса (планы выполнения) и выбирает наиболее эффективный на основе статистики данных и стоимостной модели. Для анализа планов выполнения используются команды EXPLAIN и EXPLAIN ANALYZE.

\subsection{Обеспечение целостности данных}

СУБД предоставляет декларативные механизмы для поддержания целостности:
\begin{itemize}
  \item \textbf{Ограничения первичного ключа (PRIMARY KEY):} гарантируют уникальность записей;
  \item \textbf{Ограничения внешнего ключа (FOREIGN KEY):} поддерживают ссылочную целостность между таблицами;
  \item \textbf{Ограничения уникальности (UNIQUE):} запрещают дублирование значений в столбце;
  \item \textbf{Ограничения значений (CHECK):} проверяют соответствие данных бизнес-правилам;
  \item \textbf{Ограничения NOT NULL:} запрещают пустые значения.
\end{itemize}

Дополнительно СУБД поддерживает триггеры — хранимые процедуры, автоматически выполняющиеся при определённых событиях (вставка, обновление, удаление), что позволяет реализовывать сложную бизнес-логику на уровне базы данных.

\subsection{Репликация и отказоустойчивость}

Репликация — это процесс создания и поддержания копий данных на нескольких серверах. Репликация обеспечивает:
\begin{itemize}
  \item отказоустойчивость: при выходе из строя основного сервера система может переключиться на реплику;
  \item масштабирование чтения: запросы на чтение могут распределяться между репликами;
  \item географическое распределение: размещение данных ближе к пользователям снижает задержки.
\end{itemize}

Различают синхронную репликацию (гарантирует отсутствие потери данных, но увеличивает задержки) и асинхронную (лучшая производительность, но возможна потеря последних изменений при сбое).

Современные распределённые СУБД используют алгоритмы консенсуса (Paxos~\cite{lamport1998}, Raft~\cite{ongaro2014}) для обеспечения согласованности данных при репликации и обработки отказов узлов.

\subsection{Резервное копирование и восстановление}

Стратегия резервного копирования определяется требованиями к двум ключевым метрикам:
\begin{itemize}
  \item \textbf{RPO (Recovery Point Objective):} максимально допустимый объём потери данных;
  \item \textbf{RTO (Recovery Time Objective):} максимально допустимое время восстановления работы системы.
\end{itemize}

СУБД поддерживают различные типы резервных копий: полные (complete backup), инкрементные (incremental backup), дифференциальные (differential backup). Критически важна регулярная проверка процедур восстановления: резервная копия имеет ценность только если её можно успешно восстановить.

\section{Язык SQL как инструмент взаимодействия с базами данных}

\subsection{SQL: история и стандартизация}

SQL (Structured Query Language) — декларативный язык программирования, предназначенный для управления данными в реляционных базах данных. SQL был разработан в IBM в начале 1970-х годов на основе реляционной алгебры Кодда. Первый стандарт SQL был принят ANSI в 1986 году, а ISO — в 1987 году.

Принципиальным преимуществом SQL является его декларативность: пользователь описывает \textit{что} нужно получить, а не \textit{как} это сделать. СУБД самостоятельно определяет оптимальный способ выполнения запроса на основе статистики данных и доступных индексов.

SQL включает несколько подъязыков:
\begin{itemize}
  \item \textbf{DDL (Data Definition Language):} определение структуры данных (CREATE, ALTER, DROP);
  \item \textbf{DML (Data Manipulation Language):} манипулирование данными (SELECT, INSERT, UPDATE, DELETE);
  \item \textbf{DCL (Data Control Language):} управление доступом (GRANT, REVOKE);
  \item \textbf{TCL (Transaction Control Language):} управление транзакциями (BEGIN, COMMIT, ROLLBACK).
\end{itemize}

\subsection{Основные операторы SQL}

Оператор SELECT является центральным элементом языка запросов:

\begin{lstlisting}[language=SQL, caption={Basic SELECT query structure}]
SELECT column1, column2, aggregate_function(column3)
FROM table1
JOIN table2 ON table1.id = table2.foreign_id
WHERE condition
GROUP BY column1, column2
HAVING aggregate_condition
ORDER BY column1 DESC
LIMIT 100;
\end{lstlisting}

Современные версии SQL поддерживают расширенные возможности:
\begin{itemize}
  \item оконные функции (window functions) для аналитических вычислений;
  \item общие табличные выражения (CTE — Common Table Expressions) для структурирования сложных запросов;
  \item операции с JSON/JSONB данными;
  \item рекурсивные запросы для работы с иерархическими данными.
\end{itemize}

\subsection{Расширения и диалекты SQL}

Несмотря на существование стандарта ISO SQL, различные СУБД вносят собственные расширения, формируя диалекты языка: T-SQL (Microsoft SQL Server), PL/SQL (Oracle), PL/pgSQL (PostgreSQL). Эти расширения включают процедурные конструкции (циклы, условия, обработку исключений), позволяющие создавать сложную бизнес-логику внутри базы данных.

\subsection{Применение SQL в различных отраслях}

SQL нашёл применение во всех отраслях, требующих хранения и анализа структурированных данных:
\begin{itemize}
  \item \textbf{Финансовый сектор:} обработка транзакций, анализ кредитных рисков, выявление мошенничества;
  \item \textbf{Электронная коммерция:} управление каталогами товаров, обработка заказов, анализ поведения покупателей;
  \item \textbf{Социальные сети:} хранение пользовательских данных, лент новостей, взаимосвязей между пользователями;
  \item \textbf{Государственный сектор:} ведение реестров, учёт предоставляемых услуг, статистическая отчётность.
\end{itemize}

\subsection{Ограничения SQL и альтернативные подходы}

Несмотря на широкое распространение, SQL имеет ряд ограничений:
\begin{itemize}
  \item сложность освоения для непрограммистов (несмотря на первоначальные цели простоты);
  \item различия в диалектах усложняют переносимость кода между СУБД;
  \item неэффективность для определённых типов запросов (например, графовые обходы);
  \item необходимость объектно-реляционного отображения (ORM) при работе с объектно-ориентированными языками программирования.
\end{itemize}

Для специфических задач разрабатываются альтернативные языки запросов: Cypher для графовых баз данных Neo4j, MongoDB Query Language для документных баз данных, GraphQL для API.

\section{Концепции целостности и надёжности в системах хранения данных}

\subsection{Типы целостности данных}

Целостность данных означает точность, полноту и непротиворечивость информации в базе данных на протяжении всего жизненного цикла. Различают несколько типов целостности:

\textbf{Сущностная целостность} гарантирует уникальность каждой записи в таблице. Реализуется через первичные ключи, которые не могут содержать пустых значений и должны быть уникальными для каждой строки.

\textbf{Ссылочная целостность} обеспечивает корректность связей между таблицами. Внешние ключи предотвращают создание «висячих ссылок» — записей, ссылающихся на несуществующие данные. СУБД поддерживает каскадные операции: CASCADE (автоматическое удаление или обновление связанных записей), SET NULL (установка NULL при удалении ссылаемой записи), RESTRICT (запрет удаления при наличии ссылок).

\textbf{Семантическая целостность} гарантирует соответствие данных бизнес-правилам предметной области. Реализуется через ограничения CHECK, триггеры и хранимые процедуры.

\subsection{Уровни изоляции транзакций}

Стандарт SQL определяет четыре уровня изоляции транзакций, различающихся типами допускаемых аномалий параллельного доступа:

\textbf{Read Uncommitted} — минимальный уровень изоляции, допускающий чтение незафиксированных изменений других транзакций («грязное» чтение). Используется редко из-за высокого риска получения некорректных данных.

\textbf{Read Committed} — транзакция видит только зафиксированные изменения других транзакций. Предотвращает «грязное» чтение, но допускает неповторяющееся чтение (повторный запрос может вернуть изменённые данные).

\textbf{Repeatable Read} — гарантирует, что повторное чтение тех же данных вернёт одинаковый результат. Предотвращает «грязное» и неповторяющееся чтение, но допускает фантомное чтение (появление новых строк, удовлетворяющих условию запроса).

\textbf{Serializable} — наивысший уровень изоляции, обеспечивающий полную изолированность транзакций. Результат параллельного выполнения транзакций эквивалентен их последовательному выполнению.

Выбор уровня изоляции определяется балансом между согласованностью данных и производительностью системы.

\subsection{Теорема CAP и распределённые системы}

Теорема CAP (Brewer's theorem) утверждает, что распределённая система может одновременно гарантировать не более двух из трёх свойств:
\begin{itemize}
  \item \textbf{Consistency (согласованность):} все узлы системы видят одинаковые данные в один момент времени;
  \item \textbf{Availability (доступность):} система отвечает на каждый запрос (успешно или с ошибкой);
  \item \textbf{Partition tolerance (устойчивость к разделению):} система продолжает функционировать при потере связи между узлами.
\end{itemize}

На практике разделение сети (partition) неизбежно в распределённых системах, поэтому выбор сводится к компромиссу между согласованностью и доступностью. Реляционные СУБД традиционно выбирают согласованность (CP-системы), в то время как многие NoSQL-системы предпочитают доступность (AP-системы).

Современные распределённые СУБД, такие как Google Spanner~\cite{corbett2012} и CockroachDB~\cite{cockroach}, используют инновационные подходы (например, синхронизация атомных часов) для обеспечения строгой согласованности в глобально распределённых системах.

\subsection{Модели согласованности}

В распределённых системах существует спектр моделей согласованности:

\textbf{Строгая согласованность (strong consistency):} любое чтение возвращает результат последней записи. Достигается за счёт координации между узлами, что увеличивает задержки.

\textbf{Согласованность в конечном счёте (eventual consistency):} если в систему не поступают новые обновления, со временем все узлы придут к согласованному состоянию. Обеспечивает низкие задержки и высокую доступность.

\textbf{Согласованность на чтение собственных записей (read-your-writes):} пользователь всегда видит результаты собственных операций записи, что важно для удобства работы с системой.

\textbf{Монотонное чтение (monotonic reads):} последующие чтения не возвращают более старые версии данных.

\subsection{Соответствие нормативным требованиям}

Организации, работающие с персональными данными, должны соблюдать нормативные требования:

\textbf{GDPR (General Data Protection Regulation)}~\cite{gdpr} — регламент Европейского союза, устанавливающий строгие правила обработки персональных данных граждан ЕС. Ключевые требования: согласие на обработку данных, право на удаление данных («право быть забытым»), уведомление о нарушениях безопасности в течение 72 часов, назначение офицера по защите данных.

\textbf{Федеральный закон № 152-ФЗ} «О персональных данных» регулирует обработку персональных данных в Российской Федерации, устанавливая требования к организационным и техническим мерам защиты.

Технические меры защиты данных включают:
\begin{itemize}
  \item шифрование данных при хранении (at-rest) и передаче (in-transit);
  \item управление доступом на основе ролей (RBAC) и атрибутов (ABAC);
  \item аудит всех операций с данными;
  \item анонимизацию и псевдонимизацию персональных данных;
  \item регулярные оценки влияния на защиту данных (DPIA).
\end{itemize}

\section{Облачные подходы и онлайн-инструменты для работы с базами данных}

\subsection{Облачные вычисления и базы данных как услуга}

Облачные вычисления представляют модель предоставления IT-ресурсов по требованию через Интернет с оплатой по факту использования. Database as a Service (DBaaS) — это модель облачных вычислений, при которой провайдер берёт на себя администрирование, обслуживание и масштабирование баз данных, а заказчик фокусируется на разработке приложений и работе с данными.

Преимущества облачных баз данных:
\begin{itemize}
  \item снижение капитальных затрат: нет необходимости приобретать серверное оборудование;
  \item эластичность: ресурсы можно масштабировать в соответствии с текущей нагрузкой;
  \item автоматическое резервное копирование и восстановление;
  \item географическое распределение для снижения задержек;
  \item высокая доступность и автоматическое переключение при сбоях.
\end{itemize}

Недостатки включают:
\begin{itemize}
  \item зависимость от интернет-соединения и доступности провайдера;
  \item ограниченный контроль над инфраструктурой;
  \item риски безопасности при передаче данных третьей стороне;
  \item сложность миграции данных между провайдерами (vendor lock-in).
\end{itemize}

\subsection{Типы облачных хранилищ}

\textbf{Объектное хранилище} (Amazon S3, Google Cloud Storage, Azure Blob Storage) организует данные как объекты с метаданными и уникальными идентификаторами. Обеспечивает практически неограниченную масштабируемость и подходит для хранения неструктурированных данных: медиафайлов, резервных копий, архивов.

\textbf{Блочное хранилище} (Amazon EBS, Azure Disk Storage) предоставляет виртуальные диски для использования операционными системами и СУБД. Обеспечивает высокую производительность и низкие задержки, необходимые для транзакционных баз данных.

\textbf{Файловое хранилище} (Amazon EFS, Azure Files) предоставляет общую файловую систему, доступную множеству вычислительных узлов, что упрощает совместное использование данных.

\subsection{Облачные сервисы реляционных баз данных}

\textbf{Amazon RDS (Relational Database Service)} — управляемый сервис, поддерживающий несколько движков СУБД: MySQL, PostgreSQL, MariaDB, Oracle, Microsoft SQL Server. Автоматизирует резервное копирование, применение обновлений, репликацию и масштабирование.

\textbf{Amazon Aurora} — облачная СУБД, разработанная Amazon, совместимая с MySQL и PostgreSQL API, но использующая распределённую архитектуру хранения для обеспечения высокой производительности и отказоустойчивости.

\textbf{Google Cloud SQL} — управляемый сервис для MySQL, PostgreSQL и SQL Server, интегрированный с экосистемой Google Cloud.

\textbf{Azure SQL Database} — управляемая реляционная СУБД от Microsoft на основе SQL Server, предлагающая встроенный искусственный интеллект для автоматической оптимизации производительности.

\subsection{Облачные сервисы нереляционных баз данных}

\textbf{Amazon DynamoDB} — полностью управляемая NoSQL-база данных типа «ключ-значение» и документная, обеспечивающая микросекундные задержки при любом масштабе.

\textbf{Azure Cosmos DB} — глобально распределённая мультимодельная база данных, поддерживающая различные API (SQL, MongoDB, Cassandra, Gremlin), с гарантированно низкими задержками и настраиваемыми уровнями согласованности.

\textbf{Google Firestore} — документная NoSQL-база данных для мобильных, веб-приложений и серверной разработки, обеспечивающая автоматическую синхронизацию данных между клиентами.

\textbf{MongoDB Atlas} — полностью управляемый облачный сервис для MongoDB, доступный на платформах AWS, Azure и Google Cloud.

\subsection{Инструменты мониторинга и администрирования}

Эффективное управление базами данных требует постоянного мониторинга производительности и состояния системы. Современные инструменты включают:

\textbf{Системы мониторинга:} Prometheus~\cite{prometheus}, Grafana, Datadog, New Relic предоставляют сбор метрик, визуализацию и алертинг.

\textbf{Веб-интерфейсы администрирования:} phpMyAdmin (MySQL), pgAdmin (PostgreSQL), MongoDB Compass обеспечивают графический интерфейс для управления базами данных.

\textbf{Инструменты миграции схем:} Flyway, Liquibase позволяют версионировать схему базы данных и автоматизировать применение изменений.

\textbf{Инструменты DevOps:} Terraform, Ansible, Kubernetes операторы обеспечивают инфраструктуру как код (Infrastructure as Code) и автоматизацию развёртывания.

\subsection{Архитектуры обработки данных}

\textbf{ETL/ELT процессы:} Extract-Transform-Load (извлечение, преобразование, загрузка) — традиционный подход к интеграции данных. ELT (Extract-Load-Transform) — современный подход, при котором сырые данные сначала загружаются в целевое хранилище, а затем преобразуются с использованием его вычислительных мощностей.

\textbf{Data Lake vs Data Warehouse:} Data Lake (озеро данных) хранит сырые данные в исходном формате, обеспечивая гибкость для различных видов анализа. Data Warehouse (хранилище данных) содержит структурированные, очищенные данные, оптимизированные для бизнес-аналитики.

\textbf{Lambda и Kappa архитектуры:} подходы к обработке больших объёмов данных, сочетающие пакетную и потоковую обработку с использованием технологий вроде Apache Kafka~\cite{kafka}, Apache Spark, Apache Flink.

\section{Заключение}

Базы данных и системы управления ими являются критически важным компонентом современной информационной инфраструктуры. От эффективности организации данных зависит производительность приложений, надёжность систем и способность организаций извлекать ценность из накопленной информации.

Реляционная модель данных, предложенная Коддом более пятидесяти лет назад~\cite{codd1970}, сохраняет свою актуальность благодаря строгой математической основе, формальным механизмам целостности и универсальности языка SQL. Процесс нормализации обеспечивает устранение избыточности и предотвращение аномалий обновления. Реляционные СУБД остаются оптимальным выбором для транзакционных систем, требующих строгих гарантий ACID~\cite{gray1993, haerder1983}.

Вместе с тем, развитие веб-приложений и требования к обработке больших данных стимулировали появление нереляционных (NoSQL) технологий~\cite{sadalage2012}. Документные, колоночные, графовые базы данных и хранилища «ключ-значение» предлагают альтернативные модели данных, оптимизированные для специфических сценариев использования: высокая скорость записи, гибкие схемы, горизонтальное масштабирование, анализ связей.

Современная архитектура приложений часто основана на принципе полиглотной персистентности, комбинируя различные типы хранилищ данных в соответствии с требованиями различных компонентов системы. Выбор технологии определяется анализом требований к согласованности, доступности, производительности и масштабируемости.

Функциональные возможности СУБД существенно расширились: от базовых операций CRUD до сложных механизмов транзакционной обработки, репликации, шардирования, полнотекстового поиска, аналитики временных рядов. Язык SQL, несмотря на существование диалектов, остаётся универсальным инструментом взаимодействия с данными, постоянно развиваясь и включая новые возможности: оконные функции, работу с JSON, рекурсивные запросы.

Концепции целостности и надёжности данных приобретают особую важность в контексте распределённых систем. Теорема CAP и различные модели согласованности (от строгой до eventual consistency) предоставляют инструментарий для проектирования систем с требуемым балансом между производительностью и гарантиями. Алгоритмы консенсуса~\cite{ongaro2014, lamport1998} обеспечивают координацию в распределённых кластерах. Современные решения, такие как Google Spanner~\cite{corbett2012}, демонстрируют возможность достижения строгой согласованности в глобально распределённых системах.

Облачные технологии кардинально изменили подход к развёртыванию и эксплуатации баз данных. Модель Database as a Service (DBaaS) снижает барьеры входа, автоматизирует рутинные операции и обеспечивает эластичность ресурсов. Инструменты мониторинга~\cite{prometheus}, версионирования схем и автоматизации развёртывания становятся неотъемлемой частью современных DevOps-практик.

Соблюдение нормативных требований по защите персональных данных~\cite{gdpr} требует внедрения комплексных технических и организационных мер: шифрования, управления доступом, аудита, анонимизации данных. СУБД предоставляют встроенные механизмы для реализации этих мер, но ответственность за их корректную настройку и использование лежит на организациях.

Для специалистов в области интерактивных медиа и цифрового искусства понимание принципов организации баз данных особенно важно. Мультимедийные проекты часто работают с большими объёмами разнородных данных: метаданные медиафайлов, пользовательские взаимодействия, пространственно-временные характеристики контента. Эффективная архитектура данных, комбинирующая реляционные БД для метаданных, объектные хранилища для медиафайлов и специализированные NoSQL-решения для специфических задач, является основой масштабируемых и производительных систем.

Область баз данных продолжает активно развиваться. Направления исследований и разработок включают: NewSQL-системы, объединяющие преимущества реляционных и нереляционных подходов; базы данных in-memory для экстремально низких задержек; автоматическую оптимизацию с использованием машинного обучения; blockchain-технологии для обеспечения неизменяемости данных; квантовые базы данных. Специалисты в области информационных технологий должны постоянно следить за развитием технологий и адаптировать свои знания и навыки к изменяющимся требованиям индустрии.

% ----------------- список литературы -----------------
\printbibliography

\end{document}
