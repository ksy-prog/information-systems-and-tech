\documentclass{SibFU-docs}
\addbibresource{report.bib}

% мета-информация из класса
\newcommand{\Institute}{Гуманитарный институт}
\newcommand{\Department}{Прикладная информатика в искусстве и интерактивных медиа}
\newcommand{\WorkTitle}{Автоматизированные информационные системы и автоматизированные системы управления: концепции, архитектура, жизненный цикл, технологии, безопасность и примеры применения}
\newcommand{\Group}{ГФ25-02Б}

\begin{document}

\makecovertitle{\Institute}{\Department}{\WorkTitle}{\Group}

\begin{abstract}
В реферате комплексно рассмотрены вопросы создания и эксплуатации автоматизированных информационных систем (АИС) и автоматизированных систем управления (АСУ): их назначение, архитектурные подходы, жизненный цикл, современные технологические тенденции (облако, большие данные, искусственный интеллект, IoT, мобильные и low-code решения, RPA), аспекты безопасности и устойчивости, а также многоотраслевые примеры применения (промышленность, медицина, торговля, финансы, образование, госсектор, транспорт и сельское хозяйство). Цель работы — синтезировать конспектные материалы и источники \cite{petrov2023ais,ivanova2022management,smirnov2023trends,kuznetsova2022implementation,experts2023comparison,voronov2023architecture} в единую структуру для формирования целостного представления о современном состоянии и перспективах развития АИС/АСУ. Показана роль архитектуры, этапов жизненного цикла и интеграции технологий в повышении эффективности и надежности систем. Описаны ключевые вызовы безопасности и методы защиты. Приведены реальные отраслевые сценарии, демонстрирующие экономическую и социальную значимость автоматизации.
\end{abstract}

\tableofcontents

\section*{Введение}
\addcontentsline{toc}{section}{Введение}
Автоматизированные информационные системы (АИС) и автоматизированные системы управления (АСУ) являются фундаментом цифровой трансформации организаций. Их внедрение позволяет переходить от разрозненных ручных операций к управлению на основе данных, повышая прозрачность, скорость реакции и качество решений. В данной работе реферируются и синтезируются публикации и нормативные материалы, освещающие: базовые концепции и назначение АИС \cite{petrov2023ais,ivanova2022management}, архитектурные принципы и компоненты \cite{voronov2023architecture,kozlov2022components,belova2022integration}, методы и этапы жизненного цикла \cite{smirnov2023lifecycle,kuznetsov2022development,ivanova2023stages,popov2022management}, современные технологические тренды \cite{novikov2023automation,orlov2022cloud,petrov2023ai,sidorov2022iot,kuznetsova2023rpa,association2023trends,research2023analytics}, вопросы безопасности и устойчивости \cite{anderson2020security,nist2023framework,smith2022resilience}, а также отраслевые примеры применения \cite{johnson2023automation,rodriguez2022healthcare,smith2023retail}. Использование широкого спектра источников обеспечивает репрезентативность обзора и позволяет выявить взаимосвязь между архитектурой, технологиями, жизненным циклом и безопасностью. Цель реферата — сформировать целостную картину о современном состоянии АИС/АСУ, определить ключевые факторы успеха их внедрения и направления дальнейшего развития. Методологически работа опирается на сравнительно-аналитический подход: сопоставление концепций, практик и рекомендаций из разных источников для выделения общих закономерностей и лучших практик.

\section{Назначение и характеристики автоматизированных информационных систем}
Автоматизированные информационные системы внедряются для оптимизации работы с данными, устранения ручных рутинных операций и обеспечения целостности бизнес-процессов \cite{petrov2023ais,ivanova2022management}. Их ключевые цели включают: ускорение обработки, снижение ошибок, систематизацию и доступность информации, повышение прозрачности и управляемости. Актуальные исследования \cite{smirnov2023trends,kuznetsova2022implementation} подтверждают, что переход к АИС существенно повышает операционную эффективность.

\subsection{Эволюция понятия АИС}
От первых локальных автоматизированных рабочих мест (АРМ) бухгалтеров и операторов складов до современных экосистем — траектория развития АИС характеризуется ростом интеграционности и интеллектуальности. Переход от пакетной (batch) обработки к потоковой (stream) позволил системам работать с телеметрией и событиями в реальном времени, а распространение интерфейсов REST/GraphQL упростило построение композиции сервисов. Настоящее поколение АИС сочетает транзакционные контуры (ERP), аналитические платформы (Data Warehouse, Data Lakehouse), сервисы взаимодействия (CRM, порталы самообслуживания), подсистемы автоматизации процессов (BPM, RPA) и слой наблюдаемости (логирование, трассировка, метрики). Такая конвергенция формирует условия для Data-Driven управления и адаптивной оптимизации бизнес-процессов.

\subsection{Ключевые метрики эффективности}
Для мониторинга ценности внедрения формируют метрики (KPI): время цикла операции (Cycle Time), процент автоматизированных шагов процесса (Automation Ratio), доля повторно используемых наборов данных (Data Reuse Index), коэффициент качества данных (Data Quality Score: полнота, актуальность, консистентность), удовлетворенность пользователей (User Satisfaction), среднее время обнаружения и устранения инцидента (MTTD/MTTR). Сопоставление базовой и целевой метрик перед стартом проекта позволяет объективно измерять эффект и корректировать дорожную карту.

\subsection{Экономическая модель ценности}
Экономический эффект выражается через совокупную стоимость владения (TCO) и возврат инвестиций (ROI). Источники прямой экономии: сокращение трудозатрат, снижение ошибок и потерь, уменьшение страховых запасов. Косвенные выгоды: ускоренное принятие решений, рост удовлетворенности клиентов (NPS, Customer Effort Score), снижение штрафов комплаенса. Для учёта временной стоимости денег применяются NPV и IRR. Формирование «финансовой матрицы» связывает каждый функциональный блок АИС с предполагаемыми денежными потоками.

\subsection{Зрелость управления данными}
Без зрелой Data Governance АИС не достигает целевого уровня эффективности. Модель зрелости: начальный (данные фрагментированы), управляемый (определены владельцы наборов данных), стандартизованный (единые справочники, каталог, линейка качества), оптимизированный (активное улучшение качества, мониторинг показателей), адаптивный (предиктивная и прескриптивная аналитика с замкнутым контуром обратной связи). Переход между уровнями сопровождается внедрением ролей Data Steward, каталогов метаданных и процессов классификации (PII, конфиденциальность).

\subsection{Цели внедрения и практическая ценность}
Сокращение времени циклов процессов достигается за счет устранения повторного ввода данных и автоматизации формирования отчетов. Снижение количества ошибок опирается на единые справочники, контроль допустимых значений и встроенную валидацию. Упорядочивание данных формирует единое информационное пространство, позволяющее исключить локальные «острова автоматизации». Повышение контроля обеспечивается аудитом действий и возможностью быстрого построения нормативных показателей.

\subsection{Ключевые характеристики качества}
К полноте охвата относятся способность системы поддерживать все целевые процессы без привлечения внешних разрозненных средств. Точность отражает актуальность и достоверность данных, критичную для принятия решений (например, состояние складских запасов или остатки лекарств \cite{kuznetsova2022implementation}). Своевременность обеспечивается событиями реального времени и интеграцией с оборудованием/датчиками. Доступность проявляется в навигационной структуре, поиске и ролевом разграничении. Защищенность рассматривается в связке с безопасностью (см. раздел 5) и включает конфиденциальность, целостность и отказоустойчивость.

\subsection{Компонентный состав}
Архитектура базового уровня включает техническую (аппаратно-сетевую), программную (системное и прикладное ПО), информационную (данные, базы знаний, документооборот), организационную (регламенты, инструкции) и кадровую составляющие \cite{petrov2023ais,ivanova2022management}. Системы устойчивы только при гармоничном развитии всех компонентов: наличие продвинутого ПО бессмысленно без качественных данных и обученного персонала.

\subsection{Отрасли применения}
АИС применяются в бизнесе (CRM, ERP), производстве (АСУ ТП), медицине (МИС), образовании (LMS), государственном управлении (ГИС) \cite{experts2023comparison,smirnov2023trends}. Универсальность концепции делает их основой цифровой инфраструктуры.

\subsection{Преимущества и факторы выбора}
Преимущества: экономия времени, снижение затрат, повышение качества, улучшение контроля и конкурентные эффекты \cite{kuznetsova2022implementation}. Критерии выбора: соответствие задачам, простота использования, масштабируемость, поддержка, совокупная стоимость владения \cite{experts2023comparison}. Необходим баланс между функциональной насыщенностью и реальной полезностью.

\section{Архитектура и компоненты автоматизированных систем управления}
Архитектура АСУ определяет организацию потоков данных и распределение функций между уровнями \cite{voronov2023architecture,kozlov2022components}. Систематизация архитектурных паттернов позволяет снижать интеграционные риски и повышать адаптивность.

\subsection{Сравнительная таблица архитектурных подходов}
Для систематизации выбора архитектурной стратегии целесообразно использовать сравнительную матрицу характеристик:
\begin{table}[h!]
\centering
\caption{Сравнение архитектурных подходов}
\begin{tabular}{|p{3cm}|p{3.2cm}|p{3.2cm}|p{3.2cm}|p{3.2cm}|}
\hline
	\textbf{Критерий} & \textbf{Монолит} & \textbf{Микросервисы} & \textbf{Событийная / потоковая} & \textbf{Гибрид (edge+cloud)} \\ \hline
Масштабируемость & Вертикальная, ограниченная & Горизонтальная по сервисам & Горизонтальная + выделение потоков & Гибкая: локально критичное / облако аналитика \\ \hline
Сложность управления & Низкая в начале / высокая при росте & Средняя / требует оркестрации & Повышенная (управление событиями) & Повышенная (координация слоёв) \\ \hline
Устойчивость к отказам & Низкая (единая точка) & Высокая (изоляция сервисов) & Высокая (обработка событий) & Очень высокая (дублирование уровней) \\ \hline
Время вывода изменений & Более длительное при большом коде & Быстрое для отдельных сервисов & Быстрое, но требует тестирования потоков & Зависит от синхронизации уровней \\ \hline
Транзакционная целостность & Простая & Сложнее (Saga, 2PC нежелателен) & Сложнее (eventual consistency) & Комбинированные стратегии \\ \hline
Наблюдаемость & Локальные логи & Распределённые трассировки & Метрики потоков + трассировки & Многоуровневая наблюдаемость \\ \hline
Типовые сценарии & Небольшие приложения, начальный этап & Быстрорастущие домены & Высокая частота событий (IoT, телеметрия) & Критичные real-time + аналитика \\ \hline
\end{tabular}
\end{table}


\subsection{Микросервисная и событийная декомпозиция}
Микросервисы обеспечивают независимое масштабирование и ускоряют релизные циклы: сервис телеметрии, сервис нормализации сигналов, сервис прогнозного анализа, сервис управляющих правил, сервис визуализации. Событийная шина (Kafka, NATS, Pulsar) поддерживает publish/subscribe и replay механизмы. Паттерн Event Sourcing фиксирует каждое изменение состояния, упрощая аудит и восстановление. CQRS разделяет потоки команд (изменяют состояние) и запросов (чтение оптимизированных проекций).

\subsection{Цифровые двойники}
Цифровой двойник агрегирует структурные (топология), поведенческие (динамика показателей), контекстные (условия эксплуатации) данные. На его основе симулируют сценарии: изменение режимов нагрева печи, перераспределение логистических потоков, оптимизация графика обслуживания оборудования. Связь двойника с реальными потоками обеспечивает сравнение эталонной и фактической траекторий и раннее обнаружение отклонений.

\subsection{Паттерны надежности и устойчивости}
Circuit Breaker предотвращает каскадные провалы, Bulkhead сегментирует ресурсы, Retry с экспоненциальной задержкой снижает нагрузку, Idempotency гарантирует корректность повторных команд, Graceful Degradation сохраняет критический функционал при отказе вспомогательных компонентов. В совокупности они формируют «архитектурный каркас отказоустойчивости».

\subsection{DataOps и интеграционный слой}
DataOps внедряет практики CI/CD для конвейеров данных: автоматические тесты трансформаций, верификация схем (Schema Registry), мониторинг дельты качества. Интеграционный слой предоставляет унифицированные API (REST/GraphQL/gRPC) и событийные каналы для публикации изменений.

\subsection{Базовые функциональные блоки}
Типовая декомпозиция: блок сбора (датчики, счетчики, сканеры) — блок обработки (серверы анализа) — блок принятия решений (алгоритмы, контроллеры) — пользовательский интерфейс (панели, веб-клиенты) — хранилище данных (БД, озера данных) \cite{belova2022integration}. Непрерывный цикл: измерение → нормализация → анализ → команда → визуализация → архив.

\subsection{Взаимодействие и потоки}
Сенсорные данные поступают по промышленным протоколам (Modbus, OPC UA) в агрегирующие шлюзы, далее в аналитическое ядро, где формируются управляющие решения (например, коррекция температурного режима в климатической системе). Интерфейсы операторов отображают KPI и события.

\subsection{Архитектурные стили}
Централизованный (монолит) — простой деплой, но ограниченная масштабируемость. Децентрализованный — автономия подсистем, снижает риски единой точки отказа. Клиент–сервер — баланс нагрузки и управляемость. Облачный — гибкая масштабируемость и географическая независимость \cite{nikolaev2023trends,ministry2023requirements}. Современная тенденция — гибридные мультиярусные архитектуры, объединяющие edge и облако.

\subsection{Принципы проектирования}
Рост (масштабируемость), стабильность, защита, адаптивность, совместимость \cite{voronov2023architecture,belova2022integration}. Дополнительно: наблюдаемость (логирование, трассировка), управляемость конфигураций (Infrastructure as Code), модульность.

\subsection{Отраслевые различия}
Промышленность задействует многоуровневую пирамиду: поле (датчики), уровень контроллеров (PLC/SCADA), MES, ERP. Коммерческий сектор — сервисно-ориентированные и микросервисные архитектуры. Автоматизация зданий — федерация автономных подсистем (HVAC, охрана) со шиной интеграции \cite{belova2022integration}.

\section{Жизненный цикл автоматизированных систем}
Жизненный цикл охватывает полный путь системы от идеи до вывода из эксплуатации \cite{smirnov2023lifecycle,kuznetsov2022development}. Управление жизненным циклом снижает совокупную стоимость владения и риски отказов.

\subsection{Управление портфелем инициатив}
На входе формируется реестр инициатив с атрибутами: ценность (Value Score), сложность (Complexity Score), риск (Risk Exposure), регуляторная обязательность (Compliance Flag). Приоритезация производится матрицей Value vs Complexity и анализом зависимости проектов (Dependency Graph).

\subsection{Model-Based Systems Engineering (MBSE)}
MBSE интегрирует требования, функциональные модели, сценарии поведения и ограничения производительности в единую модель (SysML). Это повышает трассируемость: каждое требование связано с элементом архитектуры и тестовым случаем.

\subsection{Эволюционное развитие и управление изменениями}
После первой поставки запускается цикл улучшений: сбор пользовательского фидбэка, анализ журналов использования (Usage Analytics), выявление «узких мест» (Hotspots). Change Management применяет классификацию: стандартные изменения (плановые), существенные (требуют оценки влияния), срочные (инцидентные). Автоматизированные конвейеры тестов сокращают MTTR.

\subsection{Этапы жизненного цикла}
Инициация и планирование (анализ потребностей, ТЗ) \cite{ivanova2023stages}; проектирование (архитектура, прототипы); разработка (код, интеграция модулей); внедрение (настройка среды, миграция данных); эксплуатация (операционное использование, мониторинг); поддержка и развитие (обновления, оптимизация); завершение (архивация данных, утилизация ресурсов) \cite{popov2022management}.

\subsection{Детализация ключевых стадий}
На фазе планирования критично качественное выявление требований с участием стейкхолдеров — снижает риск «ползучего» расширения функционала. Проектирование включает моделирование сценариев нагрузки и угроз безопасности (security by design). Разработка опирается на стандарты кодирования и автоматизированное тестирование (CI/CD). Внедрение — поэтапный rollout (пилот → тиражирование). Эксплуатация — непрерывный мониторинг метрик производительности и инцидентов. Поддержка — управление техдолгом и регулярные релизы. Завершение — контролируемый вывод: сохранение исторических данных и оценка достигнутых KPI.

\subsection{Методологии и управление}
Гибкие подходы (Agile, DevOps) сокращают цикл обратной связи и ускоряют адаптацию \cite{experts2023methodology}. Для критичных отраслей применяются комбинированные модели (V-model + итеративные спринты) для строгой верификации.

\subsection{Значение управления жизненным циклом}
Комплексное управление обеспечивает предсказуемость бюджета, улучшает качество и ускоряет время вывода функционала на рынок \cite{smirnov2023lifecycle}. Системное документирование позволяет эффективнее передавать знания и минимизирует риски при смене команды.

\section{Современные технологии и программные решения}
Технологический стек стремительно расширяется: облачные платформы, большие данные, ИИ, IoT, мобильные решения, ERP/CRM эволюционируют, появляются low-code и RPA \cite{novikov2023automation,orlov2022cloud,petrov2023ai,sidorov2022iot,kuznetsova2023rpa,association2023trends,research2023analytics}.

\subsection{Облачные технологии}
Облако обеспечивает эластичность ресурсов, модель оплаты по факту использования и ускорение доставки функционала. Факторы выбора: SLA, геолокация дата-центров, поддержка контейнеризации и оркестрации.

\subsection{Большие данные и аналитика}
Платформы обработки (Hadoop/Spark) позволяют выявлять корреляции в операционных данных, прогнозировать спрос и оптимизировать цепочки поставок. Интеграция с BI ускоряет принятие решений \cite{research2023analytics}.

\subsection{Искусственный интеллект}
ИИ расширяет возможности: прогнозирование отказов оборудования, интеллектуальная маршрутизация заявок, адаптивное ценообразование \cite{petrov2023ai}. Модели машинного обучения требуют MLOps-подхода (версионирование данных/моделей, контроль дрейфа).

\subsection{MLOps и жизненный цикл модели}
Жизненный цикл: подготовка данных → экспериментирование (tracking параметров) → упаковка артефактов → развёртывание (batch, online, edge) → мониторинг (качество, дрейф, производительность) → переобучение. Feature Store повышает повторное использование признаков, Model Registry управляет версиями, Explainability повышает доверие к решениям.

\subsection{Edge Computing}
В сценариях низкой латентности (промышленные линии, транспорт) часть логики переносится на edge-устройства, где выполняется локальная фильтрация, агрегация и первичная аналитика. Это снижает трафик и повышает приватность, передавая в центр только релевантные события.

\subsection{DevSecOps и Zero Trust}
DevSecOps интегрирует анализ уязвимостей в конвейер разработки: SAST, DAST, секрет-сканирование, композиционный анализ зависимостей. Zero Trust вводит принцип непрерывной проверки доверия и микросегментацию, минимизируя влияние компрометации отдельного узла.

\subsection{Интернет вещей}
IoT создаёт непрерывный поток телеметрии с оборудования \cite{sidorov2022iot}, позволяя переходить к предиктивному обслуживанию. Edge-обработка снижает задержки и нагрузку на центральное хранилище.

\subsection{Мобильные и удалённый доступ}
Мобильные интерфейсы повышают оперативность управления и поддерживают распределённые команды. Ключевые требования: офлайн-режим, безопасная аутентификация, адаптивная UI.

\subsection{ERP нового поколения}
Модульность и API-интеграции упрощают постепенное расширение функциональности, повышают совместимость с экосистемой \cite{novikov2023automation}. Встраивание аналитики и ИИ — тренд к «умным» операционным ядрам.

\subsection{Интеллектуальные CRM}
CRM эволюционирует в платформу управления отношениями на всем пути клиента, включая анализ намерений и автоматизацию коммуникаций.

\subsection{Low-code / No-code}
Ускоряют разработку, уменьшая барьер между бизнесом и ИТ. Риски: теневая ИТ, сложность масштабирования и контроля качества. Митигируется governance-политиками.

\subsection{Robotic Process Automation (RPA)}
RPA автоматизирует высокорутинные сценарии (перенос данных, формирование документов) \cite{kuznetsova2023rpa}. В сочетании с ИИ формирует интеллектуальную автоматизацию (сбор → классификация → принятие решений).

\subsection{Serverless и функция как сервис}
Модель FaaS (Function as a Service) позволяет запускать изолированные функции по событию без управления инфраструктурой. Применимо для нерегулярных задач: генерация отчётов по расписанию, обработка входящих файлов, вебхуков из внешних сервисов. Ограничения: холодный старт, ограничения по времени выполнения, необходимость адаптации к stateless-подходу.

\subsection{Data Mesh и федеративное владение данными}
Data Mesh — организационная парадигма распределения ответственности за доменные наборы данных между кросс-функциональными командами. Доменные «продукты данных» описываются контрактами (Data Product Contract) с SLO по качеству и доступности. Центрально поддерживаются платформенные возможности: каталог, безопасность, инфраструктурные шаблоны.

\subsection{AIOps и интеллектуальное наблюдение}
AIOps применяет машинное обучение для корреляции событий инфраструктуры, сокращая шум оповещений и ускоряя определение первопричины инцидентов. Используются алгоритмы кластеризации логов, обнаружения аномалий метрик и автоматического предложения ремедиационных действий.

\subsection{Compliance Automation}
Автоматизация комплаенса включает отслеживание конфигураций (Configuration Drift Detection), проверку соответствия политик безопасности, генерацию отчётности для аудиторов. Политики описываются декларативно (Policy as Code), интегрируются в CI/CD.

\subsection{Observability Stack}
Наблюдаемость строится на трёх столпах: метрики (Prometheus/OpenMetrics), логи (ELK/OpenSearch), трассировки (OpenTelemetry). Добавление событий бизнес-процессов формирует четвёртый уровень — Business Observability, позволяющий связывать инфраструктурные аномалии с влиянием на KPI.

\subsection{Энергоэффективность и устойчивость инфраструктуры}
В контексте ESG цели оптимизации ресурсов (энергопотребление, углеродный след) становятся функциональными требованиями. Метрики: PUE (коэффициент эффективности использования энергии ЦОД), Carbon Intensity, Water Usage. Алгоритмы перераспределяют нагрузки между зонами с более низким углеродным профилем.

\subsection{API Management и интеграционные шлюзы}
Управление API включает версионирование, лимитирование (Rate Limiting), авторизацию (OAuth2/OIDC), мониторинг использования. Интеграционные шлюзы выполняют трансформацию протоколов, кэширование, инспекцию безопасности и централизуют политики.

\subsection{Таблица технологических трендов}
\begin{table}[h!]
\centering
\caption{Технологические тренды и их влияние}
\begin{tabular}{|p{3cm}|p{4.2cm}|p{4.2cm}|p{4.2cm}|}
\hline
	\textbf{Тренд} & \textbf{Ценность} & \textbf{Риски} & \textbf{Митигирующие меры} \\ \hline
Облако & Эластичность, TTM & Зависимость от провайдера & Мульти‑облако, абстракции \\ \hline
Искусственный интеллект & Прогноз, оптимизация & Предвзятость, дрейф & Мониторинг качества, Explainability \\ \hline
Data Mesh & Ускорение масштабирования данных & Фрагментация стандартов & Общие платформенные сервисы \\ \hline
Serverless & Снижение операционных затрат & Холодные старты & Прогрев, смешанная архитектура \\ \hline
RPA + ИИ & Авто рутины, скорость & Ошибки классификации & Контроль качества, журналация \\ \hline
Zero Trust & Снижение периметр-рисков & Сложность внедрения & Пошаговое внедрение, автоматизация политик \\ \hline
ESG оптимизация & Снижение затрат, репутация & Стоимость измерений & Автоматический сбор метрик \\ \hline
\end{tabular}
\end{table}


\subsection{Тенденции развития}
Рост доли прогнозной и автономной автоматизации, интеграция 5G, расширение практик цифровых двойников, углубление персонализации интерфейсов \cite{association2023trends}.

\section{Безопасность и устойчивость функционирования}
Безопасность — ключевой слой качества АИС/АСУ, покрывающий конфиденциальность, целостность, доступность и устойчивость \cite{anderson2020security,nist2023framework}. Угрожающая среда динамична, требуется многоуровневый подход.

\subsection{Типология угроз}
Внешние атаки (эксплуатация уязвимостей, фишинг), внутренние (умышленное или случайное нарушение), технические сбои, программные дефекты, человеческий фактор \cite{anderson2020security}. Каждая категория требует специфических мер.

\subsection{Матричный анализ рисков}
Риск оценивается по двум осям: вероятность (Rare → Almost Certain) и влияние (Minor → Severe). Формируется тепловая карта (Heat Map), где красные зоны требуют немедленных мер (сегментация сети, усиление мониторинга), жёлтые — планового снижения, зелёные — принятия. Для высоких рисков назначаются владельцы и показатели снижения (Risk Reduction KPI).

\subsection{Метрики безопасности}
MTTD, MTTR, Coverage Security Tests, Vulnerability Remediation Time, Patch Compliance Rate — ядро измерительной рамки. Сокращение совокупной экспозиции уязвимостей отражает эффективность процессов.

\subsection{Управление ключами и криптография}
Применение Hardware Security Module (HSM) повышает защищенность операций создания/хранения ключей. Ротация ключей и сертификатов автоматизируется (Certificate Lifecycle Management). Подготовка к постквантовой криптографии включает инвентаризацию используемых алгоритмов и пилотирование гибридных схем.

\subsection{Этика и конфиденциальность}
Принципы: минимизация (сбор только необходимых данных), прозрачность (информирование о целях обработки), объяснимость (доступность логики принятия автоматизированных решений), защита от предвзятости (Bias Mitigation). Реестр обработок и механизмы отзыва согласия повышают доверие пользователей.

\subsection{План непрерывности и аварийное восстановление}
BCP описывает критические процессы, целевые RTO/RPO, альтернативные площадки. Disaster Recovery тестируется через сценарии имитации отказа (Game Days) и периодическое восстановление из резервных копий (Restore Drills). Наличие чётких планов снижает масштаб бизнес-простоя.

\subsection{Меры защиты}
Технический уровень: сегментация сети, WAF, IDS/IPS, шифрование, принцип минимальных привилегий, Zero Trust. Организационный: политики, регламенты, обучение персонала. Физический: контроль доступа, видеонаблюдение. Административный: аудит, журналирование, управление учётными записями \cite{nist2023framework}.

\subsection{Zero Trust Reference Model}
Слой идентичности (Identity Provider), слой устройств (Device Posture), сеть (Micro-Segmentation), приложения (Policy Enforcement Points), данные (Classification, Encryption). Политики динамически оценивают контекст: местоположение, тип устройства, риск-профиль.

\subsection{Automation of Threat Detection}
Использование поведенческой аналитики (UEBA) и корреляции событий SIEM позволяет автоматически выявлять отклонения паттернов доступа. Интеграция с SOAR ускоряет реакцию за счёт автоматизированных плейбуков ремедиации.

\subsection{Security Champions и обучение}
Практика Security Champions встроена в продуктовые команды: ответственные за продвижение безопасных практик, ревью архитектур и кода, взаимодействие с центральной группой безопасности. Это снижает время закрытия уязвимостей.

\subsection{Privacy by Design}
Принципы интеграции приватности: дефолтная защита (privacy by default), минимизация хранения, сегрегация данных, псевдонимизация, настройка гранулярного доступа и регулярная переоценка чувствительности.

\subsection{Таблица мер безопасности}
\begin{table}[h!]
\centering
\caption{Многоуровневые меры безопасности}
\begin{tabular}{|p{3cm}|p{4cm}|p{4cm}|p{4cm}|}
\hline
	\textbf{Уровень} & \textbf{Примеры мер} & \textbf{Цель} & \textbf{Метрики} \\ \hline
Сеть & Сегментация, VPN, IDS/IPS & Снижение поверхности атаки & Кол-во сегментов, инциденты на сегмент \\ \hline
Идентичность & MFA, адаптивная аутентификация & Устойчивость к компрометации акаунтов & Процент MFA, неудачные входы \\ \hline
Приложения & SAST/DAST, WAF, секрет-сканирование & Раннее выявление уязвимостей & Время ремедиации, плотность дефектов \\ \hline
Данные & Шифрование, DLP, классификация & Предотвращение утечек & Процент классифицированных наборов \\ \hline
Операции & SOAR, журналы, алёрт корреляция & Сокращение времени реагирования & MTTD, MTTR \\ \hline
\end{tabular}
\end{table}


\subsection{Отказоустойчивость и непрерывность}
Резервирование компонентов (кластеризация, репликация БД), регулярные бэкапы с проверкой восстановления, план аварийного реагирования (DRP), тестирование сценариев отказа, наблюдаемость и SLO.

\subsection{Текущие вызовы и тренды}
Расширение атак на цепочки поставок, применение ИИ злоумышленниками, рост количества IoT-устройств увеличивает поверхность атаки, переход в облако требует корректного контроля границ периметра. Ответ: Security by Design, DevSecOps, поведенческая аналитика, шифрование повсеместно (data-at-rest / in-transit).

\section{Примеры применения в отраслях}
Многоотраслевые кейсы демонстрируют экономические и качественные эффекты от автоматизации \cite{johnson2023automation,johnson2023automation,rodriguez2022healthcare,smith2023retail}. Ниже приведён обзор.

\subsection{Промышленность}
Роботизированные сборочные линии, SCADA мониторинг, предиктивное обслуживание сокращают простои и повышают стандартизацию качества.

Дополнительное направление — внедрение Process Mining поверх журналов событий для выявления фактических траекторий процесса и сравнения их с эталонными моделями, что ускоряет оптимизацию. Цифровые двойники линий выпуска позволяют тестировать изменения параметров без риска для реального производства.

\subsection{Медицина}
МИС обеспечивают целостные электронные истории, поддерживают телемедицину, интеграцию диагностических изображений, анализа и лекарственных назначений \cite{rodriguez2022healthcare}.

\subsection{Торговля и логистика}
Автоматизация складских операций (WMS), динамическое пополнение запасов, персонализированные рекомендации в e-commerce \cite{smith2023retail}. Логистические платформы оптимизируют маршруты.

Дополнительно: геофенсинг контролирует соблюдение маршрута перевозки, IoT-датчики температуры и влажности поддерживают условия хранения, а предиктивные алгоритмы ETA повышают точность доставки. Omni-channel интеграция объединяет физические магазины и онлайн-платформы в единую систему.

\subsection{Финансы}
Высокопроизводительные платёжные шлюзы, антифрод-аналитика в реальном времени, роботизированная обработка заявок.

Адаптивные AML модели сокращают ложные срабатывания, а распределенные реестры (DLT) повышают прозрачность расчётов. Алгоритмическая торговля опирается на низколатентные каналы и комплексные модели риска.

\subsection{Образование}
LMS управляют учебным контентом, отслеживают прогресс; адаптивные курсы формируют индивидуальные траектории.

Дополнительно внедряются виртуальные лаборатории и симуляторы, позволяющие получать практический опыт без физического оборудования. Аналитика вовлеченности (Engagement Analytics) выявляет студентов в зоне риска и инициирует проактивную поддержку.

\subsection{Государственный сектор}
Порталы госуслуг сокращают транзакционные издержки, повышают прозрачность процедур, внедрение межведомственного обмена данными ускоряет сервисы.

Развитие включает открытые данные (Open Data) и CivicTech платформы для участия граждан в принятии решений. Автоматизация комплаенса снижает вероятность коррупционных нарушений.

\subsection{Транспорт}
Системы управления движением, интеллектуальное распределение мощности инфраструктуры, прогнозирование загруженности.

Умные транспортные системы интегрируют данные видеонаблюдения, погодных датчиков и мобильных устройств, формируя адаптивное управление потоками. Цифровые двойники дорожных сетей повышают точность планирования ремонтов.

\subsection{Сельское хозяйство}
Точное земледелие (датчики почвы, дроны) оптимизирует использование удобрений, повышает урожайность и экологическую устойчивость.

Спектральный анализ состояния растений и прогноз фитопатогенов позволяют вовремя корректировать агротехнику. Модели оптимизации ирригации сокращают водопотребление и углеродный след производства.

\subsection{Дополнительные межотраслевые аспекты}
	\textbf{Интероперабельность}: стандарты обмена (FHIR в медицине, OPC UA в промышленности) снижают затраты интеграции. \textbf{Гибридные сценарии}: объединение реального времени (операционное управление) и пакетной аналитики (стратегическое планирование). \textbf{Ценность данных}: вторичное монетизирование обезличенных агрегатов (Data Marketplace) при соблюдении приватности.

\subsection{Экологическая и социальная устойчивость}
АИС поддерживают расчёт углеродного следа, мониторинг выбросов, оптимизацию потребления ресурсов; социальные аспекты — доступность сервисов, инклюзивность интерфейсов (Accessibility), прозрачность алгоритмов принятия решений.

\subsection{Показатели зрелости внедрения}
Модель зрелости внедрения: 1) Локальная автоматизация; 2) Интеграция основных процессов; 3) Сквозная оркестрация; 4) Предиктивное управление; 5) Автономная оптимизация. Переход сопровождается увеличением показателя автоматизации и сокращением среднего времени цикла.

\section*{Глоссарий}
\addcontentsline{toc}{section}{Глоссарий}
\begin{description}
	\item[АИС] Автоматизированная информационная система.
	\item[АСУ] Автоматизированная система управления.
	\item[SCADA] Supervisory Control And Data Acquisition — система диспетчерского управления.
	\item[DevSecOps] Практики интеграции безопасности в цикл разработки и эксплуатации.
	\item[Data Mesh] Децентрализованная модель владения и публикации данных как продуктов.
	\item[Digital Twin] Цифровой двойник физического объекта или процесса.
	\item[Feature Store] Репозиторий управляемых признаков для ML моделей.
	\item[UEBA] User and Entity Behavior Analytics — аналитика поведенческих аномалий.
	\item[SOAR] Security Orchestration, Automation and Response — платформа автоматизации реакций.
	\item[PUE] Power Usage Effectiveness — метрика эффективности энергопотребления ЦОД.
\end{description}

\section*{Список сокращений}
\addcontentsline{toc}{section}{Список сокращений}
\begin{tabular}{p{3cm}p{12cm}}
API & Application Programming Interface\\
CI/CD & Continuous Integration / Continuous Delivery\\
KPI & Key Performance Indicator\\
ROI & Return On Investment\\
NPV & Net Present Value\\
IRR & Internal Rate of Return\\
MTTD & Mean Time To Detect\\
MTTR & Mean Time To Respond/Repair\\
RTO/RPO & Recovery Time / Recovery Point Objective\\
ESG & Environmental, Social, Governance\\
AML & Anti-Money Laundering\\
BCP & Business Continuity Planning\\
DR & Disaster Recovery\\
SLA & Service Level Agreement\\
SLO & Service Level Objective\\
HSM & Hardware Security Module\\
DLT & Distributed Ledger Technology\\
FaaS & Function as a Service\\
WAF & Web Application Firewall\\
IDS/IPS & Intrusion Detection/Prevention System\\
MFA & Multi-Factor Authentication\\
BI & Business Intelligence\\
ETL/ELT & Extract Transform Load / Extract Load Transform\\
\end{tabular}


Спектральный анализ состояния растений и прогноз фитопатогенов позволяют вовремя корректировать агротехнику. Модели оптимизации ирригации сокращают водопотребление и углеродный след производства.

\section*{Заключение}
\addcontentsline{toc}{section}{Заключение}
В ходе реферата продемонстрирована взаимосвязь между архитектурой, жизненным циклом, технологическим стеком и безопасностью автоматизированных информационных систем и систем управления. Ключевые факторы успешности внедрения: корректно собранные требования, модульная масштабируемая архитектура, опора на данные и аналитику, встроенная безопасность (security by design), непрерывное развитие (DevOps/DevSecOps), адаптация к отраслевым особенностям. Современные тренды (облако, ИИ, IoT, RPA, low-code) ускоряют цикл создания ценности и демократизируют разработку, одновременно расширяя поверхность атаки и усложняя управление качеством.

Дальнейшее развитие АИС/АСУ будет связано с усилением роли цифровых двойников, автономных агентов управления, расширением практик MLOps для масштабируемого жизненного цикла моделей и интеграцией превентивной поведенческой безопасности (UEBA). Внедрение Zero Trust, квантово-устойчивой криптографии и расширенной аналитики устойчивости (Sustainability Metrics) станет стандартом. Комплексное управление данными и рисками позволит извлекать максимальную ценность при контролируемой совокупной стоимости владения и приемлемом профиле операционных и кибернетических рисков.

\printbibliography

\end{document}
